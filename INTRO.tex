\pagestyle{plain}

\chapter*{Introdução\smallskip\subtitulo{O manancial subterrâneo\break de Dostoiévski}}
\addcontentsline{toc}{chapter}{Introdução, \textit{por Lucas Simone}}

\begin{flushright}
\textsc{lucas simone}
\end{flushright}

Publicada na primeira metade de 1864, a novela \textit{Memórias do subsolo}
inaugurou uma nova fase na produção literária de Fiódor Dostoiévski. Nas
amargas reminiscências do rancoroso homem subterrâneo, o autor afasta"-se
sensivelmente das obras de sua chamada ``primeira fase'', tão caras à crítica e
ao público russos durante os anos 1840--50. O \textit{romance social} --- que tem em
\textit{Gente pobre} seu primeiro e mais relevante exemplo --- cedia terreno a
uma nova experimentação na prosa dostoievskiana, ao tipo de criação artística
que seria mais tarde interpretada como a prefiguração do existencialismo do
século~\textsc{xx}.

Geralmente lidas como o embrião dos extensos romances dos ``anos miraculosos'',
como \textit{Crime e castigo} e \textit{Os irmãos Karamázov}, as
\textit{Memórias} têm, no entanto, vida própria. A obra desvela a
transfiguração que se operou em Dostoiévski após seu retorno do longo exílio
siberiano, sua mudança de atitude em relação ao futuro da Rússia e ao papel do
escritor na sociedade.  Reflete, ainda, as agruras de sua atribulada vida
pessoal, em meio à doença de sua primeira esposa Maria --- que acabaria falecendo
pouco tempo depois ---, à culpa gerada por suas aventuras extraconjugais, às
constantes e crescentes dívidas. Mas principalmente, refletem, com especial
intensidade, as inquietações e os debates dos conturbados anos 1860.

De além das fronteiras russas, notícias de hostilidades chegavam por todos os
lados: a invasão francesa do México (1861--67), a Guerra Civil Americana
(1861--65), as sucessivas guerras na Europa, que envolveram todas as potências
continentais e aceleraram os violentos processos de unificação da Itália e da
Alemanha. Um mundo em constante transformação. O turbilhão não deixaria de
afetar a Rússia: após a humilhante derrota na Guerra da Crimeia (1853--56) e a
morte do severo tsar Nikolai~\textsc{i} --- responsável, anos antes, pela condenação de
Dostoiévski ao degredo ---, o país, agora governado por Aleksandr~\textsc{ii},
experimentou um breve período de euforia e relativa liberdade política, em que
se testemunhou o fim do regime da servidão, em 1861, além de uma série de
reformas em praticamente todas as esferas da vida russa. Data desse período uma
nova safra de periódicos voltados para a literatura, como a revista
\textit{Vriêmia}, dos irmãos Mikhail e Fiódor Dostoiévski, e seu sucessor
imediato, o \textit{Epokha}, publicação em que viria à luz o texto original das
\textit{Memórias}. Além dos jornais e revistas locais, textos oriundos da
Europa Ocidental também ajudavam a moldar a nova paisagem intelectual do país.
O ano de 1864, por exemplo, testemunhou a primeira tradução russa de \textit{A
origem das espécies}, de Charles Darwin (1809--82), e novas ideias no campo da
filosofia --- como o utilitarismo, o positivismo etc. --- chegavam em levas sucessivas à Rússia.

Em meio a tão profundas mudanças, é especialmente relevante para a literatura
russa do período --- e para a obra de Dostoiévski em particular --- o florescimento
de um novo movimento político: o dos \textit{naródniki}, os populistas.
Preocupados com o futuro do campesinato, após a emancipação, e com o avanço do
capitalismo na Rússia, pensadores como Aleksandr Herzen (1812--70), Nikolai
Tchernychevski (1828--89), Mikhail Bakunin (1814--76) e Serguei Netcháiev
(1847--82) propuseram diferentes maneiras de derrubar o tsarismo e instaurar uma
sociedade igualitária. Assim, diferentemente dos intelectuais dos anos 1840, 
a geração dos sessenta preocupava"-se mais com a ação que com a mera
produção de literatura “engajada”. O surgimento dessa nova safra da
\textit{intelligentsia} gerou o tópico mais efervescente do debate político
russo com a publicação, em 1862, do romance \textit{Pais e filhos}, de Ivan
Turguêniev (1818--83): seu protagonista, o estudante Bazárov, personificava não
apenas o recém"-surgido \textit{niilismo} --- alcunha difundida justamente pelo
romance ---, mas também o racionalismo cientificista que apenas engatinhava na
Rússia de então. \textit{Pais e filhos} causou um imenso impacto na opinião
pública da época, entre conservadores e liberais, \textit{eslavófilos} e
\textit{ocidentalizantes}: já no ano seguinte, em 1863, Tchernychevski,
encarcerado na fortaleza de Pedro e Paulo, escreveria \textit{O que fazer?},
romance mais emblemático e influente da geração revolucionária dos anos 1860 e
uma resposta ao livro de Turguêniev. O livro narra as aventuras da heroína
Viera Pávlovna Rozálskaia em sua busca incessante por uma vida melhor para si e
para os que a rodeiam. A personagem mais emblemática da trama, porém, viria a
ser o jovem revolucionário Rakhmiétov, que simbolizaria o ser ideal que todos
os homens tinham potencial de se tornar.

É evidente que Dostoiévski não haveria de se eximir de tal debate. Após a
publicação de \textit{A vila de Stepántchikovo e seus habitantes} (1859),
\textit{O sonho do titio} (1859) e \textit{Humilhados e ofendidos} (1861), seu
nome voltava a ser ouvido com frequência nos círculos literários de São
Petersburgo. Porém, logo tornou"-se claro --- especialmente após a publicação de
\textit{Notas de inverno sobre impressões de verão} (1863) --- que o Dostoiévski
que retornara à capital em 1860 não tinha quase nada em comum com o jovem que,
havia pouco mais de uma década, fora incensado pelo crítico Vissarion Bielinski
(1811--48) como “o novo Gógol”. À época, o jovem Fiódor Mikháilovitch fora
condenado à morte pelo tsar por conta de sua participação no círculo liberal de
Mikhail Petrachevski (1821--66), grupo que contava com intelectuais, escritores
e até militares de baixa patente, e que se reunia para debater obras da
literatura e do pensamento político ocidental (então proibidas por Nikolai~\textsc{i}).
Embora o círculo fosse bastante heterogêneo e não possuísse nenhum programa de
ação definido, seus integrantes foram vítimas da histeria que se abateu sobre as
monarquias absolutas --- e sobre a russa em especial --- após as revoluções de
1848. A execução em praça pública de vinte e um de seus membros --- Dostoiévski
inclusive --- revelou"-se uma encenação, e a maioria teve a pena capital comutada
para anos de trabalhos forçados na Sibéria.  Contudo, a prisão, a falsa
execução e o exílio não levariam Dostoiévski a aderir à nova onda de
mobilização política dos anos 60. Ao contrário: o escritor encararia o exílio
como uma espécie de penitência, um processo de purificação de seu passado
pecaminoso. Afirmaria mais tarde que o cárcere fora sua salvação, ideia central
de seu \textit{opus magnum}, \textit{Crime e castigo} (1866). Ainda assim, a
experiência deixaria profundas marcas psicológicas em Fiódor Mikháilovitch,
servindo de base para o intenso \textit{Recordações da casa dos mortos}
(1860--2) e ecoando também no famoso monólogo a respeito da pena capital
proferido pelo Príncipe Míchkin em \textit{O idiota} (1868--9). Ademais, na
Sibéria, o contato íntimo com os reclusos --- em sua maioria camponeses
miseráveis --- levou Dostoiévski à conclusão de que, aos olhos da população mais
humilde, pouca ou nenhuma diferença havia entre os intelectuais engajados que
lutavam por sua emancipação e os senhores de terras que buscavam impedi"-la. Um
lento processo de transformação operou"-se em Dostoiévski; dele, finalmente
emergiu um homem cujas posições políticas pareceriam extremamente conservadoras
no meio literário dos anos 1860.

Remonta, portanto, aos tempos de exílio o início de seu interesse pelo
\textit{pótchvennitchestvo},\footnote{Em tradução livre, \textit{o retorno ao solo}.}
corrente de pensamento --- por vezes considerada uma variante menos radical da
eslavofilia --- que defendia o protagonismo do povo russo no processo que levaria
à redenção de todo o mundo. A salvação não passaria pelo utilitarismo e pelo
racionalismo da facção socialista da \textit{intelligentsia}, nem pela total
submissão à autocracia ou pela completa negação do Ocidente: o caminho seria o
do retorno às raízes nacionais e aos ideais cristãos, de uma maior integração
entre os setores populares e as camadas instruídas da sociedade, vistas à época
pelos \textit{pótchvenniki} como quase completamente \textit{ocidentalizadas} e
\textit{alheias ao solo} russo. Esse ideário, embora explorado de maneira mais
aprofundada em seu romance \textit{Os demônios}, de 1872, já se fazia de certa
forma presente nas \textit{Memórias}.

A maior polêmica do \textit{Subsolo}, no entanto, é mesmo com os intelectuais
socialistas em geral e com Nikolai Tchernychevski em particular: as referências
a \textit{O que fazer?}, na maioria das vezes quase literais, transbordam nas
agressivas e contundentes invectivas do monologuista ao longo da primeira parte
do livro. Não à toa, a novela foi lida e analisada, na Rússia, com muito mais
frequência pelo prisma do embate político do que por suas qualidades como obra
literária, e teve, logo após sua publicação, uma repercussão relativamente
discreta: gerou como resposta apenas uma paródia, escrita em maio de 1864 por
um de seus mais acalorados interlocutores, o escritor Mikhail Saltykov"-Shchedrin
(1826--89). Já Apollon Grigóriev (1822--64), importante crítico literário e um
dos maiores admiradores de Dostoiévski, veio a público, ainda em 1864, para
defendê"-lo, elogiando a tessitura da novela e sua profundidade.

Mas foi somente à luz de seus romances tardios que as \textit{Memórias do
subsolo} chamariam novamente a atenção do público e da crítica.  Primeiramente
na Rússia, e aos poucos também no Ocidente, a obra começou a ser lida como um
preâmbulo aos grandes romances dos últimos anos de Dostoiévski. Vassili Rózanov
(1856--1919), por exemplo, um dos mais polêmicos pensadores da virada do século
--- e que chegou a intitular a si mesmo de \textit{o homem do subsolo} ---, foi
possivelmente o primeiro crítico a ver na novela a chave para a compreensão dos
romances tardios de Dostoiévski: a irracionalidade, defendida tão
apaixonadamente pelo monologuista do subsolo, seria um elemento
intrínseco à natureza humana, e a razão jamais teria condições, por si só, de
versar sobre temas como a felicidade e o amor, que lhe seriam completamente
alheios. Rózanov --- que ironicamente haveria de se casar com Apollinária
Súslova, ex"-amante de Dostoiévski e protótipo de personagens como Polina de
\textit{O jogador} (1866) e Nastássia Filíppovna, de \textit{O idiota} --- via no
autor das \textit{Memórias} uma espécie de profeta, e por meio de seu ensaio \textit{A
lenda do Grande Inquisidor} (1906) ajudaria a impulsionar ainda mais seu
prestígio. O status de grande escritor nacional foi também consolidado pela
ascensão, na virada do século~\textsc{xx}, do movimento simbolista na Rússia, que
compartilhava muitas das premissas do escritor. Dostoiévski serviria de
influência decisiva para diversos dos representantes dessa nova tendência, como
Nikolai Minski (1855--1937) e Innokenti Ánnenski (1855--1909), entre outros.

O século~\textsc{xx}, porém, viu a obra de Dostoiévski quase sempre distante da
unanimidade em seu país. Ao longo de todo o período soviético --- a despeito de
certas concessões à primeira fase da produção dostoievskiana ---, o autor foi
relegado pela crítica oficial a um segundo patamar na plêiade dos gigantes da
literatura oitocentista. Foi apenas no período do \textit{degelo} --- nos anos
1950 --- e, posteriormente, durante a \textit{perestroika} --- já nos derradeiros
dias da União Soviética ---, que Dostoiévski ensaiou um retorno à galeria dos
grandes autores nacionais. A Rússia pós"-soviética testemunhou, nos últimos
vinte anos, um gradativo processo de revisão e resgate da obra dostoievskiana
por parte de público e crítica, além do ressurgimento do diálogo entre o autor
das \textit{Memórias} e os novos nomes da literatura russa contemporânea, como
por exemplo Vladímir Sorókin.

Se na pátria os Novecentos foram um tanto negligentes com sua memória
literária, no Ocidente Fiódor Mikháilovitch foi pouco a pouco sendo reconhecido
como um dos maiores romancistas de todos os tempos e como epítome de boa
literatura. Inicialmente em traduções pouquíssimo confiáveis --- que chegavam a
oferecer uma terceira parte das \textit{Memórias}, criadas inteiramente por seu
editor francês ---, Dostoiévski foi lido com afinco em toda a Europa Ocidental e
também no Brasil já nos primeiros anos após sua morte. Na virada do século~\textsc{xx},
versões mais próximas do texto original começaram a vir à luz, como a tradução
inglesa assinada por Constance Garnett. Esses trabalhos, no
entanto, prezavam por sua “domesticação” da prosa muitas vezes truncada que
caracteriza a pena dostoievskiana. Seu estilo, com frequência descrito até
como rude pelo leitor russo, vem aos poucos se tornando mais claro e familiar
para o público ocidental: as escolas mais contemporâneas de tradução têm
buscado uma maior fidelidade ao original, e não apenas um texto elegante na
língua de chegada, mas infiel às características mais essenciais de seu autor.

De todo modo, tais problemas tradutológicos jamais impediram que as dimensões
psicológicas e filosóficas da obra de Dostoiévski --- especialmente nas
\textit{Memórias do subsolo} --- tivessem grande importância para escritores e
pensadores do Ocidente. De Franz Kafka a André Gide; da
psicanálise de Sigmund Freud ao existencialismo de Jean"-Paul Sartre; 
do teatro pós"-moderno ao cinema americano: muitos foram os que
beberam --- e ainda bebem --- do manancial subterrâneo de Fiódor Mikháilovitch
Dostoiévski.



