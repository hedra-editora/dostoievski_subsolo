\part{Memórias do subsolo}

{
	
\mbox{}
\vfill
\thispagestyle{empty}

\small\textit{Tanto o autor das memórias como as próprias \emph{memórias} são, é claro,
inventados. Contudo, pessoas como o escritor dessas memórias não apenas podem
como devem existir em nossa sociedade, levando"-se em consideração as
circunstâncias em que se formou nossa sociedade. Quis mostrar ao público, de
maneira mais nítida que a habitual, um dos tipos de um passado recente.
Trata"-se de um dos representantes de uma geração que ainda não desapareceu.
Neste trecho, intitulado \emph{O subsolo}, a pessoa apresenta a si mesma e suas
opiniões, e tenta explicar os motivos pelos quais surgiu e deveria surgir em
nosso meio. No trecho seguinte, temos já as \emph{memórias} propriamente ditas dessa
pessoa a respeito de alguns acontecimentos de sua vida.}

}

\chapter{O subsolo}

\section{parte I}

Sou um homem doente\ldots{} Sou um homem mau. Um homem desinteressante. Acho
que sofro do fígado. Na verdade, eu não entendo absolutamente nada da
minha doença e não sei exatamente do que sofro. Não me trato e jamais
me tratei, embora respeite a medicina e os médicos. Além disso, sou
extremamente supersticioso; bem, pelo menos o suficiente para respeitar
a medicina. (Sou instruído o bastante para não ser supersticioso, mas
sou supersticioso.) Não, senhor, é de raiva que eu não quero me tratar.
Isso, certamente, não haverão de entender. Pois bem, mas eu entendo.
Obviamente, não ousarei explicar quem exatamente quero atingir, nesse
caso, com a minha raiva; sei perfeitamente bem que de maneira alguma
poderei \textit{denegrir} os médicos pelo fato de não me tratar com eles; sei
melhor que qualquer um que com tudo isso prejudicarei apenas a mim
mesmo e a mais ninguém. Mas mesmo assim, se eu não me trato é de raiva.
Se sofro do fígado, pois que sofra ainda mais!

Vivo assim já faz tempo: uns vinte anos. Agora tenho quarenta.
Antigamente trabalhava no serviço público, mas agora já não mais. Era
um funcionário mau. Era grosseiro e sentia prazer nisso. Já que eu não
aceitava propinas, então pelo menos aquilo deveria me recompensar. (Um
mau chiste; mas não vou apagá"-lo. Eu o escrevi achando que sairia muito
sagaz; mas agora que eu mesmo vi que queria apenas me mostrar de
maneira ignóbil, não vou apagar, de propósito!) Quando às vezes alguém
se aproximava da mesa em que eu trabalhava para pedir algo, eu mostrava
os dentes para eles e sentia um prazer indescritível quando conseguia
desagradar alguém. E quase sempre conseguia. A maioria eram pessoas
tímidas; é claro, afinal estavam pedindo. Mas dentre os almofadinhas
havia um oficial em especial que eu não suportava. Ele não queria se
submeter de modo algum e ficava tilintando o sabre de maneira
repugnante. Ficamos um ano e meio em guerra por causa daquele sabre. Eu
finalmente venci. Ele parou de tilintar o sabre. Mas isso foi na época
da minha juventude. Mas sabem os senhores em que consistia o ponto mais
importante de minha raiva? Toda a questão estava justamente nisso, a
maior obscenidade resumia"-se a esse fato: de que constantemente, mesmo
nos momentos do mais intenso mau humor, eu tinha a vergonhosa
consciência de que não apenas não era mau, como nem sequer era um homem
amargo, que somente fazia tipo a troco de nada e me divertia com isso.
Eu parecia espumar de raiva, mas se alguém me trouxesse algum
presentinho e me desse uma xícara de chá com um pouquinho de açúcar eu
talvez me acalmasse. Ficaria até comovido, muito embora depois,
possivelmente, mostraria os dentes para mim mesmo e de vergonha
sofreria de insônia por uns meses. Tal era meu hábito.

Menti agora há pouco quando disse que era um funcionário mau. Menti de
raiva. Eu apenas aprontava com as pessoas que vinham pedir e com o
oficial, mas na verdade nunca consegui me tornar mau. Eu a todo momento
me dava conta de que havia em mim muitíssimos elementos que se opunham
a isso. Eu os sentia fervilhando em mim, esses elementos que se
opunham. Sabia que por toda a minha vida eles iriam fervilhar em mim e
iriam querer vir à tona, mas eu não deixava, eu, de propósito, não os
deixava vir à tona. Eles me torturavam, me envergonhavam; me faziam ter
convulsões e finalmente me cansaram, como me cansaram! Mas não lhes
parece, senhores, que agora pareço me arrepender em sua presença, que
eu peço desculpa a vocês por algum motivo\ldots{}? Tenho certeza de que
pensam assim\ldots{} Mas também, garanto que para mim tanto faz se pensam\ldots{}

Não era apenas papel de mau que eu não conseguia fazer, mas de qualquer
coisa; nem de mau, nem de bom, nem de canalha. Nem de pessoa honrada,
nem de herói, nem de inseto. Agora, vou vivendo em meu canto, tentando
infundir em mim mesmo o consolo, maldoso e completamente inútil, de que
uma pessoa inteligente não pode, a sério, se fazer passar por nada, mas
de que somente um idiota se faz passar por algo. Sim, um homem
inteligente do século dezenove não só deve como é moralmente obrigado a ser
um sujeito de preferência sem caráter; uma pessoa de caráter, de ação,
é preferencialmente limitada. Essa é a minha convicção dos quarenta
anos. Tenho agora quarenta anos, e quarenta anos é uma vida inteira; é
a mais extrema velhice. Viver mais do que quarenta anos é indecente,
vulgar, imoral! Quem vive mais do que quarenta anos? Respondam
sinceramente, honestamente. Eu direi a vocês quem: os idiotas e os
miseráveis, são eles que vivem. Eu digo isso na cara de qualquer
velhote, de todos esses velhinhos honrados, de todos esses velhinhos
cheirosos de cabelos prateados! Digo na cara do mundo todo! Tenho o
direito de falar assim, porque eu mesmo viverei até os sessenta anos.
Viverei até os setenta anos! Viverei até os oitenta anos\ldots{}! Esperem!
Deixem"-me tomar fôlego\ldots{}

Certamente vocês estão achando, senhores, que eu quero fazê"-los rir. Se
enganam também nisso. Eu não sou em absoluto essa pessoa divertida que
vocês acham que eu sou ou que vocês devem achar que eu sou; aliás, se
vocês, irritados com toda essa tagarelice (e eu já sinto que estão
irritados), inventarem de me perguntar \textit{quem exatamente é você?}, então
eu responderia: sou um assessor colegiado.\footnote{ Cargo de oitavo
grau do serviço público da Rússia Imperial.} Entrei no serviço público
para ter com que comer (mas apenas para isso), e quando, no ano
passado, um parente distante deixou"-me seis mil rublos de herança, pedi
imediatamente dispensa e me enfiei no meu canto. Antes eu também vivia
nesse canto, mas agora me enfiei de vez nele. Meu apartamento é uma
porcaria, é nojento, na periferia da cidade. Minha empregada é uma
senhora do interior, velha e má de tão estúpida e ainda por cima cheira
sempre mal. Vivem me dizendo que o clima de São Petersburgo ainda me
fará mal e que os meus recursos insignificantes não bastam para viver
na cidade. Eu sei de tudo isso, sei melhor que todos esses conselheiros
e juízes, tão experientes e tão sábios. Mas eu ficarei em São
Petersburgo; eu não sairei de São Petersburgo! Eu não sairei porque\ldots{}
Ah! Mas dá na mesma se eu sairei ou não.

Aliás, do que pode falar um homem digno com supremo prazer?

A resposta: de si mesmo.

Então, vou falar de mim mesmo.


\section{parte II}

Quero agora contar a vocês, senhores, quer queiram ouvir, quer não
queiram, por que sequer por inseto eu conseguia passar. Direi a vocês
solenemente que eu muitas vezes quis passar por inseto. Mas sequer
disso eu era digno. Juro a vocês, senhores, que ter muita consciência é
uma doença; uma verdadeira e perfeita doença. Para o uso cotidiano
seria plenamente suficiente uma consciência humana comum, ou seja, a
metade, um quarto a menos que a porção que cabe a um homem evoluído em
nosso triste século dezenove e que, além disso, tenha a excepcional
infelicidade de morar em Petersburgo, a cidade mais abstrata e
intencional de todo o globo terrestre. (Existem cidades intencionais e
não intencionais.) Seria perfeitamente suficiente, por exemplo, a
consciência baseada na qual vivem todos os homens ditos imediatos e de
ação. Aposto que vocês estão pensando que eu estou escrevendo tudo isso
por ostentação, para fazer gracejos com os homens de ação, e ainda que
por uma ostentação de mau gosto estou tilintando o sabre como o tal
oficial. Mas senhores, quem é que pode vangloriar"-se de suas doenças e
ainda ostentá"-las?


Mas pensando bem, o que estou dizendo? Todos fazem isso; vangloriam"-se
de suas doenças e eu possivelmente ainda mais que todos. Não vamos
discutir; minha objeção é absurda. Mas mesmo assim tenho a forte
convicção de que não apenas ter muita consciência, mas até mesmo ter
qualquer nível de consciência é uma doença. Insisto nisso. Deixemos
isso por um momento. Digam"-me o seguinte: por que é que acontecia, como
se fosse de propósito, exatamente naqueles momentos, sim, exatamente
naqueles momentos em que eu me sentia mais capaz de sentir todas as
nuances de \textit{tudo que há de belo e sublime}, como falavam entre nós na
época, de justamente eu não apenas sentir, mas cometer esses atos
indecorosos, esses atos que\ldots{} bem, resumindo, que todos talvez façam,
mas que, como se fosse de propósito, me ocorriam justamente quando eu
mais sentia que eles eram desnecessários? Quanto mais consciência eu
tinha do bem e do tal ``belo e sublime'', mais profundamente eu me
afundava no meu lodo e mais capaz eu era de atolar nele completamente.
Mas a questão principal era que tudo isso não acontecia em mim por
acaso; era como se tivesse que ser assim. Era como se fosse minha
condição mais normal, não era de forma alguma uma doença ou um defeito,
de maneira que, afinal, até perdi a vontade de lutar contra esse
defeito. No fim das contas eu por pouco não acreditei (ou talvez na
verdade tenha acreditado) que, talvez, essa de fato era a minha
condição normal. Mas de início, no começo, que torturas não suportei
nessa luta! Eu não acreditava que era assim com os outros, e por isso
por toda a minha vida ocultei esse fato a meu respeito como se fosse um
segredo. Eu me envergonhava (talvez até agora me envergonhe); cheguei
ao ponto de sentir uma certa satisfaçãozinha secreta, anormal e infame
de voltar, então, em mais uma madrugada nojenta de Petersburgo, para o
meu canto e ter a forte consciência de que naquele dia novamente fizera
alguma obscenidade, de que o que havia sido feito novamente não podia
ser desfeito, e no âmago, secretamente, roer"-me, roer"-me por conta
daquilo, fustigar"-me e atormentar"-me até o ponto em que a amargura
tornava"-se finalmente numa doçura vergonhosa e maldita, e, finalmente,
num definitivo e verdadeiro prazer! Sim, num prazer, num prazer!
Insisto nisso. É por isso que eu comecei a dizer que cada vez mais
quero saber se de fato outras pessoas experimentam esse mesmo prazer.
Eu explico a vocês: o prazer vinha justamente da consciência demasiado
clara de minha degradação; do fato de perceber afinal que já chegara ao
fundo do poço; de que aquilo era detestável, mas de que não poderia ser
de nenhuma outra maneira; de que não havia mais saída, de que jamais
conseguiria tornar"-me outra pessoa; de que se pelo menos ainda restasse
tempo e fé para me transformar em qualquer outra coisa, então
certamente não iria querer me transformar; e de que, se quisesse,
acabaria não fazendo nada, porque na verdade talvez não houvesse em que
se transformar. Mas o principal, afinal, é que tudo isso ocorre de
acordo com as leis normais e fundamentais de uma consciência
hipertrofiada e de acordo com a inércia que decorre diretamente dessas
leis e, por conseguinte, não só não se pode transformar"-se mas
simplesmente não se pode fazer nada. Resulta, assim, dessa consciência
hipertrofiada que: está certo em ser um canalha, como se fosse um
consolo para um canalha perceber que de fato é um canalha. Mas basta\ldots{}
Pois é, disse um monte de asneiras, mas expliquei o quê\ldots{}? Como se
explica esse prazer? Mas eu explicarei! Vou dar um jeito de ir até o
final! Foi para isso que tomei a pena para escrever\ldots{}

Eu, por exemplo, sou terrivelmente cheio de amor"-próprio. Sou
desconfiado e melindroso, como um corcunda ou um anão, mas juro que
havia certos momentos em que, se acontecesse de me darem um bofetão,
ficaria até feliz com isso. Estou falando sério: possivelmente eu
conseguiria encontrar até nisso o meu tipo de prazer; o prazer do
desespero é claro, mas era no desespero que eu encontrava os mais
pungentes prazeres, especialmente quando tinha a forte consciência do
impasse em que eu me encontrava. E no momento do bofetão, nesse momento
me sentiria é esmagado pela consciência da massa a que eu seria
reduzido. Mas acima de tudo, não importa o quanto se estenda o assunto,
ainda assim eu acabo sempre sendo o maior culpado de tudo e, o que é
mais lastimável, sem ter culpa e, por assim dizer, de acordo com as
leis da natureza. Em primeiro lugar, sou culpado de ser mais
inteligente que todos ao meu redor. (Eu sempre me considerei mais
inteligente que todos ao meu redor e às vezes, acreditem ou não, até me
envergonhava disso. Pelo menos a minha vida toda eu como que virava o
rosto e nunca pude olhar as pessoas nos olhos.) E finalmente, sou
culpado pelo fato de que, se houvesse em mim qualquer generosidade, ela
seria para mim apenas mais torturante, pela consciência de sua
inutilidade. Afinal, eu certamente não saberia o que fazer com a minha
generosidade: nem perdoar, já que o ofensor possivelmente teria me
agredido de acordo com as leis da natureza, e não se pode perdoar as
leis da natureza; nem esquecer, já que podem até ser as leis da
natureza, mas são, mesmo assim, ofensivas. Finalmente, mesmo se eu
não quisesse em absoluto ser generoso, mas ao contrário, se desejasse
me vingar do ofensor, não poderia me vingar de ninguém por nenhum
motivo, porque certamente não me decidiria por fazer qualquer coisa,
mesmo que eu pudesse. E por que motivo não me decidiria? Sobre isso eu
gostaria de dizer duas palavrinhas à parte.


\section{parte III}

E, por exemplo, com as pessoas que sabem se vingar e, em geral, se defender: como
é que se faz? Porque uma vez que, suponhamos, o sentimento de vingança toma
conta delas, nesse momento já não resta mais nada em toda a sua existência além
desse sentimento. Um senhor como esse voa em direção a seu objetivo como um
touro enfurecido, com os chifres voltados para baixo, e apenas um muro talvez o
contenha. (A propósito: diante de um muro, esses senhores, os homens imediatos
e de ação, dão"-se sinceramente por vencidos. Para eles, o muro não é um desvio
como para nós, pessoas que pensam e que, por conseguinte, nada
fazem; não é um pretexto para se desviar do caminho, um pretexto no qual
geralmente nós mesmos não acreditamos, mas que sempre nos alegra muito. Não,
elas se dão por vencidas com toda a sinceridade. O muro tem para eles algo
reconfortante, algo moralmente decisivo, definitivo, talvez até mesmo algo
místico\ldots{} Mas voltemos ao muro mais tarde.) Pois bem, eu considero esse
homem imediato como um homem verdadeiramente normal, tal como gostaria de ver a
mais carinhosa mãe: a natureza, que tão amavelmente o fez surgir sobre a terra.
Eu invejo esse homem ao extremo. Ele é estúpido, não discutirei com vocês a
esse respeito, mas, talvez, um homem normal deva mesmo ser estúpido, como vocês
podem saber? Talvez isso seja até muito bonito. E fico tanto mais convencido
dessa, por assim dizer, desconfiança, porque se, por exemplo, tomarmos a
antítese do homem normal, ou seja, o homem de consciência hipertrofiada, saído,
é evidente, não do ventre da natureza, mas sim de um tubo de ensaio (isso já é
quase misticismo, senhores, mas eu desconfio até disso), esse homem artificial
a tal ponto às vezes se submete a sua antítese que, com toda a sua consciência
hipertrofiada, considera"-se genuinamente um rato, e não um homem. Um rato de
consciência hipertrofiada, de fato; mas ainda assim é um rato, enquanto o outro
é um homem, e por conseguinte\ldots{} todo o resto também. E além de tudo, ele
próprio, ele mesmo se considera um rato; ninguém pede a ele que o faça; e este
é um ponto importante. Vejamos agora esse rato em ação. Suponhamos, por
exemplo, que ele também esteja ofendido (e ele quase sempre está ofendido) e
também deseje se vingar.  Nele, talvez, a raiva acumula"-se ainda mais que no
\textit{homme de la nature et de la vérité}.\footnote{ Referência a
Jean"-Jacques Rousseau (1712--1778), possivelmente inspirada na maneira pela qual
o poeta alemão Heinrich Heine (1797--1856) a ele se referia.} Nele, talvez, o
desejozinho baixo e torpe de retribuir o ofensor com o mesmo mal corrói com
ainda mais força que no \textit{homme de la nature et de la vérité}, porque o
\textit{homme de la nature et de la vérité}, em sua estupidez inata, considera
sua vingança pura e simplesmente como justiça; já o rato, devido a sua
consciência hipertrofiada, nega essa justiça. Chega finalmente à questão em si,
ao ato de desforra propriamente dito. O infeliz rato, além da única torpeza
original, conseguiu já amontoar ao seu redor, na forma de questões e dúvidas,
diversas outras torpezas; uma única dúvida suscitou tantas questões não
resolvidas que, contra a sua vontade, a seu redor juntou"-se uma pasta funesta,
uma sujeira fétida, composta de suas dúvidas, inquietações e, finalmente, do
escarro lançado sobre ele pelos homens imediatos e de ação, que se colocam
solenemente ao redor como juízes e ditadores, e que riem dele em alto e bom
som. É claro, resta a ele fazer um gesto de desdém com sua patinha e, com um
sorriso de desprezo, no qual ele mesmo não acredita, esgueirar"-se
vergonhosamente para sua frestinha. Lá, em seu subsolo abjeto e fétido, nosso
rato, ofendido, abatido e ridicularizado, rapidamente afunda"-se num rancor
frio, venenoso e, principalmente, eterno. Por quarenta anos seguidos irá se
lembrar até dos menores e mais vergonhosos detalhes da sua ofensa, e, além
disso, a cada vez acrescentar por conta própria detalhes ainda mais
vergonhosos, maldosamente provocando"-se e irritando"-se com sua própria
fantasia. Ele próprio haverá de se envergonhar de sua fantasia, mas mesmo assim
continuará lembrando, remoendo, inventará para si um monte de histórias, sob o
pretexto de que elas também poderiam acontecer, e não perdoará nada. Talvez
comece a querer se vingar, mas de maneira inconstante, com ninharias, por
detrás do fogão, incógnito, sem crer nem em seu direito de vingar"-se, nem no
êxito de sua vingança, e sabendo de antemão que por todas as suas tentativas de
vingar"-se sofrerá ele mesmo cem vezes mais do que aquele de quem quer se
vingar, enquanto o outro talvez nem sequer se importe. No leito de morte
novamente se lembrará de tudo, com os juros acumulados depois de todo esse
tempo e\ldots{} Mas é justamente nesse semidesespero frio e asqueroso, nessa
semicrença, nesse sepultamento consciente de si mesmo, ainda em vida, movido
pelo desespero, por quarenta anos no subsolo, nesse seu impasse insolúvel,
criado forçadamente mas mesmo assim em parte duvidoso, em todo esse veneno de
desejos não realizados e que se interiorizaram, em toda essa hesitação febril,
de decisões tomadas para sempre e do arrependimento que depois de um minuto
novamente surge: é nisso que se encontra a nata desse estranho prazer de que
falava. Ele é a tal ponto sutil, a tal ponto às vezes escapa à consciência que
as pessoas um pouquinho limitadas ou até mesmo simplesmente as pessoas com
nervos de aço não perceberão dele nada de nada. ``Talvez também não entendam'',
acrescentarão vocês de sua parte com um sorriso, ``aqueles que nunca levaram um
bofetão''. E desta forma insinuam a mim polidamente que talvez em minha vida eu
também tenha levado um bofetão, e que por isso falo com conhecimento de causa.
Aposto que vocês pensam isso. Mas acalmem"-se, senhores, eu não levei um
bofetão, embora a mim seja de todo indiferente o que vocês pensem a esse
respeito. Talvez eu até mesmo lamente o fato de que em minha vida tenha
distribuído poucos bofetões. Mas basta, nem uma palavra mais sobre esse tema
que para vocês é extremamente interessante.

Continuarei tranquilamente a falar das pessoas com nervos de aço, que
não entendem certas sutilezas dos prazeres. Esses senhores, por
exemplo, em alguns casos, embora cheguem a mugir como touros, com toda
a força de suas gargantas, embora isso, suponhamos, venha a trazer"-lhes
uma grandessíssima honra, no entanto, como eu já disse, diante da
impossibilidade eles imediatamente se submetem. A impossibilidade
significa dizer um muro de pedra? Que muro de pedra? É claro que falo
das leis da natureza, das conclusões das ciências naturais, da
matemática. Se provarem, por exemplo, que você veio do macaco, não há
por que torcer o nariz, aceite como é. Se provarem a você que, na
essência, uma gotinha de sua própria gordura deve ser"-lhe mais cara do
que cem mil outros semelhantes a você e que, como resultado disso, no
fim serão solucionadas todas as tão faladas virtudes e obrigações e
outras maluquices e preconceitos, então aceite as coisas dessa maneira,
não há o que fazer, porque dois e dois são quatro, isso é matemática.
Tente objetar.

``Mas perdão'', vocês gritarão. ``É impossível rebelar"-se: dois e dois são
quatro! A natureza não lhe presta contas; ela não tem nada a ver com
seus desejos e com o fato de você gostar ou não gostar de suas leis.
Você é obrigado a aceitá"-la como ela é, e por conseguinte, também todos
os seus resultados. Ou seja, um muro é um muro\ldots{} etc. etc.'' Senhor
Deus, que tenho eu a ver com as leis da natureza e com a aritmética
quando eu, por algum motivo, não gosto dessas leis e do dois e dois são
quatro? É claro que eu não romperei esse muro com a testa se de fato
não tiver forças para rompê"-lo, mas eu não me conformarei com ele
apenas pelo fato de que é um muro de pedra e me faltam as forças para
tanto.

Como se esse muro de pedra fosse de fato um consolo e de fato encerrasse
em si pelo menos alguma palavra de conciliação, simplesmente porque
dois e dois são quatro. Oh, é o absurdo dos absurdos! É muito melhor
compreender tudo, ter consciência de tudo, todas as impossibilidades e
muros de pedra; não se conformar com nenhuma dessas impossibilidades e
desses muros de pedra, se lhe é repulsivo conformar"-se; chegar, por
meio das mais inevitáveis combinações lógicas, às mais repugnantes
conclusões a respeito do eterno tema de que até pelo muro de pedra se é
de alguma maneira culpado, embora novamente seja mais que evidente que
não se é culpado em absoluto, e, como consequência disso, rangendo os
dentes em silêncio e sem forças, imobilizar"-se voluptuosamente na
inércia, devaneando com o fato de que não há, afinal, com quem
enfurecer"-se; de que não se pode encontrar um objeto, e que talvez
jamais se possa encontrar; de que tudo não passa de adulteração, de
roubo, de trapaça, não passa de um monte de bobagens; não se sabe o
que, não se sabe quem, mas, a despeito de todas essas incertezas e
trapaças, ainda assim se sente dor, e quanto menos se sabe, mais dor se
sente!


\section{parte IV}

--- Ha"-ha"-ha! Depois disso você vai encontrar prazer até numa dor de
dente! --- gritarão vocês rindo.

--- E por que não? Até numa dor de dente há prazer --- responderei eu. ---
Tive dores de dente por um mês inteiro; sei que há prazer. É claro que
nesse caso não é em silêncio que sentem raiva, mas gemendo; e esses
gemidos não são sinceros, são gemidos maliciosos, e é na malícia que
está toda a questão. É nesses gemidos que se expressa o prazer daquele
que sofre; se não sentisse neles prazer, sequer começaria a gemer. É um
bom exemplo, senhores, e vou desenvolvê"-lo. É nesses gemidos que se
expressa, em primeiro lugar, toda a inutilidade --- tão humilhante para
nossa consciência --- dessa dor; toda a legitimidade da natureza, com a
qual vocês, é claro, não se importam, mas devido à qual vocês mesmo
assim sofrem, enquanto ela não. Expressa"-se a consciência de que não
existe para vocês um inimigo, mas que a dor existe; a consciência de
que vocês, apesar de todos os Wagenheims\footnote{ Referência a uma destacada 
família de dentistas de São Petersburgo,
cujos anúncios se espalhavam por toda a cidade durante a década de
1860.} possíveis, são completamente escravos de seus dentes; de que, se alguém
quiser, seus dentes irão parar de doer, mas se não quiserem, eles
doerão por mais três meses; e finalmente de que, se vocês ainda assim
não concordarem e continuarem a protestar, restará a vocês --- para
consolo próprio --- apenas castigar a si mesmos ou socar com ainda mais
força esse seu muro, e absolutamente mais nada. Pois é dessas ofensas
sangrentas, é dessas zombarias --- não se sabe de quem --- que começa
finalmente o prazer, que chega às vezes à mais extrema voluptuosidade.
Peço a vocês, senhores, que em algum momento ouçam com atenção os
gemidos de um homem instruído do século  dezenove que sofra dos dentes, no
segundo ou no terceiro dia de dor, quando ele começa a gemer não tanto
quanto gemia no primeiro dia, ou seja, não simplesmente pelo fato de
que os dentes doem; não como algum rude mujique, mas da maneira como
geme um homem tocado pelo progresso e pela civilização europeia, como
um homem que \textit{renunciou a sua terra e suas origens populares}, como se
diz hoje em dia. Seus gemidos tornam"-se como que desagradáveis,
imundos, raivosos, e duram vários dias e várias noites. Mas ele mesmo
sabe que com seus gemidos não trará nenhuma vantagem; sabe melhor que
ninguém que prejudica e irrita sem motivo os outros e a si mesmo; sabe
que até mesmo o público diante do qual ele se empenha e toda a sua
família já passaram a ouvi"-lo com repulsa, não acreditam nem um pouco
nele e pensam consigo mesmos que ele poderia agir de outra maneira,
simplesmente gemer, sem trinados e sem esquisitices, e que tudo não
passa de uma brincadeira, motivada pela raiva e pela malícia. Mas é
justamente em toda essa consciência e vergonha que se encontra a
voluptuosidade. ``Eu os incomodo'', diz, ``perturbo seus corações, não
deixo ninguém dormir na casa. Pois então não durmam, sintam vocês
também a cada minuto a minha dor nos dentes. Já não sou para vocês o
herói que antes pretendia parecer, mas simplesmente um homenzinho vil,
um mandrião. Pois que seja! Fico muito contente que vocês tenham me
decifrado. Acham desagradável ouvir meus gemidinhos infames? Então que
seja desagradável; pois é agora que darei um trinado ainda mais
desagradável\ldots{}'' Mesmo agora não entendem, senhores? Não, nota"-se que
são necessários um profundo desenvolvimento e uma total tomada de
consciência para entender todos os meandros dessa voluptuosidade! Vocês
riem? Fico muito contente. Meus chistes, senhores, são, é claro, de mau
tom, incoerentes, confusos, sem autoconfiança. Mas isso vem do fato de
que eu mesmo não me respeito. Pode por acaso um homem consciente em
alguma medida respeitar a si mesmo?


\section{parte V}

Mas pode, pode por acaso o homem respeitar pelo menos um pouco a si
mesmo, um homem que almeja encontrar prazer até mesmo no próprio sentimento 
de humilhação? Não digo isso agora por algum tipo de
arrependimento meloso. Eu de qualquer maneira não aguentaria dizer:
``Perdão, papaizinho, não vou mais fazer isso''; não pelo fato de que eu
não seria capaz de dizê"-lo, mas talvez pelo contrário, justamente pelo
fato de ser capaz até demais, e como! Às vezes me metia, como que de
propósito, em situações nas quais eu não era nem um pouco culpado. E
essa era a coisa mais vil. Além disso, minha alma novamente se comovia,
eu me arrependia, derramava lágrimas e, é claro, enganava a mim mesmo,
embora em nenhum momento estivesse fingindo. Meu coração parecia fazer
algo torpe\ldots{} Não poderia culpar nem mesmo as leis da natureza, embora,
ainda assim, as leis da natureza tivessem, por toda a minha vida,
constantemente me ofendido mais que qualquer coisa. Dá nojo lembrar
tudo isso, assim como na época dava nojo. Mas então, depois de um
minuto, eu já percebia maldosamente que tudo isso não passava de
mentira, mentira, uma mentira detestável, afetada; ou seja, todo o
arrependimento, toda a comoção, todas as promessas de regeneração. Se
perguntarem: para que desfigurar e torturar a si mesmo dessa maneira? A
resposta: por ser extremamente enfadonho ficar de braços cruzados; e
com isso recaía em esquisitices. É verdade, era assim. Observem melhor
por conta própria, senhores, e então entenderão que é assim. Imaginava
para mim mesmo aventuras, criava uma vida para, ao menos de alguma
forma, viver. Quantas vezes me aconteceu de, por exemplo, ofender"-me
por motivo algum, de propósito; e às vezes sabia que me ofendera por
nada, fazia de conta, mas acabava chegando ao ponto de, no final,
realmente e de fato me ofender. Por toda a minha vida eu me sentira
tentado a fazer dessas, de maneira que acabei perdendo o controle sobre
mim mesmo. Uma vez quis me apaixonar à força; até duas vezes. E sofri,
senhores, garanto"-lhes. No fundo de minha alma, não acreditava que
estava sofrendo, parecia surgir um gracejo, mas mesmo assim sofria, e
ainda por cima de uma maneira verdadeira, autêntica; sentia ciúmes,
saía de mim\ldots{} E tudo por conta do tédio, meus senhores, tudo por conta
do tédio; a inércia me esmagava. Pois o fruto direto, legítimo e
imediato da consciência é a inércia, ou seja, o ficar conscientemente
de braços cruzados. Já me referi a isso acima. Repito, reiteradamente
repito: todos os homens imediatos e de ação, por serem de ação, são
obtusos e limitados. Como isso se explica? Direi como: devido a sua
limitação, tomam as causas mais imediatas e secundárias como primeiras,
e desta forma, têm maior probabilidade e facilidade, em relação aos
outros, de convencer"-se de que encontraram uma base indefectível para
suas ações, e com isso se acalmam; e isso é o mais importante. Pois
para começar a agir, é preciso estar, de antemão, completamente calmo,
e que nenhuma dúvida tenha restado. Mas e eu, por exemplo, com que vou
me acalmar? Onde estão minhas causas primeiras, em que eu possa me
apoiar, onde estão as bases? De onde poderei tirá"-las? Eu me exercito
em pensamentos mas, por conseguinte, sinto que cada causa primeira
imediatamente arrasta consigo uma outra, ainda anterior, e assim por
diante até o infinito. É justamente essa a essência de qualquer
consciência e de qualquer pensamento. São novamente, portanto, as leis
da natureza. E qual é o resultado final? Pois é o mesmo. Lembrem"-se: há
pouco falei de vingança. (Vocês certamente não foram a fundo.) Foi
dito: o homem se vinga, porque vê nisso a justiça. Ou seja, ele
encontrou sua causa primeira, sua base; a saber: a justiça. Portanto,
ele está sob todos os aspectos tranquilo, e, por conseguinte, consegue
se vingar com tranquilidade e êxito, estando certo de que faz algo
justo e honrado. Já eu não vejo justiça nisso, tampouco enxergo virtude
alguma, e, por conseguinte, se começasse a me vingar, seria talvez
apenas de raiva. A raiva, é claro, poderia dominar tudo, todas as
minhas dúvidas, e, portanto, poderia perfeitamente bem servir como
causa primeira justamente porque não é uma causa. Mas o que se pode
fazer se nem raiva tenho? (Há pouco comecei justamente com isso.) A
minha maldade, novamente como consequência dessas malditas leis da
consciência, submete"-se a uma desintegração química. Você olha e o
objeto evapora, as razões evaporam, não se encontra um culpado, uma
ofensa torna"-se não uma ofensa, mas o destino, algo como uma dor de
dente, da qual ninguém é culpado, e novamente resta, por conseguinte,
apenas a mesma saída: socar com força o muro. Então você dá de ombros,
porque não encontrou a causa primeira. Mas tente deixar"-se levar
cegamente por seus sentimentos, sem raciocínio, sem causa primeira,
afastando a consciência pelo menos por um tempo; ame ou odeie, apenas
para não ficar de braços cruzados. Dois dias depois, quando muito, você
já desprezará a si mesmo por ter se enganado a si mesmo
propositalmente. O resultado: uma bolha de sabão e a inércia. Oh,
senhores, eu talvez me considere um homem inteligente apenas pelo fato
de que em toda a minha vida não consegui começar nem terminar nada.
Talvez eu seja um tagarela, um tagarela inofensivo e enfadonho, como
todos nós. Mas o que se pode fazer se a única e mais direta função de
qualquer homem inteligente é a tagarelice, ou seja, falar nada com
nada, conscientemente.


\section{parte VI}

Ah, se eu não fizesse nada apenas por preguiça. Deus, como eu então me
respeitaria. Respeitaria justamente porque pelo menos a preguiça teria condição
de ter em mim; pelo menos alguma característica seria de certa forma positiva
em mim, uma de que tivesse certeza. A pergunta: quem é?  A resposta: um
preguiçoso. Seria agradabilíssimo ouvir isso a respeito de mim mesmo. Ou seja,
seria definido positivamente, significaria que há algo a dizer sobre mim.
\textit{Preguiçoso!}: é um título e uma função, é até uma carreira. Não riam, é assim.
Eu seria então membro do melhor clube por direito e me preocuparia apenas em me
respeitar sem cessar.  Conheci um senhor que passou a vida inteira se gabando
de ser um perito em Lafite. Ele considerava aquilo uma de suas qualidades
positivas e nunca duvidava de si mesmo. Morreu com a consciência não somente
tranquila, mas triunfante, e tinha total razão. Eu então deveria escolher uma
carreira: seria um preguiçoso e um glutão, mas não apenas isso, e sim um que
fosse, por exemplo, simpatizante de tudo que há de belo e sublime. O que vocês
acham disso? Isso há tempos me ocorre. Esse \textit{belo e sublime} pesou"-me com força
na nuca aos quarenta anos; mas isso aos quarenta anos, já na época: oh, na
época teria sido diferente! Eu imediatamente encontraria para mim uma atividade
compatível; a saber: beber à saúde de tudo que é belo e sublime. Eu arranjaria
qualquer motivo para primeiro derramar uma lágrima em meu copo e depois bebê"-lo
em homenagem a tudo que há de belo e sublime. Eu então transformaria tudo que
há no mundo em belo e sublime; na porcaria mais indubitavelmente repelente eu
encontraria o belo e o sublime. Ficaria lacrimoso como uma esponja molhada. Um
artista, por exemplo, pintou um quadro digno de Gue. Beberia imediatamente à
saúde do artista que acaba de pintar um quadro digno de Gue, por amar tudo que
há de belo e sublime. Um autor escreveu \textit{Como cada um desejar};
beberia imediatamente à saúde ``cada um'', por amar tudo que há de \textit{belo e
sublime}.\footnote{ Em 1863, Dostoiévski envolveu"-se numa polêmica com o
escritor Mikhail Saltykov"-Shchedrin (1826--1889). O motivo da contenda foi o
quadro \textit{A última ceia}, de Nikolai Nikoláievitch Gue (1831--1894):
Dostoiévski acusou o artista de retratar de maneira anacrônica e mentirosa o
episódio bíblico. Já o artigo a que a passagem se refere foi escrito no mesmo
ano por Saltykov"-Schedrin, em defesa de Gue.} Exigiria que me respeitassem por
isso, perseguiria aqueles que me negassem o respeito.  Teria uma vida
tranquila, morreria solenemente: isso seria uma maravilha, uma verdadeira
maravilha! E deixaria então crescer tamanha pança, arranjaria um tamanho queixo
duplo, conseguiria um nariz tão vermelho de bebedeira, que qualquer um que
passasse diria, ao olhar para mim: ``Isso sim é um mérito! Isso sim é realmente
positivo!'' E digam o que quiserem, é agradabilíssimo ouvir tais apreciações em
nosso século negativo, senhores.


\section{parte VII}

Mas tudo isso não passa de sonhos dourados. Oh, digam"-me, quem foi o
primeiro a afirmar, quem foi o primeiro a proclamar que o homem apenas
comete obscenidades porque não conhece seus verdadeiros interesses; e
que se ele fosse instruído, se lhe abrissem os olhos para seus normais
e verdadeiros interesses, o homem imediatamente pararia de cometer
obscenidades, tornar"-se"-ia imediatamente bondoso e nobre, já que, sendo
instruído e compreendendo suas reais vantagens, veria justamente no bem
a sua própria vantagem, e, como é sabido que nenhum homem pode
deliberadamente agir contra suas próprias vantagens, ele
consequentemente, por necessidade, por assim dizer, passaria a fazer o
bem? Ah, criança! Oh, menino puro e inocente! Mas em primeiro lugar,
quando foi que outrora, ao longo dos últimos milênios o homem agiu
apenas em nome de suas próprias vantagens? O que fazer com os milhões
de fatos que atestam para o fato de que as pessoas
\textit{deliberadamente}, ou seja, tendo plena consciência de suas
reais vantagens, deixaram"-nas em segundo plano e lançaram"-se em direção
a outro caminho, em direção ao risco, ao acaso, sem que nada nem
ninguém as tenha forçado a isso, mas como que justamente sem desejar o
caminho indicado, e de maneira obstinada, insubordinada, abriram um
outro caminho, difícil, absurdo, buscando"-o quase nas trevas. Isso
significa, então, que lhes foi realmente mais agradável essa
obstinação, essa insubordinação do que qualquer vantagem\ldots{} Vantagem!
Mas o que é a vantagem? Aceitarão vocês a tarefa de definir com
perfeita precisão em que exatamente consiste a vantagem humana? E se
acontecer da vantagem humana \textit{às vezes} não
apenas poder, mas até mesmo dever consistir justamente em desejar, em
alguns casos, o mal, e não o vantajoso? E se for assim, se apenas pode
haver tal caso, então toda a regra cai por terra. O que vocês acham,
existem tais casos? Vocês estão rindo; riam, meus senhores, mas apenas
respondam: foram calculadas, com absoluta certeza, as vantagens
humanas? Não existem algumas que não apenas não se encaixam, mas que
tampouco podem se encaixar em qualquer classificação? Porque vocês,
senhores, pelo que me é conhecido, tiraram toda a sua lista de
vantagens humanas de uma média de dados estatísticos e de fórmulas das
ciências econômicas. As suas vantagens são o bem"-estar, a riqueza, a
liberdade, a tranquilidade, e assim por diante; de maneira que um homem
que, por exemplo, quisesse clara e deliberadamente ir contra toda essa
lista seria, de acordo com sua visão e, é claro, também de acordo com a
minha, um obscurantista ou um louco completo, não é? Mas o que é
impressionante é o seguinte: por que motivo acontece de todos esses
peritos em estatística, sábios e amantes do gênero humano, ao calcular
as vantagens humanas, constantemente deixam passar uma vantagem? Mesmo
no cômputo não a consideram na forma em que se deve considerar, e disso
depende todo o cômputo. Não faria mal algum pegar essa vantagem e
inseri"-la na lista. Mas o nefasto é exatamente isso, que essa complexa
vantagem não entra em nenhuma classificação, não cabe em nenhuma lista.
Eu, por exemplo, tenho um amigo\ldots{} Ah, meus senhores! Pois ele também é
amigo de vocês; mas de quem, de quem ele não é amigo?! Preparando"-se
para a ação, esse senhor imediatamente expõe, de maneira clara e
grandiloquente, como exatamente devo agir de acordo com as leis da
razão e da verdade. E não somente isso: falará, com agitação e paixão,
dos verdadeiros e normais interesses humanos; zombando, repreenderá os
tolos míopes que não entendem nem suas vantagens, nem o verdadeiro
sentido da virtude; e então, exatos quinze minutos depois, sem qualquer
motivo súbito ou estranho, mas justamente por alguma coisa interna,
mais forte que seus interesses, vai aprontar uma completamente
diferente, ou seja, irá visivelmente contra aquilo de que falava: tanto
contra as leis da razão, quanto contra a própria vantagem; resumindo,
contra tudo\ldots{} Advirto que meu amigo é uma personalidade coletiva, e
por isso é tão difícil culpar apenas ele. Aí é que está, senhores, não
existirá de fato algo que seja mais caro a quase todos os homens que
suas vantagens, ou (para não quebrar a lógica) existirá algo como a
mais vantajosa das vantagens (justamente aquela que se deixou passar,
como agora mesmo falávamos), que é mais importante e mais vantajosa que
todas as outras vantagens, e em nome da qual o homem, se for
necessário, estará pronto a ir contra todas as leis, ou seja, contra a
razão, a honra, a tranquilidade, o bem"-estar? Resumindo, contra todas
essas coisas belas e úteis, apenas para obter essa vantagem original e
mais vantajosa, que lhe é mais cara que tudo?

--- Bem, mas ainda assim são vantagens --- haverão vocês de me interromper.
--- Permitam"-me, nós ainda haveremos de nos explicar, pois não se trata de um
trocadilho, mas do fato de que essa vantagem é notável justamente por quebrar
todas as nossas classificações e esmagar constantemente os sistemas criados
pelos amantes do gênero humano, para felicidade do gênero humano. Resumindo,
atrapalha tudo. Mas antes de apresentar a vocês essa vantagem, queria me
comprometer pessoalmente e por isso declaro com audácia que todos esses belos
sistemas, todas essas teorias de elucidação, para a humanidade, de seus
interesses reais e normais, de maneira que ela, aspirando a alcançar
imprescindivelmente esses interesses, se tornasse imediatamente boa e nobre,
por ora não passam, em minha opinião, de logicismo! Sim, senhor, logicismo! Pois
sustentar ainda que somente esta teoria da renovação de todo o gênero humano
por meio de um sistema de suas próprias vantagens é, em minha visão, quase o
mesmo que\ldots{} quase como sustentar, por exemplo, como Buckle,\footnote{
Henry Thomas Buckle (1821--1862), historiador britânico, autor do incompleto
\textit{História da civilização na Inglaterra}.} que pela civilização o homem
se abranda e, por conseguinte, torna"-se menos sanguinário e menos capacitado
para a guerra. É aparentemente pela lógica que ele chega a tal conclusão. Mas a
tal ponto o homem tende para os sistemas e para as conclusões abstratas que
está disposto a deformar intencionalmente a verdade, está disposto a ignorar o
que vê e o que ouve, apenas para justificar a sua lógica. E por isso tomo esse
exemplo, por ser um exemplo demasiado claro. Pois olhem ao seu redor: o sangue
corre em abundância, e ainda da maneira mais alegre, como se fosse champanhe.
Vejam todo o nosso século  dezenove, no qual o próprio Buckle viveu. Vejam Napoleão:
tanto o grande, como o atual.\footnote{ Trata"-se de Napoleão Bonaparte
(1769--1821) e de seu sobrinho Luís Napoleão (1808--1873). Ambos reinaram na
França e conduziram guerras no continente europeu.} Vejam a América do Norte: a
eterna união.\footnote{ Referência à Guerra de Secessão, ocorrida nos Estados
Unidos entre 1861 e 1865.} Vejam, finalmente, o caricato
Schleswig"-Holstein\ldots{}\footnote{ A atual região alemã de Schleswig"-Holstein
foi motivo de disputa, ao longo de quase todo o século  dezenove, entre a Dinamarca,
de um lado, e a Áustria e a Prússia de outro. A primeira conflagração, que
durou três anos (entre 1848 e 1851), terminou com a vitória dos dinamarqueses;
já o segundo conflito, travado em 1864, foi vencido pela Confederação
Germânica, que anexou os territórios em litígio.} E como é que abranda em nós a
civilização? A civilização produz no homem apenas uma versatilidade de
sensações e\ldots{} definitivamente nada mais. E por meio do desenvolvimento
dessa versatilidade, o homem talvez ainda chegue ao ponto de encontrar prazer
no sangue. Isso até já aconteceu com ele. Vocês já repararam que os mais
refinados fascínoras eram quase todos senhores dos mais civilizados, a quem
todos esses diversos Átilas e Stiénkas Rázins\footnote{ Stepan Timofiêievitch
Rázin (c.~1630--1671), líder cossaco, liderou uma das maiores rebeliões
camponesas contra a monarquia em toda a história russa (1670--1).} por vezes
sequer se comparam, e que se eles não saltam tanto à vista como Átila ou
Stiénka Rázin, é justamente pelo fato de que são encontrados com demasiada
frequência, são por demais comuns, tornaram"-se por demais familiares.  No
mínimo, a civilização, se não fez o homem mais sanguinário, certamente o fez
sanguinário de uma maneira pior, mais abjeta do que antes. Antes ele via no
derramamento de sangue um ato de justiça e com a consciência tranquila
exterminava quem fosse necessário; hoje em dia, porém, embora consideremos o
derramamento de sangue uma obscenidade, mesmo assim cometemos essa obscenidade,
e ainda mais do que antes. O que é pior? Decidam por si mesmos. Dizem que
Cleópatra (perdoem"-me o exemplo tirado da história romana) adorava cravar
alfinetes de ouro no peito de suas escravas e sentia prazer com seus gritos e
convulsões.  Vocês dirão que isso aconteceu em tempos relativamente bárbaros;
que os tempos agora também são bárbaros, porque (também relativamente) agora
também cravam alfinetes; que agora também o homem, embora tenha aprendido às
vezes a ver as coisas mais claramente do que nos tempos bárbaros, está
longe de se \textit{acostumar} a agir como a razão e as ciências determinam.
Mas ainda assim vocês têm certeza absoluta de que ele sem dúvida haverá de se
acostumar, quando passarem de todo certos costumes antiquados e ruins e quando
o bom senso e a ciência reeducarem completamente e orientarem corretamente a
natureza humana.  Vocês têm certeza de que então o próprio homem parará
\textit{voluntariamente} de errar e, por assim dizer, inevitavelmente deixará
de querer unir sua vontade com seus interesses normais. E mais: então, dirão
vocês, a própria ciência ensinará ao homem (embora isso já seja um luxo, creio
eu) que na verdade ele não tem nem a vontade, nem o capricho, que nunca os
teve, e que ele não passa de algo como uma tecla de piano ou um pedal de órgão;
e que, além disso, há no mundo ainda as leis da natureza; de maneira que tudo
que ele venha a fazer, não fará em absoluto de acordo com sua vontade, mas por
si só, de acordo com as leis da natureza. Por conseguinte, basta apenas
descobrir essas leis da natureza e o homem já não responderá por seus atos e
viver será para ele extremamente fácil. Todos os atos humanos, presume"-se,
serão então calculados de acordo com essas leis, matematicamente, como numa
tabela de logaritmos, até 108 mil, e introduzidos num quadro; ou melhor que
isso, surgirão algumas edições oficiais e bem"-intencionadas, do tipo dos atuais
dicionários enciclopédicos, em que tudo será computado e determinado com
tamanha precisão que no mundo já não haverá mais atos, nem aventuras.

Então --- todos vocês dirão --- serão estabelecidas novas relações econômicas,
já de todo prontas e também calculadas com precisão matemática, de maneira que
num instante sumirão todas as questões possíveis, propriamente porque serão
dadas a elas todas as respostas possíveis. Então será construído um palácio de
cristal.\footnote{Alude"-se aqui ao ``Quarto sonho de Vera Pávlovna'', um dos
capítulos do romance \textit{O que fazer?}, escrito pelo filósofo,
jornalista e crítico literário Nikolai Gavrílovitch Tchernychevski (1828--1889),
entre 1862 e 1863. O palácio de cristal seria uma analogia de uma futura
sociedade socialista. Alguns consideram as \textit{Memórias do subsolo} uma
resposta justamente a esse romance.} Então\ldots{} Enfim, resumindo, então
chegará voando o pássaro Kagan.\footnote{ De acordo com a crença popular, o
pássaro Kagan trazia a felicidade.} É claro que de modo algum se pode garantir
(isso já sou eu que digo agora) que então tudo não será, por exemplo,
terrivelmente enfadonho (porque afinal o que é que se pode fazer quando tudo
for calculado de acordo com uma tabelinha?), embora tudo seja extremamente
sensato. É claro que se inventa de tudo por conta do enfado! Até alfinetes de
ouro se cravam por conta do enfado, mas isso não seria nada. O ruim (isso
novamente sou eu que digo) é que talvez, quem sabe, as pessoas, então, até com
os alfinetes de ouro passem a se alegrar.  Porque o homem é estúpido,
fabulosamente estúpido. Ou talvez não seja estúpido em absoluto, mas em
compensação é tão ingrato que não se encontra outro como ele, por mais que se
procure. E por isso eu não me espantaria nem um pouco se, por exemplo, de
repente, sem mais nem menos, em meio à sensatez generalizada do futuro, surgir
algum cavalheiro com uma fisionomia pouco nobre ou, melhor dizendo, retrógrada
e zombeteira, puser as mãos na cintura e disser a todos nós: e por que,
senhores, não chutamos toda essa sensatez de uma vez por todas para longe, com
o único propósito de mandar ao diabo todos esses logaritmos e de vivermos
novamente de acordo com nossa estúpida vontade! Isso ainda não seria nada, pois
o lamentável é que rapidamente encontrará seguidores: assim o homem é feito. E
tudo isso por conta do mais vão dos motivos, que aparentemente sequer merece
ser mencionado: justamente o fato de que o homem, sempre e em qualquer lugar,
seja ele quem for, amou agir como bem quis, e não em absoluto como lhe
ordenavam a razão e a vantagem; também se pode querer, afinal, contra a própria
vantagem, sendo que às vezes \textit{positivamente se deve} (essa já é uma
ideia minha). Sua própria vontade, livre e independente, seu próprio capricho,
ainda que o mais selvagem, sua fantasia, às vezes exasperada até o limite da
loucura: é de tudo isso que se trata a vantagem que deixaram passar, a mais
vantajosa das vantagens, que não se encaixa em qualquer classificação e devido
à qual todos os sistemas e teorias constantemente se quebram em pedaços. E de
onde é que todos esses sábios tiraram que o homem precisa de uma vontade normal
e virtuosa? A partir de que imaginam com tanta segurança que o homem precisa
indubitavelmente de uma vontade razoável e vantajosa? O homem precisa apenas de
uma vontade \textit{independente}, custe essa independência o que custar e leve
aonde levar. Mas também só o diabo sabe do que a vontade\ldots{}


\section{parte VIII}

--- Ha"-ha"-ha! Mas a vontade, na realidade, se você quer saber, nem existe!
--- interromperão vocês às gargalhadas. --- A ciência a essa altura a tal
ponto já conseguiu dissecar o homem, que agora já sabemos que a vontade
e o tão falado livre{}"-arbítrio não é mais que\ldots{}

--- Esperem, senhores, eu mesmo queria ter começado assim. Confesso que
até fiquei assustado. Agora há pouco eu queria gritar que só Deus sabe
do que a vontade depende e que talvez isso seja mesmo o melhor; mas aí
me lembrei da ciência e\ldots{} me contive. Mas vocês começaram a falar. E
na verdade, se de fato um dia for encontrada a fórmula de todas as
nossas vontades e caprichos, ou seja, do que dependem, de que leis
exatamente se originam, como exatamente se difundem, aonde aspiram em
tal e tal caso etc. etc., ou seja, a verdadeira fórmula matemática;
nesse caso, talvez, o homem imediatamente pararia de querer; talvez até
certamente parasse. Qual é a graça de querer de acordo com uma tabela?
Além disso: imediatamente o homem se transformaria num pedal de órgão
ou em algo do gênero; porque o que é um homem sem desejos, sem
liberdade e sem vontades senão um pedal em um órgão? O que vocês
acham? Calculemos as possibilidades: isso pode ou não acontecer?

--- Hum\ldots{} --- decidirão vocês. --- Nossas vontades na maior parte das vezes
são equivocadas por conta de uma visão equivocada de nossas vantagens.
Nós às vezes queremos uma autêntica tolice por vermos nessa tolice, em
nossa estupidez, o caminho mais fácil para a obtenção de uma suposta e
premeditada vantagem. Mas quando tudo isso for bem explicado e
computado num papel (o que é muito possível, porque seria ignóbil e
absurdo acreditar de antemão que o homem jamais conhecerá algumas leis
da natureza), então, é claro, não haverá os tão falados desejos. Porque
se a vontade em algum momento se conciliar com a razão, iremos então
raciocinar, e não propriamente querer, já que não se pode, por exemplo,
mantendo a razão, \textit{querer} coisas absurdas e dessa maneira
deliberadamente ir contra a razão e desejar o mal a si mesmo\ldots{} E uma
vez que todas as vontades e todos os raciocínios podem ser realmente
calculados, porque em algum momento acabarão descobrindo as leis de
nosso tão falado livre"-arbítrio, será então possível, portanto ---
brincadeiras à parte --- organizar"-se algo como uma tabelinha, de maneira
que nós realmente passemos a querer de acordo com essa tabelinha.
Porque se eu, por exemplo, em algum momento for calculado e me provarem
que, se eu mostrei uma figa para esse ou aquele, é justamente porque
não podia deixar de mostrar e que necessariamente deveria mostrá"-la com
aquele dedo, então o que é que restará de \textit{livre} em mim,
especialmente se eu sou um cientista e concluí em algum lugar o curso
de ciências? Eu poderia então calcular toda a minha vida pelos próximos
trinta anos; resumindo, se isso se arranjar, já não teremos o que
fazer; de qualquer maneira será preciso aceitar. E ainda temos que
repetir para nós mesmos, sem cessar, que necessariamente nesse dado
momento e nessas dadas circunstâncias a natureza não nos presta contas;
que é preciso aceitá"-la tal como ela é, e não como nós a fantasiamos, e
se nós realmente almejamos a tabelinha e o quadro, e também\ldots{} bem, e o
tubo de ensaio também, então o que se pode fazer, até o tubo de ensaio
temos que aceitar! Do contrário ela irá aceitar a si mesma, sem nós\ldots{}

--- Sim, senhor, mas é bem aqui que eu vejo um senão! Senhores, perdoem"-me por
ter começado a filosofar; são quarenta anos de subsolo! Permitam"-me fantasiar
um pouco. Vejam: a razão, senhores, é uma coisa boa, isso é indiscutível, mas a
razão é apenas a razão e satisfaz apenas as capacidades racionais do homem,
enquanto a vontade é a manifestação de toda a vida, ou seja, de toda a vida
humana, incluindo"-se a razão e todos os seus pruridos. E embora a nossa vida
nessa manifestação se reduza repetidas vezes a uma porcariazinha, ainda assim é
a vida, não apenas a extração da raiz quadrada. Eu, por exemplo, quero viver de
maneira completamente natural, apenas para satisfazer toda a minha capacidade
de viver, e não apenas para satisfazer a minha capacidade racional, ou seja,
talvez um vigésimo de toda a minha capacidade de viver. O que conhece a razão?
A razão conhece apenas o que teve tempo de conhecer (outra coisa talvez jamais
saiba; se por um lado não é um conforto, por outro por que não manifestá"-lo?),
enquanto a natureza humana age por inteiro, com tudo que há nela, consciente e
inconscientemente, e mesmo mentindo, ela vive. Suspeito, meus senhores, que
vocês olhem para mim com pena; vocês repetem para mim que um homem instruído e
evoluído, resumindo, o homem como ele será no futuro, não pode deliberadamente
querer algo que seja desvantajoso para si mesmo, que isso é matemática.
Plenamente de acordo, de fato é matemática. Mas repito a vocês pela centésima
vez, há apenas um caso, apenas um, em que o homem pode propositalmente,
conscientemente desejar para si até o que há de nocivo, de estúpido, até o que
há de mais estúpido, a saber: \textit{ter o direito} de desejar para si o que
há de mais estúpido e não ter a obrigação de desejar para si apenas o que há de
inteligente. Pode ser o que há de mais estúpido, pode ser um capricho seu, mas
na verdade, senhores, pode muito bem ser mais vantajoso para nós do que tudo
que há na terra, especialmente em alguns casos. Mas, particularmente, pode ser
mais vantajoso do que todas as vantagens mesmo nos casos em que produz um dano
evidente e contradiz as mais sensatas conclusões de nossa razão a respeito das
vantagens, porque em todo caso protege o que nos é mais importante e mais caro,
ou seja, nossa personalidade e nossa individualidade. Alguns garantem que isso
é de fato mais caro que tudo ao homem; é claro que a vontade pode, se assim se
quiser, coincidir com a razão, especialmente se não se abusar dela, mas usá"-la
moderadamente; é útil e às vezes até mesmo louvável. Mas a vontade muito
frequentemente, até na maioria das vezes, diverge completa e obstinadamente da
razão e\ldots{} e\ldots{} e sabem que isso é até útil e até às vezes muito
louvável? Senhores, suponhamos que o homem não seja estúpido. (E realmente não
podemos de forma alguma dizer isso dele, mesmo que apenas pelo fato de que, se
ele for estúpido, quem será inteligente?) Mas se não for estúpido, é mesmo
assim monstruosamente ingrato! É fenomenalmente ingrato. Penso até que a melhor
definição para o homem é: ser que caminha sobre duas pernas e que é ingrato.
Mas isso ainda não é tudo; ainda não é esse seu principal defeito; seu maior
defeito é sua constante imoralidade, constante, começando do Dilúvio até o
período de Schleswig"-Holstein dos destinos humanos. Imoralidade, e por
conseguinte, também imprudência; pois há muito tempo é sabido que a imprudência
não tem outra origem que não a imoralidade. Tentem dar uma olhada na história
da humanidade; o que vocês verão? Grandiosidade? Talvez até grandiosidade; só o
Colosso de Rodes, por exemplo, quanto não vale! Não é à toa que o senhor
Anaievski\footnote{ Afanássi Ievdokímovitch Anaievski (1788--1866), poeta e
ensaísta. Seus trabalhos foram alvo de constante zombaria nos meios
jornalísticos entre as décadas de 1840 e 1860.} atesta, a respeito dele, que
alguns dizem ser produto de mãos humanas; já outros garantem que foi criado pela
própria natureza. Variedade? Talvez também variedade; estudar apenas os
uniformes de gala de todos os tempos e de todos os povos, tanto militares como
civis; só isso, quanto não vale. E se forem incluídos os uniformes comuns já se
pode quebrar completamente a cabeça; nenhum historiador resistiria. Monotonia?
Sim, talvez monotonia: lutam e lutam, tanto agora lutam, como antes lutaram e
depois continuarão lutando: há de se concordar que é até monótono demais.
Resumindo, pode"-se dizer tudo da história do mundo, tudo que a mais exaltada
imaginação poderia conceber. Só não podemos dizer uma coisa: que ela é
racional. Engasgaríamos na primeira palavra. E há uma coisa que se encontra a
todo instante: aparecem constantemente na vida essas pessoas de boa conduta e
sensatas, esses sábios e amantes do gênero humano, que têm justamente por
objetivo comportar"-se por toda a vida com a melhor conduta e da maneira mais
sensata possível, ser uma luz, por assim dizer, para o próximo, simplesmente
para provar a eles que realmente é possível viver no mundo com uma boa conduta
e de maneira sensata. E daí? É sabido que muitos desses amantes do gênero
humano, cedo ou tarde, antes do fim de suas vidas haverão traído a si mesmos,
protagonizando alguma anedota qualquer, às vezes até das mais indecentes. Agora
eu pergunto a vocês: o que se pode esperar do homem sendo ele uma criatura
agraciada com qualidades tão estranhas? Pois cubram"-no com todas as benesses do
mundo, afundem"-no com felicidade até a cabeça, de maneira que umas bolhinhas
brotassem na superfície da felicidade, como se fosse na água; dê a ele tamanha
abundância econômica, que já não lhe restará mais nada que fazer além de
dormir, comer docinhos e cuidar da continuidade da história mundial: e então
esse homem, mesmo com tudo isso, apenas por ingratidão, apenas para difamar,
fará a vocês alguma coisa indecente. Arriscará até seus docinhos, e
propositalmente desejará a mais perniciosa tolice, a coisa mais absurda e
antieconômica, unicamente para misturar em toda essa racionalidade positiva o
seu fantástico e pernicioso elemento.  Justamente os seus sonhos fantásticos e
a sua estupidez mais que vulgar é que ele desejará manter consigo, unicamente
para comprovar a si mesmo (como se isso fosse necessário), que as pessoas serão
sempre pessoas, e não teclas de piano, e que, embora nelas sejam as próprias
leis da natureza tocando com suas mãos, ainda assim existe a ameaça de que ela
toque até o ponto em que, fora dos quadros e esquemas, não será possível querer
mais nada. E isso não é tudo: mesmo que ele realmente acabe sendo uma tecla de
piano, mesmo que isso lhe seja demonstrado pelas leis naturais e pela
matemática, nem com isso ele haverá de tomar juízo, mas fará alguma coisa
para contrariar, unicamente por ingratidão, apenas para manter sua posição.
Nesse caso, se ele não tiver os meios para isso, inventará a destruição e o
caos, inventará sofrimentos diversos e manterá contudo a sua posição! Lançará
uma maldição contra o mundo, e, já que apenas o homem pode amaldiçoar (isso é
um privilégio seu, que é o principal fator de distinção entre ele e os outros
animais), talvez apenas a maldição fará com que atinja seu objetivo, ou seja,
realmente convencer"-se de que ele é um homem, e não uma tecla de piano! Se
vocês disserem que mesmo tudo isso pode ser calculado por uma tabelinha, tanto
o caos e as trevas como a maldição, e que apenas a possibilidade de um cálculo
preliminar conterá tudo e a razão atingirá seu objetivo; nesse caso, o homem
propositalmente ficará louco para não ter razão e manter a sua posição!
Acredito nisso, respondo por isso, porque toda a questão humana parece
realmente se resumir ao fato de que o homem a todo instante precisa provar a si
mesmo que ele é um homem, e não um pedal! Mesmo que às suas próprias custas,
mas provará; mesmo que agindo como um troglodita, mas provará.  E depois disso
como não pecar, como não louvar o fato de que as coisas ainda não são assim e
que a vontade por enquanto só Deus sabe do que depende\ldots{}

Vocês gritarão para mim (se é que ainda me darão a honra de seu grito)
que ninguém quer tirar minha vontade; que apenas cuidam para fazer, de
alguma maneira, minha própria vontade, por sua própria vontade,
coincidir com meus interesses normais, com as leis da natureza e com a
aritmética.

--- Ora, meus senhores, que vontade será essa se a questão se resumir a
uma tabela e à aritmética, quando estiver em curso apenas o dois e dois
são quatro? Dois e dois serão quatro mesmo sem a minha vontade. Essa é
a minha vontade?!


\section{parte IX}

Senhores, é claro que eu estou brincando, e eu mesmo sei que minha brincadeira
não é boa; mas nem tudo se pode tomar por brincadeira.  Talvez eu brinque
rangendo os dentes. Senhores, sou atormentado por dúvidas; permitam que eu as
tenha. Vocês, por exemplo, querem afastar um homem de seus velhos costumes e
corrigir sua vontade de acordo com as exigências da ciência e do bom senso. Mas
como é que vocês sabem que não apenas é possível, mas também
\textit{necessário} reeducar dessa maneira um homem? A partir de que vocês
concluem que é tão \textit{imprescindível} corrigir a vontade humana?
Resumindo, como é que vocês sabem que tal correção trará de fato vantagem ao
homem? E, se é para falar, como é que vocês têm tanta \textit{certeza} de que
não ir contra as vantagens verdadeiras e normais, garantidas pelos argumentos
da razão e da aritmética, é de fato sempre vantajoso para o homem e de que há
uma lei para toda a humanidade? Porque isso é, por enquanto, apenas uma
suposição sua. Suponhamos que seja uma lei da lógica, mas talvez em absoluto
não da humanidade. Vocês talvez pensem, meus senhores, que sou louco.
Permitam"-me fazer uma ressalva. Concordo: o homem é um animal preferencialmente
criativo, impelido a conscientemente almejar um objetivo e a estudar as
técnicas da engenharia, ou seja, eterna e constantemente abrir caminho para si
\textit{para onde quer que seja}. Mas talvez ele às vezes queira desviar"-se
para o lado exatamente pelo fato de que está \textit{condenado} a abrir esse
caminho, e talvez ainda porque, por estúpido que seja o homem imediato e de
ação, mesmo assim às vezes ocorrerá a ele o pensamento de que o caminho acabará
quase sempre indo \textit{para onde quer que seja} e que o principal não é o
lugar aonde se vai, mas o fato de que ele simplesmente vai, e que a criança
bem"-educada, ao fazer pouco caso das técnicas da engenharia, não se entregue a
uma ociosidade ruinosa, que, como é sabido, é a mãe de todos os vícios. O homem
adora criar e abrir caminhos, isso é indiscutível. Mas por que razão ele ama
com furor também a destruição e o caos? Digam"-me! Mas sobre isso quero eu mesmo
declarar duas palavras em particular. Não é possível que talvez ele ame tanto a
destruição e o caos (e é indiscutível que ele às vezes ama muito, assim é) pelo
fato de que ele mesmo, instintivamente, teme alcançar o objetivo e concluir o
edifício criado? Como se pode saber? Talvez ele ame o edifício apenas de longe,
mas de forma alguma de perto; talvez ele apenas adore criá"-lo, não viver nele,
cedendo depois \textit{aux animaux domestiques},\footnote{No original, em
francês, ``aos animais domésticos''.} como formigas, carneiros etc. etc. As
formigas têm um gosto completamente diferente. Elas têm um edifício formidável
desse mesmo tipo, indestrutível para todo o sempre: o formigueiro.

As veneráveis formigas começaram com o formigueiro, e certamente
terminarão com o formigueiro, o que confere grande honra a sua
perseverança e positividade. Mas o homem é um ser frívolo e indecoroso
e, de maneira talvez semelhante a um jogador de xadrez, ama apenas o
processo de obtenção de um objetivo, não o objetivo em si. E quem sabe
(não se pode garantir), talvez, todo o objetivo a que a humanidade
aspira na terra consista apenas e unicamente no processo ininterrupto
de obtenção, em outras palavras, na própria vida, e não propriamente no
objetivo, que, evidentemente, não deve ser outra coisa que não o dois e
dois são quatro, uma fórmula, portanto, mesmo porque o dois e dois são
quatro já não é vida, senhores, mas sim o começo da morte. O homem pelo
menos de certa forma sempre temeu esse dois e dois são quatro, e agora
eu também temo. Suponhamos que o homem não faça outra coisa que não
procurar esse dois e dois são quatro; ele atravessa oceanos, sacrifica
a vida nessa busca, mas encontrar, de fato achar, juro por Deus que de
alguma forma teme. Pois ele sente que, assim que achar, já não haverá
então nada que procurar. Os trabalhadores, ao terminarem sua tarefa,
pelo menos recebem seu dinheiro, vão para o botequim, e depois vão
parar na polícia: já é coisa com que se ocupar por uma semana. Mas e o
homem, aonde irá? Pelo menos toda vez nota"-se nele um certo incômodo ao
alcançar semelhantes objetivos. Ele ama o processo de alcançar, mas o
alcançar já não ama de todo, o que é, certamente, terrivelmente
ridículo. Resumindo, o homem é construído de maneira cômica; nisso tudo
encerra"-se nitidamente um trocadilho. Mas que dois e dois são quatro é
algo ainda assim insuportabilíssimo. O dois e dois são quatro é, em
minha opinião, nada mais que um descaramento. O dois e dois são quatro
parece um almofadinha que tranca a sua passagem com as mãos na cintura
e cuspindo. Admito que o dois e dois são quatro é uma coisa magnífica;
mas se é para tecer louvores, então o dois e dois são cinco é às vezes
uma coisinha das mais adoráveis.

E como é que vocês podem ter uma certeza tão forte e solene de que
apenas o normal, o positivo --- resumindo, apenas o bem"-estar --- são
vantajosos ao homem? Não se equivocará a razão no que se refere às
vantagens? Porque talvez o homem não ame apenas o bem"-estar. Talvez ele
ame em igual medida o sofrimento. Talvez o sofrimento seja para ele
exatamente tão vantajoso quanto o bem"-estar. E o homem por vezes ama
terrivelmente o sofrimento, apaixonadamente, e isso é fato. Quanto a
isso não há por que consultar a história mundial; perguntem a si
mesmos, desde que sejam humanos e tenham vivido pelo menos um pouco. No
que se refere a minha opinião pessoal, amar apenas o bem"-estar é até de
certa forma indecente. Bem ou mal, às vezes também é muito bom quebrar
alguma coisa. Não estou aqui propriamente defendendo o sofrimento, mas
também não defendo o bem"-estar. Defendo\ldots{} o meu capricho e que ele me
seja garantido quando for necessário. O sofrimento, por exemplo, não é
admitido em \textit{vaudevilles}, sei disso. No palácio de cristal ele
é impensável: o sofrimento é dúvida, é negação, e que palácio de
cristal seria esse se nele se pudesse ter dúvidas? E no entanto tenho
certeza de que o homem nunca recusará o verdadeiro sofrimento, ou seja,
a destruição e o caos. O sofrimento, afinal, é a única causa da
consciência. Embora tenha anunciado no início que a consciência, em
minha visão, é a maior infelicidade do homem, sei que o homem a ama e
não a trocaria por nenhuma satisfação. A consciência é, por exemplo,
infinitamente mais elevada que o dois e dois. Depois do dois e dois é
claro que já não resta nada, não apenas a fazer, mas até a conhecer.
Será possível apenas fechar seus cinco sentidos e mergulhar em
contemplação. E mesmo que com a consciência se chegue ao mesmo
resultado --- e portanto tampouco haverá o que fazer ---, pelo menos
pode"-se às vezes fustigar a si mesmo, o que é de qualquer maneira
revigorante. Pode ser retrógrado, mas ainda é melhor que nada.


\section{parte X}

Vocês creem no edifício de cristal, eternamente indestrutível; ou seja,
num para o qual não se possa nem mostrar a língua às escondidas, nem
fazer figa com a mão dentro do bolso. E eu talvez tenha medo desse
edifício justo por ser de cristal e eternamente indestrutível e pelo
fato de que não se pode mostrar"-lhe a língua nem às escondidas.

Pois vejam só: se em vez de um palácio for um galinheiro e uma chuva
cair, eu talvez me arrastasse para dentro do galinheiro para não me
molhar, mas ainda assim não tomarei o galinheiro por um palácio como
forma de gratidão, por ter me protegido da chuva. Vocês riem, vocês até
mesmo dizem que nesse caso um galinheiro ou uma mansão darão na mesma.
Sim --- respondo eu --- mas isso se devêssemos viver apenas para não nos
molhar.

Mas o que fazer se meti na cabeça que não se vive apenas para isso, e
que, se é para viver, que se viva numa mansão. É minha vontade, é meu
desejo. Vocês só o arrancarão de mim quando mudarem os meus desejos.
Pois mudem, fascinem"-me com alguma outra coisa, deem"-me um outro ideal.
Mas enquanto isso eu não tomarei um galinheiro por palácio. Talvez o
edifício de cristal seja uma invencionice, que de acordo com as leis da
natureza ele nem deveria existir e que eu o inventei apenas como
consequência de minha própria estupidez, como consequência de alguns
hábitos antiquados e irracionais. Mas que tenho eu se não deveria
existir? Não dá no mesmo se ele existe em meus desejos ou, melhor
dizendo, existe apenas enquanto existem os meus desejos? Talvez vocês
estejam novamente rindo. Riam à vontade; aceitarei todos os gracejos e
mesmo assim não direi que estou satisfeito quando ainda quero comer;
mesmo assim sei que eu não vou me tranquilizar com um compromisso, com
um ininterrupto e periódico zero, apenas porque ele existe de acordo
com as leis da natureza e existe \textit{de fato}.
Não tomarei como a coroação de meus desejos um prédio de aluguel, com
apartamentos para locatários pobres e contratos de mil anos e, por via
das dúvidas, com uma placa do dentista Wagenheim. Destruam meus
desejos, apaguem meus ideais, mostrem"-me algo melhor, e eu os seguirei.
Vocês talvez digam que não vale a pena se envolver; mas nesse caso eu
posso responder a vocês da mesma maneira. Estamos discutindo
seriamente; mas se não quiserem me dar a honra de sua atenção, não vou
me inclinar. Tenho meu subsolo.

Enquanto isso eu ainda vivo e desejo, e que seque a minha mão se eu
levar um tijolinho que seja para esse prédio de aluguel! Não liguem
para o fato de que agora há pouco eu mesmo rejeitei o edifício de
cristal, unicamente pelo motivo de que não se pode provocá"-lo mostrando
a língua. Não disse isso em absoluto por gostar de mostrar a língua.
Talvez eu tenha me irritado com o fato de que, dentre todos os seus
edifícios, não se encontra um ao qual se possa não mostrar a língua.
Pelo contrário, deixaria que me cortassem a língua fora, apenas por
gratidão, se pelo menos as coisas fossem feitas de tal forma que eu
mesmo já não tivesse nunca mais vontade de mostrá"-la. E que tenho eu
com o fato de que é impossível fazer as coisas dessa maneira e de que é
necessário contentar"-se com os apartamentos? Por que é que eu fui feito
com tais desejos? Será possível que eu tenha sido feito apenas para
chegar à conclusão de que toda a minha constituição não passa de
enganação? Será possível que nisso esteja todo o objetivo? Não creio.

Saibam, porém, o seguinte: tenho certeza de que nós do subsolo
precisamos ser contidos com rédea curta. Ele pode até ser capaz de
permanecer em silêncio por quarenta anos no subsolo, mas se sair à luz
ele estoura, e então fala, fala, fala\ldots{}


\section{parte XI}

No fim das contas, senhores: é melhor não fazer nada! É melhor ter uma
inércia consciente! Então, viva o subsolo! Eu posso até ter dito que
invejo um homem normal ao extremo, mas, nas condições em que o vejo,
não quero ser ele (embora ainda assim não pare de invejá"-lo. Não, não,
o subsolo de qualquer maneira é mais vantajoso!). Lá pelo menos é
possível\ldots{} Ah! Sobre isso também estou mentindo! Estou mentindo porque
eu mesmo sei, como dois e dois são quatro, que o subsolo não é em
absoluto melhor, mas alguma outra coisa, completamente diferente, pela
qual anseio mas que não acho de forma alguma! Ao diabo com o subsolo!

Quanto a isso, seria melhor até mesmo o seguinte: se eu mesmo
acreditasse em alguma coisa de tudo isso que eu escrevi. Juro a vocês,
senhores, que não acredito em nenhuma, em nenhumazinha das palavras que
eu agora teci! Quer dizer, talvez acredite, mas ao mesmo tempo, não se
sabe por quê, sinto e desconfio que eu esteja mentindo como um
pescador.

--- Então por que é que escreveu tudo isso? --- vocês me dirão.

--- Mas e se eu os colocasse por uns quarenta anos sem qualquer ocupação,
e então voltasse, depois de quarenta anos, no subsolo, para descobrir a
que ponto chegaram? Será possível deixar um homem, sem fazer nada, por
quarenta anos, sozinho?

--- E isso não é vergonhoso, e isso não é humilhante?! --- vocês talvez me
digam, meneando desdenhosamente a cabeça. --- Você anseia pela vida mas
resolve as questões vitais com uma confusão lógica. E como são
inoportunas, como são insolentes as suas saídas, e, ao mesmo tempo,
como você teme! Você fala uma tolice e se contenta com ela; você diz
coisas insolentes, mas teme constantemente e pede perdão por elas. Você
garante que não teme nada, e ao mesmo tempo busca obter nosso consenso.
Você garante que range os dentes, mas ao mesmo tempo graceja para nos
fazer rir. Sabe que seus gracejos não são espirituosos, mas é nítido
que está muito satisfeito com seus méritos literários. Talvez tenha de
fato acontecido de ter sofrido, mas não respeita nem um pouco seu
sofrimento. Pode até haver verdade no que diz, mas não há pureza; com a
mais mesquinha vaidade, você põe sua verdade à mostra, expondo"-a à
vergonha e aos olhos do público\ldots{} De fato quer dizer algo, mas por
medo esconde sua última palavra, porque em você não há coragem de
manifestá"-la, mas apenas um descaramento covarde. Você se gaba da
consciência, mas apenas hesita, porque embora sua mente funcione, seu
coração está obscurecido pela perversão, e sem um coração puro uma
consciência plena e correta não pode haver. E quanta impertinência em
você, como é insistente, como é afetado! Mentiras, mentiras e mentiras!

É claro que todas essas suas palavras eu mesmo acabei de compor. Elas
também são do subsolo. Lá, por quarenta anos seguidos, ouvi essas suas
palavras por uma frestinha. Eu mesmo as inventei, já que apenas isso
poderia inventar. Não é de se espantar que eu as tenha aprendido de cor e salteado 
e que elas tenham adquirido uma forma literária\ldots{}

Mas será possível, será possível que vocês sejam de fato a tal ponto
levianos que imaginem que eu imprimirei tudo isso e ainda darei para
vocês lerem? E esse é mais um problema para mim: para quê, de fato, eu
os chamo de \textit{senhores}, para quê me dirijo a vocês como se houvesse
realmente leitores? Confissões como as que pretendo relatar não são
impressas e dadas para os outros lerem. Pelo menos não tenho tanta
firmeza em mim, e nem considero necessário ter. Mas vejam bem:
ocorreu"-me uma fantasia, e quero realizá"-la custe o que custar. A
questão é a seguinte.

Há, nas lembranças de qualquer um, coisas que não são reveladas a todos,
mas apenas aos amigos. Há ainda aquelas que sequer aos amigos são
reveladas, mas apenas a si mesmo, e ainda assim em segredo. Mas há
finalmente aquelas que até a si mesmo se teme revelar, e tais coisas
qualquer homem decente tem acumuladas aos montes. Chega"-se até o
seguinte: quanto mais decente é o homem, mais ele as tem. Eu, pelo menos,
somente há pouco tempo decidi recordar algumas de minhas aventuras
passadas, já que desde então havia evitado todas elas, até com certa
inquietação. Agora que eu não apenas recordo, como até decidi
escrevê"-las, quero justamente fazer o seguinte experimento: é possível
ao menos consigo mesmo ser completamente sincero e não temer toda a
verdade? Aproveito para fazer uma observação: Heine afirma que as
autobiografias fiéis são quase impossíveis, e que uma pessoa certamente
mentirá a respeito de si mesma. De acordo com sua opinião, Rousseau,
por exemplo, certamente inventou mentiras a seu próprio respeito em
suas confissões, e até mesmo as inventou intencionalmente, por
vaidade. Tenho certeza de que Heine está certo; compreendo muitíssimo
bem que às vezes é possível, unicamente por vaidade, acusar a si mesmo
dos mais diversos crimes, e até posso conceber muitíssimo bem de que
tipo pode ser essa vaidade. Mas Heine julgava a respeito de homens que
fazem suas confissões diante do público. Já eu escrevo apenas para mim
mesmo, e declaro de uma vez por todas que, se eu escrevo como que me
dirigindo aos leitores, é unicamente por pose, porque assim me é mais
fácil escrever. Não passa de forma, uma forma vazia, já que eu nunca
terei leitores. Já declarei isso\ldots{}

Não quero que nada me limite na redação de minhas memórias. Não vou
desenvolver uma ordem e um sistema. Vou escrevendo conforme a minha lembrança.

Mas poderiam, por exemplo, implicar com o que eu disse e me perguntar:
se você de fato não conta com leitores, então por que é que agora faz
tais exortações a si mesmo, e ainda por cima no papel? Ou seja, por que
dizer que não vai desenvolver uma ordem e um sistema, que vai escrever
conforme for lembrando etc. etc.? Por que é que você se explica? Por que se
desculpa?

--- Pois vejam só --- respondo eu.

Aqui, porém, há toda uma psicologia. Talvez seja pelo fato de que sou um
covarde. Mas talvez seja pelo fato de que eu imagino propositalmente um
público diante de mim, para me portar de maneira mais decente enquanto
estiver escrevendo. Pode haver milhares de motivos.

Mas há ainda outra coisa: para quê, a troco de quê exatamente eu quero
escrever? Se não é para o público, então pode"-se relembrar tudo
mentalmente, sem passar para o papel.

Pois é; mas no papel tudo sairá mais solene. Há nisso algo inspirador, o
juízo sobre mim mesmo será melhor, aprimora"-se o estilo. Além disso:
talvez ao escrever realmente sinta um alívio. Agora, por exemplo,
sinto"-me oprimido por uma lembrança em especial do passado. Lembrei"-me
dela com clareza outro dia, e desde então permaneceu comigo, como um
enfadonho tema musical que não quer sair da cabeça. E no entanto é
necessário tirá"-la da cabeça. Tenho centenas dessas lembranças; mas de
quando em quando, de centenas delas, uma qualquer se sobressai e
oprime. Creio, por algum motivo, que se eu a escrever, ela acabará
saindo da minha cabeça. E por que não tentar?

Finalmente: estou enfadado, não costumo fazer nada. Escrever é, de fato,
como um trabalho. Dizem que pelo trabalho o homem torna"-se bom e
honrado. É uma chance, pelo menos.

Está nevando agora. Uma neve quase molhada, amarela, turva. Ontem também
choveu, e nos dias anteriores também. Creio que tenha sido por conta da
neve molhada que me lembrei da anedota que agora não me sai da cabeça.
Então que seja uma crônica a propósito de neve molhada.

\chapter[A propósito da neve molhada]{A propósito da neve\break molhada}

% \epigraph{%
% Quando da densa treva da ilusão,\\
% com a cálida voz da persuasão,\\
% ergui a tua alma que caíra\\
% e, cheia de profundos descompassos,\\
% ficaste a maldizer, torcendo os braços,\\
% o vício que há muito te cingira;

% Quando tua olvidada consciência,\\
% punida por memórias foi enfim,\\
% me deste a saber em confidência\\
% tudo o que houvera antes de mim.

% De repente, cobrindo o próprio rosto\\
% com as mãos, rebentaste num só pranto,\\
% repleta de vergonha e de desgosto,\\
% indignada em pleno desencanto\\
% etc. etc. etc\ldots{}}{\textit{De uma poesia de N.A.~Nekrássov}\footnotemark}
% \footnotetext{ Nikolai
% Aleksêievitch Nekrássov (1821--1878), poeta, prosador e crítico literário. Sua
% obra teve profunda influência nos círculos liberais e radicais da
% \textit{intelligentsia}. A tradução do poema é de Rafael Frate.}


\setlength{\epigraphwidth}{.55\textwidth}
\begin{epigraphs} 
\qitem{
Quando da densa treva da ilusão,\\
com a cálida voz da persuasão,\\
ergui a tua alma que caíra\\
e, cheia de profundos descompassos,\\
ficaste a maldizer, torcendo os braços,\\
o vício que há muito te cingira;

Quando tua olvidada consciência,\\
punida por memórias foi enfim,\\
me deste a saber em confidência\\
tudo o que houvera antes de mim.

De repente, cobrindo o próprio rosto\\
com as mãos, rebentaste num só pranto,\\
repleta de vergonha e de desgosto,\\
indignada em pleno desencanto\\
\textit{etc. etc. etc}\ldots{}
}{\textsc{n.\,a.\,nekrássov}\footnotemark}
\end{epigraphs}
\footnotetext{ Nikolai
Aleksêievitch Nekrássov (1821--1878), poeta, prosador e crítico literário. Sua
obra teve profunda influência nos círculos liberais e radicais da
\textit{intelligentsia}. A tradução do poema é de Rafael Frate.}
%\medskip

\section{parte I}

Na época eu não tinha mais que vinte e quatro anos. Já então minha vida
era sombria, desordenada e solitária ao ponto da selvageria. Não me
dava com ninguém e até evitava falar, ficando cada vez mais enfurnado
em meu canto. Durante o serviço, na repartição, eu até tentava não
olhar para ninguém, e percebia muito bem que meus colegas não só me
consideravam um excêntrico, como ainda --- e cada vez mais eu sentia isso
também --- pareciam me observar com uma certa repugnância. Veio"-me à
cabeça: por que é que não parece que estão olhando com repugnância para
ninguém além de mim? Um dos nossos colegas de repartição tinha um rosto
repulsivo e bexiguento, até mesmo com um ar de bandido. Creio que eu
não teria coragem de olhar para ninguém com um rosto indecente
daqueles. Um outro tinha o uniforme tão sujo que ao redor dele
sentia"-se um cheiro ruim. E no entanto nenhum desses senhores se
perturbava, nem a respeito de sua roupa, nem a respeito de seu rosto,
nem moralmente de forma alguma. Nem um nem outro imaginava que olhavam
para eles com repugnância; e se imaginassem, para eles tanto faria,
desde que a chefia não desse atenção àquilo. Parecia"-me agora
absolutamente claro que eu mesmo, devido a minha vaidade sem limites, e
portanto também às exigências que fazia em relação a mim mesmo,
encarava"-me muito frequentemente com uma insatisfação enfurecida, que
beirava a repugnância, e era por isso que atribuía mentalmente esse
olhar aos outros. Eu odiava, por exemplo, meu rosto, achava"-o
detestável, e até mesmo suspeitava que nele havia uma expressão de
alguma maneira infame, e por isso, a cada vez que eu aparecia no
serviço, esforçava"-me terrivelmente para me comportar da maneira mais
natural possível, para que não desconfiassem de minha infâmia, e para
que meu rosto expressasse a maior nobreza possível. ``Que seja um rosto
feio'', pensava eu, ``mas que pelo menos seja nobre, expressivo e, acima
de tudo, que seja \textit{extremamente} inteligente''. Mas eu tinha a
dolorosa certeza de que meu rosto nunca conseguiria expressar todas
aquelas qualidades. Mas o mais terrível de tudo era que eu o achava
decididamente estúpido. Eu me contentaria plenamente com o ar de
inteligência. Ao ponto de concordar com a expressão infame, desde que
apenas achassem meu rosto ao mesmo tempo terrivelmente inteligente.

Eu, obviamente, odiava todos os colegas de repartição, do primeiro ao
último, desprezava todos eles, mas ao mesmo tempo como que os temia.
Acontecia de eu às vezes colocá"-los acima de mim. Isso acontecia às
vezes comigo naquela época, e subitamente: ora desprezava, ora colocava
acima de mim. Um homem evoluído e digno não pode ser vaidoso sem um
grau ilimitado de exigência em relação a si próprio e sem desprezar a
si mesmo, em outros momentos, ao ponto do ódio. Mas fosse desprezando,
fosse colocando acima de mim, eu abaixava os olhos para qualquer um que
passasse por mim. Eu até fazia experiências: aguentaria eu olhar para
tal pessoa, por exemplo? Era sempre eu o primeiro a baixar os olhos.
Isso me atormentava e me enfurecia. Tinha também um medo doentio de
parecer ridículo, e por isso adorava servilmente a rotina em tudo que
se referia à aparência; entregava"-me apaixonadamente ao senso comum e
temia do fundo do coração qualquer excentricidade em mim. Mas como eu
poderia suportar? Eu, de uma maneira doentia, era evoluído, evoluído
como deve ser o homem de nosso tempo. Eles de qualquer forma eram todos
obtusos e parecidos uns com os outros como um rebanho de carneiros.
Talvez somente a mim em toda a repartição parecesse constantemente ser
eu covarde e servil; e exatamente por isso também parecia que eu
era evoluído. Mas não apenas parecia, na realidade era assim de fato:
eu era covarde e servil. Digo isso sem qualquer incômodo. Qualquer
homem digno em nossa época é e deve ser covarde e servil. É sua
condição normal. Estou profundamente convencido disso. Ele é feito
dessa maneira e com esse propósito. E não apenas no presente, por
quaisquer circunstâncias casuais, mas em absolutamente qualquer época
um homem digno deve ser covarde e servil. Esta é uma lei da natureza
para qualquer homem digno da terra. E se ocorre de algum deles tomar
coragem por algum motivo, isso não deve entusiasmá"-lo ou consolá"-lo:
ele de qualquer maneira vai se acovardar diante de outra coisa. Tal é a
única e eterna saída. Somente os asnos e outros animais do gênero
tentam bancar os valentes, e mesmo esses somente até encontrarem aquele
muro de sempre. Nem vale a pena dar atenção a eles, porque eles não
significam absolutamente nada.

Havia mais uma questão que na época me atormentava: exatamente o fato de
que ninguém se parecia comigo e de que eu não me parecia com ninguém.
``Eu estou sozinho, e eles são \textit{todos}'', pensava eu, melancólico.

Com isso notava"-se que eu não passava de um menino.

Acontecia também o oposto. Pois às vezes era muito desagradável ir para
a repartição: chegou ao ponto em que eu muitas vezes voltava do
trabalho doente. Mas de repente, sem mais nem menos, começava uma fase
de ceticismo e indiferença (tudo comigo eram fases) e eu mesmo
começava a rir de minha intolerância e de minha repulsa e acusava a mim
mesmo de ser um \textit{romântico}. Ora não queria conversar com
ninguém, ora chegava ao ponto de não apenas começar a conversar, como
ainda inventava de fazer amizades com eles. Toda a repulsa subitamente
desaparecia de uma vez, sem mais nem menos. Talvez, quem sabe, ela
nunca tenha existido em mim, fosse fajuta, tirada de livros. Eu até
agora ainda não resolvi essa questão. Em uma ocasião fiz de vez amizade
com eles, comecei a frequentar a casa deles, jogar \textit{préférence},
beber vodca, falar de promoções\ldots{} Mas permitam"-me fazer aqui uma
digressão.

Nós, russos, de um modo geral, nunca tivemos desses estúpidos e lunáticos
românticos alemães e especialmente franceses, a quem nada afeta, nem que a
terra trema sob seus pés, nem que a França inteira morra nas barricadas, são
sempre os mesmos, não mudam por nada e vão todos cantar suas canções lunáticas,
por assim dizer, até o fim de suas vidas, porque são uns tolos. Já aqui, na
nossa terra russa, não há tolos; isso é sabido; e é nisso que nos distinguimos
das terras estrangeiras. Por conseguinte, as naturezas lunáticas não ocorrem em
nosso país em seu estado puro. Foram os nossos publicistas e críticos
\textit{positivos} de então que, à caça na época de seus Kostanjoglos e de seus titios
Piôtr Ivánovitch,\footnote{ O proprietário de terras Konstanjoglo é um
personagem da segunda parte do romance \textit{Almas mortas}, de Nikolai Gógol
(1809--1852).  Piôtr Ivánovitch Adúiev é um personagem do romance \textit{Uma
história comum}, de Ivan Gontcharov (1812--1891). A imagem explorada por
Dostoiévski em ambas as referências é a do homem prático e empreendedor.} e por
tolice tomando"-os como o nosso ideal, pensavam com exagero a respeito de nossos
românticos, considerando"-os tão lunáticos quanto os da Alemanha e da França.
Pelo contrário, as características do nosso romântico são completa e
diretamente opostas às lunáticas europeias, e não há um padrãozinho europeu
sequer que aqui convenha. (Permitam"-me usar essa palavra: \textit{romântico}, uma
palavrinha antiga, respeitável, justa e conhecida de todos.) As características
do nosso romântico são o tudo compreender, ``o tudo ver e o ver com
frequência de maneira incomparavelmente mais clara que a de nossas mentes mais
absolutamente positivas''; não fazer concessões a nada nem a ninguém mas ao
mesmo tempo não sentir aversão por nada; esquivar"-se de tudo, ceder em tudo,
agir com todos diplomaticamente; não perder jamais de vista um objetivo útil e
prático (uma casa dada pelo governo, uma pensãozinha, uma estrelinha no peito),
perceber esse objetivo com todo o entusiasmo e por meio de seus livrinhos de
versinhos líricos e ao mesmo tempo manter intacto em si até o fim de seus dias
\textit{o belo e o sublime}, e além disso manter a si mesmo bem guardado e envolto num
algodãozinho, como se fosse algum tipo de bibelô, desde que seja, por exemplo,
em prol desse mesmo \textit{belo e sublime}. Nosso romântico é um homem generoso, além
de ser o primeiríssimo farsante dentre todos os nossos farsantes, eu lhes
asseguro\ldots{} até por experiência. É claro, tudo isso se o romântico for
inteligente. Mas o que é que estou dizendo! O romântico é sempre inteligente,
eu queria apenas ressaltar que, embora tenham existido em nosso país românticos
tolos, eles não contam, e existiram unicamente porque ainda na flor da idade
transformaram"-se definitivamente em alemães e, para guardar com maior conforto
seu bibelô, instalaram"-se em algum lugar lá fora, em Weimar ou na Floresta
Negra. Eu, por exemplo, desprezava abertamente minha atuação profissional e não
me queixava apenas por necessidade, porque afinal estava lá e recebia dinheiro
por isso. O resultado disso era que, percebam, eu apesar de tudo acabava não me
queixando. É mais provável que o nosso romântico enlouqueça (o que, aliás,
acontece raramente) antes de começar a se queixar; isso se não tiver em mente
outra carreira, e ainda jamais será expulso aos empurrões. Talvez o levem para
um hospício vestido de \textit{Rei da Espanha},\footnote{ Refere"-se ao louco
Poprischin, personagem da novela \textit{Diário de um louco}, de Nikolai
Gógol.} mas só no caso de ter ficado louco demais. Mas entre nós só enlouquecem
os fracotes e os loirinhos. Um número incontável de românticos posteriormente
galgam patentes significativas. Uma versatilidade extraordinária! E que
capacidade para os mais contraditórios sentimentos! Eu na época me consolava,
assim como hoje, com esses pensamentos. É por conta disso que entre nós há
tantas pessoas de \textit{caráter generoso}, que, mesmo estando prestes a sofrer a
queda final, não perdem jamais seus ideais; e embora não movam um dedo por seus
ideais, embora sejam perfeitos bandidos e ladrões, ainda assim defendem seus
ideais originais com lágrimas nos olhos e têm um coração extraordinariamente
honesto. Sim, senhor: apenas entre nós o mais perfeito canalha pode ter um
coração de todo honesto e ser até mesmo elevado, ao mesmo tempo em que continua
sendo, até certo ponto, um canalha. Repito, é frequentemente dos nossos
românticos que saem, a três por quatro, esses espertalhões de negócios (uso a
palavra \textit{espertalhão} com gosto), que subitamente demonstram um tal senso de
realidade e um conhecimento positivo que fazem os extasiados superiores e o
público estalarem os lábios de estupefação.

Uma versatilidade verdadeiramente impressionante, e sabe Deus no que ela
pode se transformar e se moldar nas circunstâncias vindouras e o que
ela pressagia para nosso futuro. Mas não é nada mau esse material! Não
é por conta de qualquer tipo de patriotismo barato e ridículo que eu
estou dizendo isso. Aliás, tenho certeza de que vocês novamente estão
pensando que estou de brincadeira. Mas quem sabe? Talvez seja o
contrário. Talvez tenham certeza de que eu de fato penso assim. De
qualquer maneira, senhores, ambas as opiniões, por parte de vocês,
considerarei uma honra para mim e uma especial satisfação. E queiram
perdoar minha digressão.

Com meus colegas, é claro, não conseguia manter a amizade, muito
rapidamente brigava com eles e, por conta da inexperiência pueril de
então, até parava de cumprimentá"-los, como se tivesse cortado relações.
Isso na verdade só aconteceu comigo uma vez. Em geral eu estava sempre
sozinho.

Em casa eu mais lia que qualquer outra coisa. Tinha vontade de reprimir,
com sensações exteriores, tudo aquilo que continuamente vinha se
acumulando em mim. E a única possibilidade que eu tinha de sensações
exteriores era a leitura. A leitura sem dúvida ajudava bastante: me
perturbava, deleitava, torturava. Mas de tempos em tempos me enfastiava
terrivelmente. Mas mesmo assim queria me movimentar, e subitamente eu
me afundava numa perversãozinha --- não numa perversão ---, numa
perversãozinha obscura, subterrânea e vil. As paixõezinhas em mim eram
agudas e pungentes, por conta de minha costumeira e doentia
irritabilidade. Tinha ímpetos histéricos, que me provocavam lágrimas e
convulsões. Além da leitura, não havia aonde ir; ou seja, não havia
nada que eu prezasse ao meu redor e que me atraísse. Além disso, fervia
em mim a melancolia; uma sede histérica de contradições, contrastes e
com isso eu me entregava à perversão. Não comecei em absoluto a falar
disso agora para me justificar\ldots{} Aliás, não! Minto! Eu queria
justamente me justificar. Faço aqui uma observaçãozinha, senhores, para
mim mesmo. Não quero mentir. Dei minha palavra.

Eu me entregava solitário à perversão, de madrugada, secretamente, com
medo, de maneira torpe e com uma vergonha que não me deixava sequer nos
momentos mais repugnantes e que nesses momentos chegava a beirar o
insulto. Já naquela época eu carregava na alma o meu subsolo. Tinha um
medo terrível de que acabassem me vendo, de que me encontrassem, me
reconhecessem. Andava por diversos lugares extremamente escuros.

Uma vez, passando de madrugada em frente a uma tavernazinha, vi, pela
janela iluminada, uns senhores brigando com tacos junto a uma mesa de
bilhar e um deles ser jogado pela janela. Em outras ocasiões, eu teria
ficado enojado; mas na época por um momento pareceu que eu ficara com
inveja daquele senhor que fora atirado, e invejava tanto que até entrei
na taverna, na sala de bilhar: ``quem sabe --- pensei --- eu também não
arrumo uma briga e sou jogado pela janela''.

Eu não estava bêbado, mas o que fazer: até que grau de histeria a
melancolia não pode levar! Mas não deu em nada. No fim das contas eu
não era capaz nem de saltar pela janela, e acabei saindo sem brigar.

Já no primeiro passo um dos oficiais me colocou no meu lugar.

Eu estava parado junto à mesa de bilhar e desavisadamente obstruí o
caminho pelo qual o outro tinha de passar; ele me segurou pelos ombros
e, sem dizer uma palavra, inadvertidamente e sem dar nenhuma
explicação, moveu"-me do lugar em que eu estava para outro. Depois
passou como se nem tivesse notado. Eu teria perdoado até mesmo uma
surra, mas de forma alguma poderia perdoar o fato de ele ter me movido
de forma tão resoluta sem sequer me notar.

Sabe Deus o que eu daria na época por uma briga verdadeira, mais justa,
mais decente, mais, digamos, \textit{literária}! Agiram comigo como se
eu fosse uma mosca. O tal oficial tinha dez \textit{verchoks} de
altura;\footnote{ O \textit{verchok} era uma medida russa equivalente a
4,44 cm. A estatura das pessoas era expressa em \textit{verchoks} acima
de dois \textit{archins} (71,12 cm). Com isso, a altura do oficial era
de aproximadamente 1,86 m.} já eu sou um homem baixinho e desnutrido. A
briga, na verdade, estava em minhas mãos: bastaria protestar e,
certamente, me jogariam pela janela. Mas eu pensei melhor e preferi\ldots{}
recuar amargamente.

Saí da taverna desconcertado e agitado e fui direto para casa; no outro
dia, continuei minha perversãozinha de maneira ainda mais tímida,
amedrontada e triste do que antes, como se fosse com lágrimas nos
olhos; mas mesmo assim continuei. Não pensem, porém, que recuei diante
do oficial por covardia: no fundo nunca fui um covarde, embora
constantemente me acovardasse na prática; mas não riam por enquanto, há
uma explicação para isso. Tenho explicação para tudo, tenham certeza
disso.

Ah, se esse oficial fosse dos que concordam em bater"-se num duelo! Mas
não, ele era exatamente desses senhores (infelizmente há muito
desaparecidos!) que preferiam agir com tacos de bilhar ou, como o
tenente Pirogov de Gógol,\footnote{ Personagem da novela \textit{Avenida
Niêvski}, de Nikolai Gógol.} apoiando"-se nas autoridades. Mas não
duelavam, e de qualquer maneira com gente da minha laia, um civil,
considerariam um duelo algo indecente; consideravam os duelos em geral
algo inconcebível, típico de livre"-pensador e francês. Eles mesmos,
porém, ofendiam os outros à vontade, especialmente quando tinham um e
oitenta de altura.

Não recuei por covardia, mas sim por conta de uma vaidade sem qualquer limite.
Não me assustei com os quase um e oitenta de altura e com o fato de que me
dariam uma bela surra e de que me jogariam pela janela.  Coragem física, é
fato, teria de sobra; o que faltava em mim era a coragem moral. Temia que
nenhum dos presentes --- do atrevido marcador de pontos até o último
funcionariozinho pútrido que ficava sempre de um lado para outro com sua gola
toda ensebada --- entendesse e que todos me ridicularizassem quando eu
protestasse e começasse a falar com eles numa linguagem livresca. Porque sobre
o ponto de honra --- ou seja, não sobre a honra, mas sobre o ponto de honra
(\textit{point d'honneur}) --- entre nós por enquanto é impossível falar de
outra maneira que não por meio de uma linguagem livresca. Numa linguagem normal
não se menciona o \textit{ponto de honra}. Eu estava plenamente convencido (um lapso
de realidade, apesar de todo o romantismo!) de que todos iriam simplesmente
morrer de rir e de que o oficial simplesmente --- ou seja, não de maneira
inofensiva --- me daria uma surra, mas também de que certamente me daria golpes
com o joelho ao redor da mesa de bilhar, e de que depois ficaria com pena e me
jogaria pela janela. É claro que essa história miserável para mim daquela forma
simplesmente não poderia acabar. Mais tarde, eu encontrava com frequência
aquele oficial na rua e prestava bastante atenção nele. Só não sei se ele me
reconhecia. É possível que não; chego a essa conclusão por conta de alguns
indícios. Já eu, eu olhava para ele com ódio e inveja, e isso durou\ldots{}
alguns anos! O meu ódio ficou até mesmo mais forte e cresceu com os anos. No
começo, aos poucos, eu comecei a juntar informações sobre esse oficial. Era
muito difícil para mim, porque eu não conhecia ninguém. Mas uma vez alguém o
chamou pelo nome na rua, quando eu o seguia de longe, como que afeiçoado por
ele, e com isso descobri seu sobrenome. Numa outra ocasião, eu o segui até sua
casa e dei uma moeda de dez copeques ao zelador, que me informou onde ele
morava, em que andar, se vivia sozinho ou com alguém etc.; em suma, tudo que se
podia saber por um zelador. Certa vez, de manhã cedo, embora eu nunca tivesse
me metido a escritor, me veio subitamente à cabeça a ideia de descrever o
oficial num tom pejorativo e caricato, por meio de uma crônica. Escrevi com
deleite essa crônica. Denunciei"-o, cheguei a caluniá"-lo; no início, disfarcei
seu sobrenome de uma maneira que era possível reconhecê"-lo imediatamente, mas
depois, raciocinando mais claramente, mudei e mandei"-a para a revista
\textit{Anais da pátria}. Mas na época ainda não se publicavam essas
invectivas, e com isso a minha crônica não foi aceita.  Aquilo foi para mim uma
grande lástima. Às vezes a raiva simplesmente me sufocava. Finalmente, decidi
desafiar meu adversário para um duelo.  Elaborei uma carta para ele, belíssima
e encantadora, exortando"-o a se desculpar diante de mim; em caso de uma recusa,
sugeria, de maneira bastante dura, um duelo. A carta foi elaborada de uma forma
tal que, se o oficial entendesse um pouquinho que fosse do \textit{belo e sublime},
viria sem falta até mim, correndo, e se lançaria em meus braços para me
oferecer sua amizade. E como isso seria bom! Como viveríamos bem a partir de
então! Como viveríamos bem! Ele me defenderia com sua eminência; eu o
enobreceria com minha maturidade e\ldots{} bem, e com minhas ideias e muitas
outras coisas que pudesse haver! Imaginem, na época dois anos já haviam se
passado desde que ele me ofendera; o meu desafio era o mais revoltante
anacronismo, a despeito de toda a habilidade de minha carta, que explicavam e
ocultavam o anacronismo. Mas graças a Deus (até hoje em lágrimas agradeço o
Todo"-Poderoso) eu não enviei a carta. Sinto um frio na espinha só de pensar no
que poderia ter acontecido se eu tivesse mandado. E de repente\ldots{} de
repente eu me vinguei da maneira mais simples e mais genial! Ocorreu"-me
subitamente a ideia mais genial. Às vezes, nos feriados, eu ia até a Niêvski
depois das três horas e passeava ao sol. Ou seja, eu não passeava em absoluto,
e sim experimentava incontáveis torturas, humilhações e arroubos de mau humor;
mas era certamente disso que eu precisava. Eu ficava serpenteando de um lado
para o outro da maneira mais horrenda por entre os transeuntes, cedendo
continuamente a passagem ora para generais, ora para oficiais da cavalaria e
hussardos, ora para damas; eu sentia, nesses momentos, dores agudas no coração
e um calor nas costas só de pensar na miséria de minhas roupas, na miséria e na
vulgaridade de minha figurinha bajuladora. Era um tormento terrível, uma
humilhação contínua e insuportável oriunda do pensamento, que logo se
transformava numa sensação contínua e imediata de que eu era uma mosca diante
de todo o mundo, uma mosca nojenta e indecente. Mais inteligente que todos,
mais evoluído que todos, mais nobre que todos --- e isso era óbvio ---, mas uma
mosca que constantemente cedia aos outros, que era humilhada por todos e
ofendida por todos. Com que propósito eu escolhi para mim esse tormento, com
que propósito eu andava pela Niêvski? Isso eu não sei. Mas algo me
\textit{arrastava} para lá a cada possibilidade.

Na época eu já começara a experimentar acessos daqueles prazeres de que
eu falei no primeiro capítulo. Depois da história com o oficial,
sentia"-me ainda mais atraído para lá: era na Niêvski que eu mais o
encontrava, era lá que eu o admirava. Ele ia para lá mais vezes nos
feriados. Embora também cedesse a passagem para generais e para pessoas
especialmente eminentes e também serpenteasse como uma enguia entre
eles, nas pessoas do mesmo nível que nós ou até um pouco acima de nós
ele simplesmente pisava; caminhava diretamente de encontro a elas, como
se diante dele houvesse um espaço vazio, e não cedia passagem de
maneira alguma. Ao olhar para ele, eu me deleitava com a minha própria
raiva e\ldots{} desviava amarguradamente dele toda vez. Ficava atormentado
com o fato de que sequer na rua conseguia ficar de alguma maneira em pé
de igualdade com ele. ``Por que razão você toda vez desvia primeiro?'',
questionava a mim mesmo numa histeria enfurecida, acordando às vezes
quase às três da madrugada. ``Por que precisamente você, e não ele? Não
há uma lei para isso, afinal, não está escrito em lugar nenhum. Que
seja meio a meio, como geralmente acontece quando pessoas delicadas se
encontram: ele cede até a metade e você até a metade, e então ambos
seguem em frente, respeitando mutuamente um ao outro.'' Mas não era
assim, e de qualquer maneira era eu que desviava e ele sequer notava
que eu lhe dera passagem. E então, de repente, o pensamento mais
incrível me veio à mente. ``Mas o que aconteceria'', pensava eu, ``o que
aconteceria se eu me deparasse com ele e\ldots{} eu não desse passagem? Não
desse passagem de propósito, mesmo que fosse preciso empurrá"-lo? Como
seria isso?'' Esse pensamento impertinente aos poucos tomou conta de
mim de tal maneira que eu já não tinha descanso. Sonhava com aquilo
constantemente, terrivelmente, e ia com frequência e de propósito à
Niêvski para imaginar com mais clareza como eu faria quando chegasse o
momento. Estava em êxtase. Cada vez mais me parecia que a ideia era
provável e até mesmo possível. ``É claro que não seria o caso de
empurrá"-lo'', pensava eu, já me enternecendo antecipadamente de alegria,
``mas de simplesmente não dar passagem, de esbarrar nele, não de maneira
que machuque, mas ombro a ombro, exatamente como dita a dignidade; de
maneira que eu dê nele com a mesma força que ele der em mim''. Eu
finalmente me decidi por completo. Mas as preparações tomaram"-me muito
tempo. A primeira coisa era que, no momento da execução, seria
necessário estar com uma aparência das mais dignas e com a roupa mais
cuidadosa. ``De qualquer maneira, se isso por exemplo gerar uma situação
pública (sendo que lá o público é de um extremo luxo: por lá passa a
Condessa, passa o príncipe \textsc{d.}, passa todo o mundo literário), é preciso
estar bem vestido; é algo que impressiona, que nos colocará de certa
forma em pé de igualdade aos olhos da alta sociedade.'' Com esse
objetivo, pedi um adiantamento do salário e comprei um par de luvas
pretas e um chapéu bastante razoável no estabelecimento de Tchúrkin. As
luvas pretas mais respeitáveis e de mais bom tom que as de cor
amarelo"-limão que eu pretendia comprar inicialmente. ``É uma cor muito
chamativa, dá muito a impressão de que a pessoa quer aparecer'', e então
não levei as de cor amarelo"-limão. Eu havia tempos já preparara uma boa
camisa, com abotoaduras brancas de osso; mas o capote era um grande
atraso. O casaco em si era até bem aceitável, esquentava; mas era de
algodão, com gola de pele de guaxinim, o que constituía o cúmulo da
miséria. Era necessário trocar a gola a qualquer custo e conseguir uma
de pele de castor, igual às dos oficiais. Para isso, comecei a andar
pelo Gostiny Dvor\footnote{ Enorme mercado construído no século dezoito e
localizado no centro histórico de São Petersburgo.} 
e, após algumas tentativas, decidi"-me por uma
pele barata de castor de origem alemã. Essas peles alemãs, embora muito
rapidamente fiquem gastas e adquiram um aspecto dos mais miseráveis, no
início, quando novas, parecem até muito decentes; e eu só precisaria
dela uma única vez. Perguntei o preço: ainda assim era cara. Seguindo
um raciocínio lógico, decidi vender minha gola de pele de guaxinim.

A quantia que faltava, para mim bastante significativa, decidi pedir
emprestada junto a Anton Antônitch Siêtotchkin, chefe da minha seção,
homem humilde, porém sério e resoluto, que não emprestava dinheiro para
ninguém, mas a quem, mais ou menos na época de minha admissão no
serviço, fora especialmente recomendado pela figura importante que me
designara para o cargo. Eu me torturava terrivelmente. Pedir dinheiro
para Anton Antônitch me pareceu abominável e vergonhoso. Passei até
duas ou três noites sem dormir; na época eu geralmente dormia pouco,
vivia com febre; meu coração por vezes parecia parar ou começava
subitamente a pular, pular, pular!\ldots{} Anton Antônitch primeiro ficou
surpreso, depois fez uma careta, depois ficou pensativo, mas, apesar de
tudo, emprestou o dinheiro, pedindo"-me um recibo que atestasse seu
direito de receber o dinheiro emprestado dentro de duas semanas,
descontando do meu pagamento. Assim, tudo estava pronto; a bela pele de
castor tomou o trono da miserável pele de guaxinim, e aos poucos
comecei a agir. Não se podia afinal arriscar logo de cara, à toa; era
preciso saber conduzir a questão com habilidade, justamente aos poucos.
Mas reconheço que, após diversas tentativas, comecei a entrar em
desespero: não esbarrávamos um no outro de jeito nenhum, não havia
como! Quão preparado eu estava, eu sequer hesitava! Parecia que
estávamos prestes a esbarrar, mas quando eu via --- novamente eu cedera o
caminho e ele passara sem nem reparar em mim. Eu até rezava ao me     
aproximar dele, para que Deus me inspirasse um pouco de firmeza. Uma
vez pareceu que eu estava de todo decidido, mas no fim das contas eu
apenas caí aos pés dele, porque no último instante, a uma distância
mínima, faltou ânimo. Ele passou calmamente por cima de mim, e eu, como
uma bolinha, rolei para um lado. Naquela noite, fiquei doente de novo,
com febre e delirante. E, subitamente, tudo acabou da melhor maneira
possível. Na madrugada anterior, eu decidira definitivamente não
executar meu nefasto intuito e deixar tudo por isso mesmo; com esse
objetivo, saí pela última vez em direção à Niêvski, apenas para
conferir: conseguiria deixar tudo por aquilo mesmo? De repente, a três
passos do meu inimigo, tomei a decisão, de maneira inesperada, fechei
os olhos e\ldots{} batemos com força ombro a ombro! Eu não me afastei nem um
pouquinho e segui adiante completamente em pé de igualdade! Ele sequer
olhou para os lados e fez cara de que não tinha notado; mas apenas
fingiu, tenho certeza disso. Continuo até agora tendo certeza disso! É
claro que saí perdendo: ele era mais forte, mas essa não era a questão.
A questão era que eu alcançara meu objetivo, mantivera minha dignidade,
não cedera nem um palmo e colocara"-me socialmente, em público, em pé de
igualdade com ele. Voltei para casa totalmente vingado por tudo. Estava
em êxtase. Celebrava cantando árias italianas. É claro que eu não vou
descrever a vocês o que aconteceu comigo depois de três dias; se leram
o primeiro capítulo, ``O subsolo'', poderão adivinhar por conta própria.
O oficial depois foi transferido para algum lugar; já faz uns catorze
anos que não o vejo. O que será dele agora, de meu queridinho? Em quem
estará pisando?



\section{parte II}

Mas acabava"-se a fase de minha perversãozinha, e eu então ficava
terrivelmente enjoado. Começava o arrependimento, que eu tentava
afastar: sentia"-me por demais enjoado. Pouco a pouco, porém, até àquilo
me acostumei. Acostumava"-me a tudo; quer dizer, não é que me
acostumasse, mas de certa forma concordava voluntariamente em suportar.
Mas eu tinha uma saída, algo que resolvia tudo: refugiar"-me no \textit{belo e
sublime}, é claro, em sonhos. Sonhava horrivelmente, sonhava por
três meses ininterruptos, enfurnado em meu canto, e, acreditem, nesses
momentos eu não me parecia com aquele senhor que, em meio à confusão de
seu coração de galinha, costurara à gola de seu casaco a pele de castor
alemã. Subitamente eu me tornava um herói. Eu, então, não aceitaria
sequer uma visita do meu tenente de dez \textit{verchoks}. Não podia
então sequer imaginá"-lo. O que eram meus sonhos e como eu poderia me
satisfazer com eles, é difícil dizer agora, mas então eu me satisfazia
com eles. Até agora, aliás, em parte me satisfaço com eles. Os sonhos
particularmente doces e intensos aconteciam após minha perversãozinha;
causavam arrependimento e lágrimas, maldições e enlevo. Havia momentos
de um arrebatamento tão positivo, de tamanha felicidade, que não se
sentia em mim nem o menor indício de zombaria, juro por Deus. Havia fé,
esperança, amor. Acontece que eu acreditava cegamente que, por algum
milagre, por alguma circunstância externa, tudo aquilo subitamente se
abriria, se ampliaria; que de repente surgisse no horizonte uma
atividade compatível comigo, algo benéfico, sublime e, principalmente,
algo \textit{completamente pronto} para mim (o que
exatamente eu nunca soube, mas principalmente que fosse algo
completamente pronto), e que então eu adentraria o mundo de Deus, quem
sabe até montado num cavalo branco e com uma coroa de louros. Eu não
podia sequer conceber um papel secundário, e exatamente por isso é que,
na realidade, era com muita tranquilidade que ocupava o último dos
papéis. Ou herói ou imundície, não havia meio"-termo. E era isso que me
arruinava, porque na imundície eu me consolava com o pensamento de que
em outro momento seria o herói, e o herói encobria a imundície: um
homem comum, pensava eu, sente vergonha em sujar"-se, enquanto que o
herói é elevado demais para se sujar completamente; e por conseguinte
pode sujar"-se. É notável que esses arroubos do \textit{belo e sublime}
acometiam"-me até durante minha perversãozinha, justamente quando eu já
me encontrava no fundo do abismo. Acometiam"-me em surtos isolados, como
que me lembrando de sua existência, mas sem destruir, porém, com sua
aparição, a minha perversãozinha; pelo contrário, era como se a
reforçasse com o contraste, surgindo na medida exata para um belo
molho. Esse molho consistia em contradições e sofrimentos, numa
torturante análise interna, e todos esses tormentos e tormentozinhos
acrescentavam certo sabor picante, davam até sentido à minha
perversãozinha; em suma, cumpriam perfeitamente a função de um bom
molho. Tudo isso não era destituído de uma certa profundidade. Poderia
eu concordar com uma perversãozinha simples, vulgar, espontânea, de
escrivão, e suportar sobre mim toda essa imundície?! O que é que
poderia então nela me seduzir e me fazer sair à noite na rua? Não, eu
tinha uma nobre escapatória para tudo\ldots{}

Mas quanto amor, senhores, quanto amor senti, então, nesses meus sonhos,
nesses \textit{refúgios no belo e no sublime}: embora fosse um amor
fantástico, embora não se aplicasse na prática a nada de humano, ainda
assim era tão grande esse amor que depois sequer se sentia a
necessidade de aplicá"-lo na prática: seria um luxo excessivo. Tudo,
aliás, terminava da melhor maneira: numa transição preguiçosa e
arrebatadora em direção à arte, ou seja, para as mais belas formas do
ser, formas completas, prontas, fortemente roubadas dos poetas e
romancistas e adaptadas a toda sorte de serviços e exigências. Eu, por
exemplo, a todos triunfava; todos, é evidente, eram reduzidos a cinzas
e forçados a reconhecer voluntariamente todas as minhas qualidades, e
eu perdoava a todos. Apaixonava"-me, era um famoso poeta, um
camareiro"-mor;\footnote{ No original \textit{kamerguer}, título de
nobreza de quarto nível na hierarquia russa de patentes civis e militares.}
recebia uma fortuna imensa, que eu imediatamente doava em prol da
humanidade, ao mesmo tempo em que confessava diante de todo o povo as
minhas ações vergonhosas que, é evidente, não eram apenas ações
vergonhosas, mas que continham em si muitíssimo do \textit{belo e do sublime},
algo manfredesco.\footnote{ Referência ao poema dramático
\textit{Manfred}, escrito por Lord Byron (1788--1824) em 1816--7.} Todos
choravam e me beijavam (do contrário que patetas não seriam), enquanto
eu caminhava descalço e faminto para pregar as novas ideias e derrotar
os retrógrados em Austerlitz.\footnote{ A 2 de dezembro de 1805,
próximo a essa cidade, os exércitos de Napoleão impingiram uma
severa derrota às tropas da Terceira Coalizão, formada por Áustria,
Rússia, Inglaterra, Portugal, Suécia e os reinos de Nápoles e da
Sicília.} Depois começavam a tocar uma marcha, era declarada a anistia,
o papa concordava em partir de Roma para o Brasil;\footnote{ Menção ao
conflito entre Napoleão e o papa Pio \textsc{vii}. Em 1809, este excomungou o
imperador francês, que o manteve como prisioneiro até 1814, quando pôde
afinal retornar a Roma.} depois acontecia um baile para toda a Itália
na Villa Borghese,\footnote{ A Villa Borghese foi construída pelo
arquiteto Flaminio Ponzio (1560--1613) no início do século dezessete.
Duzentos anos mais tarde, pertencia a Camillo Borghese, marido de
Pauline Bonaparte (irmã de Napoleão). A passagem possivelmente se
refere ao baile realizado em 1806, quando da instituição do império
francês.} que ficava às margens do lago de Como, já que para esse fim
exclusivamente o lago de Como era deslocado para Roma; depois havia uma
cena nos arbustos etc. etc\ldots{} Como se vocês não soubessem! Vocês dirão
que é vulgar e infame levar tudo isso a público depois de tantos
arrebatamentos e lágrimas que eu mesmo confessei. Mas por que afinal é
infame? Será possível que pensem que eu me envergonho de tudo isso, e
que tudo isso é mais estúpido do que qualquer coisa em suas vidas,
senhores? Além do mais, acreditem, algumas coisas eram até muito bem
compostas\ldots{} Nem tudo acontecia junto ao lago de Como. Mas por outro
lado, vocês estão certos; é de fato vulgar e infame. E o mais infame de
tudo é que eu tenha começado a me justificar diante de vocês. E é ainda
mais infame que eu esteja fazendo agora essa observação. Mas basta,
aliás, senão isso não acaba nunca: sempre haverá uma coisa mais infame
que a outra\ldots{}

Jamais conseguia passar mais de três meses seguidos sonhando, e começava
a sentir uma necessidade incontrolável de lançar"-me à sociedade. Lançar"-me
à sociedade, para mim, significava visitar meu chefe de seção, Anton
Antônitch Siêtotchkin. Era o único conhecido de longa data que eu
tivera em toda a vida, e agora fico até surpreso com essa constatação.
Mas mesmo a ele eu só visitava quando chegava essa fase e meus sonhos
atingiam uma felicidade tal que eu precisava imediatamente e sem falta
abraçar as pessoas, abraçar toda a humanidade; para isso, era
necessário ter pelo menos uma pessoa de carne e osso, que de fato
existisse. Era necessário visitar Anton Antônitch, contudo, às
terças"-feiras (seu dia de visitas), e, por conseguinte, era necessário
condicionar minha necessidade de abraçar a humanidade para ocorrer na
terça"-feira. Anton Antônitch residia nas Cinco Esquinas,\footnote{
Cruzamento no centro de São Petersburgo entre a avenida Zágorodny, a
travessa Tchernychov (hoje rua Lomonóssov) e as ruas Raziêzjaia e
Tróitskaia (hoje rua Rubinstein).} no quarto andar, num apartamento de
quatro cômodos um menor que o outro, de pé"-direito baixo, e que tinham
o aspecto mais barato e amarelado. Moravam com ele as duas filhas e a
tia delas, que servia chá. Uma das filhas tinha treze anos, a outra
tinha catorze; ambas tinham um narizinho arrebitado, e eu ficava
horrivelmente acanhado diante delas, já que ficavam cochichando uma com
a outra e rindo. O dono da casa geralmente se acomodava no escritório,
num sofá de couro em frente à mesa, na companhia de algum convidado de
cabelos grisalhos, um funcionário do nosso departamento, às vezes até
de outros. Jamais vi mais do que dois ou três convidados, e sempre os
mesmos. Falavam de impostos, das negociações no Senado, do salário, de
promoções, de Sua Excelência, da melhor maneira de agradá"-lo etc. etc.
Eu tinha paciência para ficar sentado como um idiota ao lado daquelas
pessoas por quatro horas, ouvindo o que elas diziam mas sem saber sobre
o que falar com elas. Sentia"-me estupefato, de vez em quando começava a
suar, uma paralisia abatia"-se sobre mim; mas isso tudo era bom e útil.
Ao voltar para casa, por algum tempo eu punha de lado o desejo de
abraçar toda a humanidade.

Eu tinha, porém, mais um conhecido, por assim dizer: Símonov, que fora
meu colega de escola. Eu talvez tivesse muitos colegas de escola em
Petersburgo, mas não me dava com eles e até parei de cumprimentá"-los na
rua. Eu talvez até tenha pedido transferência para outro departamento
apenas para não ficar perto deles e para romper de vez com minha
infância detestável. Maldita seja aquela escola, malditos sejam aqueles
anos de suplício! Resumindo, perdi contato com meus colegas
imediatamente, assim que saí para a liberdade. Restaram umas duas ou
três pessoas que eu ainda cumprimentava ao vê"-las. Entre elas estava
Símonov, que na época da escola não se distinguia em nada, era discreto
e calado; nele, porém, eu discernia uma certa independência de caráter
e até honestidade. Penso até que não era dos mais limitados. Tínhamos
às vezes momentos bastante agradáveis, mas que duravam muito pouco e
que acabavam tornando"-se nebulosos. Ele, pelo visto, ficava incomodado
com essas lembranças, e aparentemente temia que eu recaísse em meu
antigo espírito. Eu suspeitava que ele me achava repugnante, mas mesmo
assim queria visitá"-lo, já que não tinha certeza disso.

Certa vez, então, numa quinta"-feira, não podendo suportar minha solidão
e sabendo que na quinta"-feira as portas de Anton Antônitch estariam
fechadas, lembrei"-me de Símonov. Subindo até o quarto andar, onde ele
morava, pensei justamente que eu incomodava aquele senhor e que era
inútil ir até a casa dele. Mas, uma vez que, como que de propósito,
esse tipo de reflexão sempre acabava fazendo com que eu me metesse numa
situação desagradável, decidi entrar. Fazia quase um ano que eu vira
Símonov pela última vez.


\section{parte III}

Na casa dele, encontrei outros dois de meus colegas de escola.
Conversavam, pelo visto, a respeito de algum assunto importante. Nenhum
deles prestou a mínima atenção à minha chegada, o que era até estranho,
porque eu não os via já fazia alguns anos. Nitidamente me consideravam
a mais ordinária das moscas. Não tinham me maltratado daquela maneira
nem mesmo na escola, embora lá todos me odiassem. É claro que eu
entendia que agora eles deviam me desprezar pelo fracasso de minha
carreira profissional e pelo fato de que eu decaíra muito, me vestia
muito mal etc., o que aos olhos deles era o símbolo de minha
incapacidade e insignificância. Mas ainda assim eu não esperava tamanho
desprezo. Símonov ficou até mesmo surpreso com minha chegada.
Antigamente ele também parecia surpreso com minha chegada. Tudo isso me
deixava desconcertado; sentei"-me um tanto aborrecido e comecei a
escutar o que eles diziam.

Travavam um debate muito sério, até acalorado, sobre um almoço de despedida que
esses senhores queriam organizar, já no dia seguinte, para seu colega Zverkov,
que servia como oficial e estava de partida para uma província distante.
\textit{Monsieur} Zverkov fora o tempo todo também meu colega de escola.
Comecei a odiá"-lo particularmente nos últimos anos. Nos primeiros anos ele era
apenas um menino bonitinho e esperto que todos adoravam. Eu, porém, o odiava
também nos primeiros anos, justamente pelo fato de que era um menino bonitinho
e esperto.  Tirava constantemente notas baixas, que foram piorando conforme o
tempo passava; mas conseguiu concluir com êxito a escola, já que era protegido
de alguém. Em seu último ano na nossa escola, recebeu uma herança de duzentas
almas, e, uma vez que quase todos nós éramos pobres, começou até a
vangloriar"-se diante de nós. Era um tipo vulgar ao extremo, mas, ao mesmo
tempo, um bom rapaz, mesmo quando se vangloriava. E, a despeito das formas
superficiais, fantásticas e empoladas de honra e pompa, todos nós, com exceção
de alguns poucos, chegávamos mesmo a bajular Zverkov quanto mais ele se
vangloriava. E não era para tirar algum tipo de vantagem que o bajulavam, mas
por ele ser uma pessoa favorecida pela natureza. Além disso, era de certa forma
consensual considerar Zverkov um especialista em matéria de trato e de boas
maneiras. Esse último ponto me enfurecia particularmente. Eu odiava o som
estridente e autoconfiante de sua voz, a forma como adorava seus próprios
gracejos, que eram terrivelmente estúpidos, embora ele tivesse uma linguagem
ousada; eu odiava seu rosto bonito, mas estúpido (que eu, porém, trocaria de
bom grado pelo meu rosto inteligente) e seus modos desenvoltos de oficial,
próprios dos anos quarenta. Eu odiava quando ele falava de seu sucesso, no
futuro, com as mulheres (ele não ousava começar a tratar com as mulheres sem
ter ainda suas dragonas de oficial, e as esperava com impaciência), e quando
dizia que pretendia bater"-se a todo instante em duelos. Lembro"-me de quando eu,
que ficava sempre calado, de repente ataquei Zverkov quando ele, discutindo
certa vez no intervalo com os colegas sobre suas futuras aventuras e finalmente
empolgando"-se como um jovem cãozinho ao sol, declarou, de repente, que nenhuma
das moças camponesas de sua aldeia ficaria sem sua atenção, que aquilo era o
\textit{droit de seigneur},\footnote{ No original, em francês, ``direito de
senhor''. Refere"-se ao suposto costume medieval --- em latim chamado de
\textit{jus primae noctis} --- segundo o qual as camponesas deveriam passar sua
lua de mel com seu senhor, e não com seu marido.} e que os mujiques, se
ousassem protestar, seriam todos açoitados e teriam todos eles, aqueles
canalhas barbudos, seu \textit{obrok}\footnote{ Tributo pago pelos camponeses
ao dono das terras.} dobrado. Os grosseirões aplaudiram; já eu o ataquei, mas
não em absoluto por pena das moças e de seus pais, e sim simplesmente pelo fato
de que haviam aplaudido daquela maneira aquele inútil. Eu venci, então, mas
Zverkov, embora estúpido, era jovial e insolente, e por isso se safou rindo, e
de tal forma que eu, na verdade, acabei não vencendo em absoluto: o riso ficou
do lado dele.  Depois, ele ainda me venceu algumas vezes, mas sem maldade, como
que brincando, de passagem, rindo. Eu maldosamente e com desprezo deixava"-o sem
resposta. Após concluirmos os estudos, ele fez menção de se aproximar de mim;
não me opus, mesmo porque me sentia lisonjeado; mas rápida e naturalmente nós
nos afastamos. Mais tarde, ouvi falar de seu sucesso, no quartel, como tenente
e de suas \textit{farras}. Depois começaram outros boatos: de como ele
\textit{progredia} no serviço. Na rua ele já não me cumprimentava, e eu
suspeitava que ele temia comprometer"-se por cumprimentar uma pessoa tão
insignificante quanto eu. Eu também o havia visto uma vez no teatro, na
terceira fileira de camarotes, já usando insígnias. Ele bajulava e se pavoneava
diante das filhas de um velho general. Nos últimos três anos ele decaíra muito,
embora estivesse, como antes, bastante bonito e tratável; parecia inchado,
começara a engordar; era visível que ficaria completamente obeso antes dos
trinta. Era para esse Zverkov, que finalmente partia, que os nossos colegas
queriam organizar o almoço. Ao longo daqueles três anos, eles mantiveram
contato com ele, embora eles mesmos, no íntimo, não pensassem estar em pé de
igualdade com ele, tenho certeza disso.

Um dos convidados de Símonov era Ferfítchkin, russo de origem alemã: de
baixa estatura, rosto simiesco, um imbecil que sempre ria de todos, meu
pior inimigo desde os primeiros anos de escola, infame, insolente, um
fanfarrãozinho que aparentava uma vaidade das mais melindradas, embora
no fundo fosse, é claro, um covardão. Era um daqueles admiradores de
Zverkov que o bajulavam por interesse e que frequentemente pegavam com
ele dinheiro emprestado. O outro convidado de Símonov, Trudoliúbov, era
um indivíduo pouco digno de nota, um jovem militar de alta estatura e
fisionomia fria, bastante íntegro, mas que reverenciava qualquer tipo
de sucesso e que só pensava em promoções. Tratava Zverkov como se fosse
algum parente distante, e isso, por mais estúpido que possa parecer,
dava a ele certa importância entre nós. Sempre me considerou um nada;
tratava"-me de maneira não exatamente cortês, mas tolerável.

--- Se cada um der sete rublos --- começou Trudoliúbov ---, nós três juntos
somamos vinte e um rublinhos. Dá para comer bem. Zverkov, é claro, não
vai pagar.

--- Mas é claro, se nós é que estamos convidando --- decidiu Símonov.

--- Não é possível que vocês estejam pensando --- intrometeu"-se Ferfítchkin
com arrogância e ardor, como um criado atrevido gabando"-se das medalhas
de seu senhor general --- que o Zverkov vai nos deixar pagar tudo
sozinhos. Ele aceitará por delicadeza, mas por outro lado vai pagar do
próprio bolso uma \textit{meia dúzia}.

--- Mas meia dúzia para nós quatro --- notou Trudoliúbov, dando atenção
apenas à meia dúzia.

--- Pois bem, três. Com Zverkov, quatro. Vinte e um rublos no
Hôtel de Paris, amanhã às cinco horas --- concluiu
finalmente Símonov, que fora escolhido como organizador.

--- Como assim vinte e um? --- disse eu com certa agitação, até mesmo
aparentando estar ofendido. --- Contando comigo não serão vinte e um, mas
sim vinte e oito rublos.


Eu tinha a impressão de que me oferecer para ir daquela maneira tão
súbita e inesperada seria até muito bonito, e de que todos eles
imediatamente se dariam por vencidos e olhariam para mim com respeito.

--- Você também quer ir, então? --- indagou com insatisfação Símonov, como
que evitando olhar para mim. Ele me conhecia como a palma da mão.

Fiquei enraivecido por ele me conhecer como a palma da mão.

--- E por que não? Ao que me parece também fui colega, e reconheço que
fico até ofendido por terem me deixado de lado --- comecei novamente a
me exaltar.

--- E onde nós poderíamos encontrá"-lo? --- intrometeu"-se Ferfítchkin de
forma grosseira.

--- Você nunca se deu bem com o Zverkov --- acrescentou Trudoliúbov,
fechando o semblante. Mas eu já me aferrara àquilo e não queria largar.

--- Creio que ninguém tem o direito de julgar a esse respeito --- objetei
com a voz trêmula, como se tivesse acontecido sabe Deus o quê. --- Talvez
seja exatamente por isso que eu agora queira, porque antes não nos
dávamos bem.

--- Mas quem é que entende você\ldots{} Com esses pensamentos elevados\ldots{} ---
disse Trudoliúbov, rindo.

--- Vamos incluí"-lo --- decidiu Símonov, dirigindo"-se a mim. --- Amanhã às
cinco horas no Hôtel de Paris; não se engane.

--- O dinheiro! --- fez menção de começar Ferfítchkin a meia"-voz, acenando
para Símonov em minha direção, mas interrompeu o que dizia, porque até
Símonov ficou constrangido.

--- Basta --- disse Trudoliúbov, levantando"-se. --- Se ele quer tanto assim
ir, que vá.

--- É que temos nosso círculo de amigos --- irritou"-se Ferfítchkin, também
pegando seu chapéu. --- Não é uma reunião oficial. Talvez nós não
queiramos em absoluto que você vá\ldots{}

Saíram. Ferfítchkin, ao sair, sequer me cumprimentou; Trudoliúbov
inclinou"-se de leve, sem me olhar. Símonov, com quem fiquei a sós,
parecia irritado e perplexo, e olhava para mim de uma maneira estranha.
Não se sentou e não me convidou a fazê"-lo.

--- Hum\ldots{} Sim\ldots{} Amanhã, então. Você dará o dinheiro agora? Não estou
dizendo para saber ao certo --- murmurou ele constrangido.

Fiquei enrubescido, mas ao me enrubescer, lembrei"-me de que, desde
tempos imemoriais, devia a Símonov quinze rublos, de que aliás eu
jamais me esquecia, mas que, porém, jamais devolvia.

--- Você tem que concordar, Símonov, que eu não tinha como saber ao vir
para cá\ldots{} E lamento muito por ter esquecido\ldots{}

--- Tudo bem, tudo bem, não importa. Pague amanhã depois do almoço. Eu só
queria saber\ldots{} Mas por favor\ldots{}

Ele interrompeu sua fala e pôs"-se a andar pela sala ainda mais irritado.
Ao caminhar, começou a pisar com força com os saltos dos sapatos.

--- Não estarei atrapalhando? --- perguntei após um silêncio breve.

--- Oh, não! --- disse ele, subitamente agitado. --- Para dizer a verdade,
sim. Sabe, eu ainda precisava passar\ldots{} Num lugar aqui perto\ldots{} ---
acrescentou ele como que se desculpando e, em parte, num tom de voz
envergonhado.

--- Ah, meu Deus! Por que você não \textit{di"-i"-isse}?! --- gritei eu, pegando o
chapéu, com um ar surpreso, porém desenvolto, que me acometera sabe Deus
como.

--- É aqui perto\ldots{} Logo ao lado\ldots{} --- repetiu Símonov, acompanhando"-me até
a saída e com um ar agitado que não combinava nem um pouco com ele. ---
Então amanhã às cinco em ponto! --- gritou ele na escada: parecia muito
satisfeito com a minha saída. Eu estava enfurecido.

--- Mas eu tinha, tinha que me intrometer! --- dizia eu, rangendo os dentes,
ao caminhar pela rua --- E para um canalha desses, o porco do Zverkov! É
claro que eu não devo ir; é claro que eu tenho que deixar para lá: sou
obrigado a ir, por acaso? Amanhã mesmo informarei ao Símonov pelo
correio da cidade\ldots{}

Mas eu me enfurecera justamente por saber que certamente iria; iria de
propósito; e quanto mais indelicada, quanto mais inconveniente fosse
minha presença, mais provável era que eu fosse.

Havia até um bom empecilho para que eu não fosse: não tinha dinheiro.
Somando tudo, tinha comigo nove rublos. Mas desse dinheiro, precisava
dar sete rublos já no dia seguinte como pagamento mensal ao Apollon,
meu criado, que vivia comigo por sete rublos, sem contar a comida.

Não dar o dinheiro seria impossível, a julgar pelo caráter de Apollon.
Mas desse canalha, dessa minha chaga, falarei em algum momento mais
adiante.

Eu porém sabia que de qualquer maneira não daria o dinheiro e que sem
dúvida acabaria indo.

Naquela noite tive os sonhos mais descabidos. Não é de estranhar: a
noite inteira me oprimiam as lembranças dos anos de tormentos em minha
vida escolar, e eu não conseguia me desvencilhar delas. Fui enfurnado
naquela escola por meus parentes distantes, dos quais eu dependia e de
quem nunca mais soube desde então; enfiaram"-me lá, órfão, já
amedrontado por seus reproches, já melancólico, taciturno e ferozmente
desconfiado de tudo. Meus colegas me receberam com gracejos maldosos e
impiedosos pelo fato de que eu não me parecia com nenhum deles. Mas eu
não podia suportar os gracejos; eu não conseguia me adaptar como eles
se adaptavam uns aos outros. Peguei ódio deles instantaneamente, e me
isolei de todos numa soberba assustadiça, ofendida e desmesurada. A
grosseria deles me indignava. Riam com cinismo da minha cara, de minha
figura desajeitada; e no entanto, que rostos estúpidos eles mesmos
tinham! Em nossa escola, as expressões dos rostos pareciam apalermar"-se
e degenerar"-se. Quantas crianças bonitas não entravam em nossa escola.
E, depois de alguns anos, era até repugnante olhar para elas. Mesmo aos
dezesseis anos eu ainda ficava terrivelmente pasmo com eles; já então
ficava assombrado com a pequenez de seu pensamento, a estupidez de seus
afazeres, de seus jogos, de suas conversas. Eles não entendiam coisas
tão imprescindíveis, não se interessavam por assuntos tão inspiradores
e surpreendentes que, involuntariamente, comecei a considerá"-los
inferiores a mim. Não era vaidade ofendida que me levava a isso, e,
pelo amor de Deus, não me venham com aquelas objeções de sempre, que já
me enjoaram: ``que eu apenas sonhava, enquanto eles realmente
compreendiam a vida''. Eles não entendiam nada, vida real nenhuma, e,
juro, era isso que mais me indignava neles. Pelo contrário, a realidade
mais notória, mais evidente era entendida por eles de uma maneira
excepcionalmente estúpida, e já àquela altura haviam se acostumado a
reverenciar apenas o sucesso. De tudo que era justo, mas diminuído e
esquecido, eles riam da forma mais cruel e vergonhosa. Entendiam altas
posições como inteligência; aos dezesseis anos já falavam de benesses.
É claro que muito disso era por estupidez, pelos maus exemplos que os
cercavam continuamente na infância e na adolescência. Eram
monstruosamente devassos. É claro que muito disso não passava de
aparências e de um cinismo afetado; é claro que muito de infantilidade
e um certo frescor faiscavam neles mesmo em sua devassidão; mas mesmo o
frescor era neles pouco atraente e manifestava"-se envolto em certa
galhofaria. Eu os odiava terrivelmente, embora possivelmente fosse
ainda pior que eles. Pagavam"-me na mesma moeda, e não escondiam sua
repulsa em relação a mim. Mas deles eu já não desejava o amor; pelo
contrário, ansiava constantemente que me humilhassem. Para me livrar de
seus gracejos, comecei propositalmente a estudar cada vez mais, e
acabei conseguindo me colocar entre os primeiros da classe. Aquilo os
impressionou. Além disso, eles começaram aos poucos a compreender que
eu começara a ler livros que eles não poderiam ler, a entender coisas
(que não faziam parte de nosso currículo) de que eles nunca sequer
ouviram falar. Encaravam isso com um ar de ferocidade, de malícia, mas
acabavam se sujeitando moralmente, ainda mais pelo fato de que até os
professores me davam atenção por conta disso. Os gracejos pararam, mas
a hostilidade permaneceu, e as relações que se estabeleceram eram frias
e tensas. No fim, eu já não podia me conter: com os anos, desenvolvera
uma necessidade de pessoas, de amigos. Tentei começar a me aproximar
dos outros; mas essa aproximação sempre parecia artificial, e assim
evidentemente acabava. Tive certa vez um amigo. Mas no fundo eu já era
um déspota; queria dominar sua alma de maneira irrestrita; queria
inspirar nele o desprezo pelo meio que o cercava; exigia dele um
rompimento arrogante e definitivo com esse meio. Eu o assustei com
minha amizade apaixonada; por mim ele chorou, teve convulsões; era uma
alma ingênua e dedicada; mas quando ele se entregou a mim por inteiro,
imediatamente passei a odiá"-lo e o afastei de mim, como se ele me fosse
necessário apenas para que eu obtivesse a vitória sobre ele, apenas
para sua submissão. Mas eu não podia vencer todos; meu amigo também não
se parecia com nenhum deles e era uma raríssima exceção. Minha primeira
atitude após o término da escola foi abandonar o serviço para qual fora
designado, para romper todos os laços, amaldiçoar o passado e cobri"-lo
de pó\ldots{} E agora quem sabe por que diabos eu me arrastei até a casa do
tal Símonov\ldots{}!

Pulei da cama de manhã bem cedo; levantei já agitado, como se tudo fosse
começar a acontecer imediatamente. Mas eu acreditava que uma
reviravolta radical estava prestes a acontecer em minha vida, que
aconteceria naquele mesmo dia. Talvez por falta de hábito, durante toda
a minha vida, qualquer acontecimento externo, por insignificante que
fosse, fazia"-me crer que uma reviravolta estava para acontecer em minha
vida. Fui, porém, para o serviço como sempre, mas escapei para casa
duas horas antes para me preparar. O mais importante, pensava eu, era
não chegar antes de todos, do contrário pensariam que eu estava por
demais empolgado. Mas havia milhares de coisas importantes como essa, e
todas elas me deixavam agitado ao ponto de me enfraquecer. Limpei por
conta própria minhas botas mais uma vez; nada no mundo teria feito
Apollon limpá"-las duas vezes no mesmo dia, por achar que aquilo estava
além de seu trabalho. Limpei eu mesmo, após ter roubado uma escova na
antessala para que ele não percebesse de alguma maneira e começasse
depois a desdenhar de mim. Em seguida, examinei detalhadamente minha
roupa e achei tudo muito velho, surrado, puído. Tornara"-me muito
desleixado. O uniforme talvez estivesse em bom estado, mas eu não podia
ir almoçar de uniforme. O pior de tudo é que nas calças, bem no joelho,
havia uma imensa mancha amarela. Eu pressentia que somente aquela
mancha já me privaria de uns nove décimos de minha dignidade. Eu também
sabia que era muito baixo pensar daquela maneira. ``Mas agora não é hora
de ficar pensando; agora é hora da realidade'', pensei eu, caindo em
desânimo. Mesmo então, sabia também perfeitamente que estava exagerando
extraordinariamente todos esses fatos; mas o que se podia fazer: já não
podia me conter, eu tremia de febre. Com desespero eu imaginava a
arrogância e a frieza com que me receberia aquele \textit{canalha} do Zverkov;
o desprezo obtuso e impenetrável com que olharia para mim o estúpido
Trudoliúbov; a maneira desagradável e insolente com que riria, às
minhas custas, o inútil Ferfítchkin, tudo para agradar o Zverkov; o
quão bem perceberia tudo isso Símonov, e como me desprezaria pela
baixeza de minha vaidade e de minha covardia, e, principalmente, o
quanto tudo aquilo seria mesquinho, não literário, ordinário. É claro
que o melhor seria não ir em absoluto. Mas isso já era de todo
impossível: se começasse a me sentir tentado, eu me atiraria com tudo,
de cabeça. Depois ficaria a vida toda me afrontando: ``Ficou com medo,
ficou com medo da realidade, ficou com medo!'' Pelo contrário, queria
muitíssimo provar para toda aquela \textit{chusma}, que eu não era em absoluto
o covarde que eu mesmo imaginava ser. E mais ainda: no mais extremo
paroxismo dessa covardia febril, sonhava em sair por cima, vencê"-los,
arrebatá"-los, forçá"-los a me amar; ao menos ``por meus pensamentos
elevados e por minha indubitável perspicácia''. Eles abandonariam
Zverkov, ele seria posto de lado, ficaria calado e envergonhado, e eu
destruiria Zverkov. Depois, talvez fizesse as pazes com ele e bebesse
para selar a amizade, mas o que era o pior e mais ofensivo para mim era
o fato de que eu na época já sabia, sabia com toda a certeza, que na
verdade não precisava de nada daquilo, que na verdade eu não desejava
em absoluto destruí"-los, submetê"-los, atraí"-los, e que eu mesmo não
daria um tostão sequer pelos resultados, se eu porventura conseguisse
obtê"-los. Ah, como eu pedi a Deus para que aquele dia passasse
depressa! Com uma tristeza indescritível eu me aproximei da janela,
abri o postigo e observei, em meio às opacas brumas, a neve molhada que
caía pesadamente\ldots{}

Finalmente, meu imprestável relógio de parede bateu as cinco. Peguei meu
chapéu e, tentando não olhar para Apollon --- que estava desde a manhã
esperando a entrega do salário, mas que por orgulho não queria ser o
primeiro a falar a respeito ---, deslizei por ele, passei pela porta e
parti, numa carruagem de luxo que eu alugara pelos meus últimos
cinquenta copeques, em direção ao Hôtel de \mbox{Paris}.


\section{parte IV}

Já na véspera eu sabia que seria o primeiro a chegar. Mas a questão já
não era ser o primeiro.

Não apenas não havia ninguém, como até mesmo quase não encontrei nosso
cômodo. A mesa ainda não estava totalmente posta. O que significava
aquilo? Depois de muitas perguntas, consegui finalmente arrancar dos
criados que a mesa para o almoço fora reservada para as seis horas, e
não para as cinco. O que confirmaram no bufê. Comecei a ficar com
vergonha de perguntar. Eram ainda cinco e vinte e cinco. Se eles
mudaram a hora, deveriam de qualquer maneira ter avisado; para isso
temos o correio da cidade, e não me sujeitar àquela \textit{vergonha} diante
de mim e\ldots{} e pelo menos diante dos criados. Sentei"-me; o criado começou
a pôr a mesa; em sua presença, sentia"-me ainda mais ofendido. Pouco
antes das seis, além das luminárias já acesas, trouxeram velas para o
cômodo. O criado não pensou, porém, em trazê"-las no momento em que eu
chegara. No cômodo vizinho almoçavam, em mesas diferentes, dois
hóspedes algo soturnos, aparentemente irritados e em silêncio. Em um
dos cômodos mais distantes, faziam muito barulho; até mesmo gritavam;
ouvia"-se a gargalhada de uma verdadeira tropa; ouviam"-se uns
desagradáveis guinchos em francês: era um almoço com senhoras.
Resumindo, era tudo nauseante. Raras vezes passei por um momento mais
desagradável, de maneira que, quando eles apareceram todos juntos,
precisamente às seis horas, eu, no primeiro instante, fiquei contente
como se fossem meus libertadores, e por pouco não me esqueci de que
precisava parecer ofendido.

Zverkov entrou na frente de todos, nitidamente no comando. Ele e todos
os outros riam; mas, ao me ver, Zverkov fez ar de importante,
aproximou"-se sem pressa, inclinando"-se um pouco da cintura para cima,
como que se exibindo, e apertou"-me a mão, gentilmente, mas não muito,
com uma polidez cautelosa, quase de general, como se, ao apertar a
minha mão, se protegesse de alguma coisa. Eu imaginara, ao contrário
daquilo, que ele, logo ao entrar, começaria a rir da mesma maneira que
antes, um riso fininho, com guinchos, e que já nas primeiras palavras
começaria com suas brincadeiras e chistes sem graça. Eu me preparara
para eles desde a noite anterior, mas de forma alguma poderia ter
esperado tamanha arrogância, uma afabilidade tão soberana. Então agora
ele já se considerava completa e infinitamente superior a mim em todos
os sentidos? Se ao menos ele quisesse apenas me ofender com toda aquela
pose de general, estaria tudo bem, pensei; de qualquer maneira eu não
ligaria. Mas e se de fato, sem qualquer intenção de ofender, àquela
cabecinha de vento tivesse ocorrido a sério a ideia de que ele era
infinitamente superior a mim e que não poderia olhar para mim de outra
maneira que não como um protetor? Bastou aquela suposição para eu
começar a me sentir sufocado.

--- Fiquei surpreso ao saber de seu desejo de participar --- começou ele,
ciciando, sibilando e arrastando as palavras, o que antes não acontecia
com ele. --- Nós temos nos encontrado muito pouco com você. Você nos
evita. Mas não precisa. Não somos tão terríveis quanto lhe parece. Bem,
de qualquer maneira fico contente que re"-a"-te"-mos\ldots{}

E virou"-se desdenhosamente para colocar seu chapéu na janela.

--- Está há muito tempo esperando? --- perguntou Trudoliúbov.

--- Cheguei pontualmente às cinco horas, como foi estabelecido ontem ---
respondi, numa voz alta e irritada, que sugeria uma explosão iminente.

--- Mas você não o avisou de que o horário havia mudado? --- dirigiu"-se
Trudoliúbov a Símonov.

--- Não avisei. Esqueci --- respondeu o outro, mas sem qualquer
arrependimento, e, sem sequer pedir desculpas a mim, foi providenciar
os petiscos.

--- Então você já está aqui há uma hora, pobrezinho! --- gritou
zombeteiramente Zverkov, já que, em sua concepção, aquilo devia ser de
fato engraçadíssimo. O canalha Ferfítchkin o acompanhava, com sua voz
infame e sonora, como a de um cachorrinho. Ele também parecia achar
engraçada e constrangedora a minha situação.

--- Não tem graça nenhuma! --- gritei para Ferfítchkin, irritando"-me mais e
mais. --- Os culpados são outros, não eu. Fizeram pouco caso de me
avisar. Isso\ldots{} isso\ldots{} isso é\ldots{} simplesmente um absurdo.

--- Não apenas um absurdo, é mais que isso --- resmungou Trudoliúbov,
ingenuamente me defendendo. --- Você está sendo brando demais. É
simplesmente uma descortesia. É claro que não intencional. E como você,
Símonov\ldots{} Hum!

--- Se fizessem isso comigo --- notou Ferfítchkin --- eu teria\ldots{}

--- Mas você deveria ter mandado trazerem alguma coisa --- interrompeu
Zverkov ---, ou simplesmente pedido para almoçar sem nos esperar.

--- Mas vocês hão de concordar que eu poderia ter feito isso sem qualquer
autorização sua --- cortei eu. --- Se eu esperei é porque\ldots{}

--- Vamos nos sentar, senhores --- gritou Símonov ao entrar. --- Está tudo
pronto; fico responsável pelo champanhe, está geladíssimo\ldots{} E se eu
não conheço sua casa, onde é que eu poderia encontrá"-lo? --- dirigiu"-se
a mim subitamente, mas de novo sem olhar para mim. Nitidamente ele
tinha algo contra mim. Pelo visto, depois do dia anterior tinha
repensado algumas coisas.

Todos se sentaram; eu também me sentei. A mesa era redonda. À minha
esquerda ficou Trudoliúbov, à direita Símonov. Zverkov sentou"-se no
lado oposto; Ferfítchkin ao lado, entre ele e Trudoliúbov.

--- Di"-i"-iga"-me, você\ldots{} está trabalhando em algum departamento público? ---
disse Zverkov, ainda me dando atenção. Vendo que eu estava
constrangido, ele realmente pensava que era preciso me pajear e, por
assim dizer, me animar. ``Mas será que ele quer que eu dê com uma
garrafa nele?'', pensei, enfurecido. Por falta de costume, eu me
irritava de maneira excessivamente rápida.

--- Na repartição de\ldots{} --- respondi de maneira entrecortada, olhando para o
prato.

--- E\ldots{} o s"-senhor acha prov"-veitoso? Di"-iga, e o que o mo"-otivou a
deixar seu antigo serviço?

--- O que me mo"-o"-otivou foi que eu quis deixar meu antigo serviço ---
estiquei eu três vezes mais que ele, já quase sem me conter.
Ferfítchkin bufou. Símonov olhou ironicamente para mim; Trudoliúbov
parou de comer e começou a me examinar com curiosidade.

Zverkov ficou incomodado, mas não quis demonstrar.

--- Be"-e"-em, e de quanto é seu ordenado?

--- Que ordenado?

--- O seu s"-salário.

--- Mas que interrogatório é esse!

Eu porém acabei dizendo quanto recebia de salário. Fiquei terrivelmente
corado.

--- É modesto --- notou Zverkov com ar de importância.

--- Pois é, não dá para almoçar em cafés restaurantes! --- acrescentou
Ferfítchkin de maneira impertinente.

--- A meu ver, é até simplesmente miserável --- notou em tom sério
Trudoliúbov.

--- E como você emagreceu, como mudou\ldots{} desde aquela época\ldots{} ---
acrescentou Zverkov, já com algum veneno, com uma compaixão atrevida,
examinando"-me e a meu traje.

--- Mas basta de constrangê"-lo --- gritou Ferfítchkin, dando risinhos.

--- Meu caro senhor, saiba que eu não estou constrangido --- estourei,
afinal. --- Ouça bem! Estou almoçando aqui, num \textit{café restaurante}, com o
meu dinheiro, com o meu, e não com dinheiro alheio, perceba,
\textit{monsieur} Ferfítchkin.

--- Co"-omo! E quem aqui não está almoçando com seu próprio dinheiro? Você
parece que\ldots{} --- disse em tom ferrenho Ferfítchkin, corando como um
pimentão e olhando encolerizado em meus olhos.

--- Be"-em --- respondi, sentindo que fora longe demais. --- Mas creio que
seria melhor começarmos uma conversa mais inteligente.

--- Pelo visto você pretende demonstrar sua inteligência?

--- Não se preocupem, isso aqui seria de todo desnecessário.

--- Mas então, meu caro senhor, o que é que você está aí grasnando, hein?
Por acaso perdeu um parafuso nesse seu \textit{lepartamento}?

--- Basta, senhores, basta! --- gritou Zverkov imperiosamente.

--- Como isso é estúpido! --- resmungou Símonov.

--- De fato é estúpido, nós nos reunimos aqui, na companhia de amigos,
para nos despedir de um bom camarada que vai partir, e você querendo
acertar as contas --- começou Trudoliúbov, dirigindo"-se asperamente
apenas para mim. --- Foi você mesmo quem se convidou ontem, então não
perturbe a harmonia do grupo\ldots{}

--- Basta, basta --- gritou Zverkov. --- Parem, senhores, isso não vai bem. É
melhor eu contar a vocês como por pouco não me casei anteontem\ldots{}

E então começou uma fanfarrice qualquer a respeito de como esse senhor
dois dias antes por pouco não se casara. Sobre o casamento, porém, não
havia uma palavra sequer, mas no relato surgiam a todo instante
generais, coronéis e até moços de câmara,\footnote{ No original,
\textit{kamer"-iúnker}, título de nobreza de quinto a oitavo nível na
hierarquia russa de patentes civis e militares. Foi o título concedido a
Púchkin.} sendo que Zverkov parecia ser um dos mais importantes entre
eles. Começou o riso de aprovação; Ferfítchkin até guinchava.

Todos me largaram, e eu fiquei ali sentado, abatido e devastado.

``Senhor, isso lá é companhia para mim!'' pensei. ``E que papel de idiota
fiz diante deles! Fui muito permissivo com Ferfítchkin, porém. Pensam
esses brutamontes que me deram a honra de me conceder um lugar à mesa,
mas não entendem que fui eu, eu é que estou dando a eles a honra, não
eles a mim! ``Emagreceu! Sua roupa!'' Ah, essas malditas calças! Zverkov
agora há pouco já percebeu a mancha amarela no joelho\ldots{} E para que
ficar aqui?! Devo agora mesmo, neste momento levantar da mesa, pegar o
chapéu e simplesmente sair sem dizer uma palavra\ldots{} Por desprezo! E
amanhã teremos que pelo menos duelar. Canalhas. Sete rublos é que não
vão me fazer falta. Talvez pensem que\ldots{} Mas o diabo que os carregue!
Não me fazem falta sete rublos! Sairei agora mesmo!\ldots{}''

É claro que eu fiquei.

Em desespero, bebi copos e copos de Lafite e xerez. Pela falta de
hábito, fiquei bêbado depressa, e com a embriaguez cresceu também o
despeito. Fiquei de repente com vontade de ofendê"-los todos da maneira
mais temerária e depois partir. Aproveitar uma brecha na conversa e
revelar"-me; e eles que dissessem: é um sujeito ridículo, mas pelo menos
é inteligente\ldots{} e\ldots{} e\ldots{} resumindo, ao diabo com eles!

Eu olhava para cada um deles com um ar insolente e olhos embriagados.
Mas eles pareciam já ter me esquecido por completo. Faziam barulho e
gritavam alegremente. Só Zverkov falava. Comecei a prestar atenção.
Zverkov falava de uma dama pomposa qualquer, que ele conseguira
finalmente fazer com que ela se declarasse (é claro que mentia aos
quatro ventos) e que nesse assunto o ajudava particularmente seu amigo
íntimo, um principezinho qualquer, o hussardo Kôlia, que tinha três mil
almas.

--- E no entanto esse Kôlia, que tem três mil almas, não deu as caras aqui
para se despedir de você --- interrompi subitamente a conversa. Num
instante todos se calaram.

--- A essa hora já está bêbado --- concordou finalmente em me notar
Trudoliúbov, lançando um olhar de desprezo em minha direção. Zverkov me
observava em silêncio, como se eu fosse um inseto. Baixei os olhos.
Símonov pôs"-se rapidamente a servir champanhe.

Trudoliúbov levantou seu copo, e todos o imitaram, menos eu.

--- À sua saúde, e uma boa viagem! --- gritou ele para Zverkov. --- Aos velhos
tempos, senhores, e ao nosso futuro, viva!

Todos beberam e correram para beijar Zverkov. Não me movi; o copo cheio
permanecia diante de mim, intocado.

--- E você, não vai beber? --- rugiu Trudoliúbov, perdendo a paciência e
dirigindo"-se ameaçadoramente a mim.

--- Eu gostaria de fazer um brinde especialmente de minha parte\ldots{} e então
beberei, senhor Trudoliúbov.

--- Brutalhão nojento! --- resmungou Símonov.

Endireitei"-me na cadeira e peguei o copo em delírio, preparando"-me para
algo incomum, mas sem saber ainda ao certo o que eu diria.

--- \textit{Silence}! --- gritou Ferfítchkin. --- Aí vem algo inteligente!

Zverkov esperava, bastante sério, já compreendendo o que se passava.

--- Senhor tenente Zverkov --- comecei. --- Você sabe que eu odeio o
empolamento, as pessoas empoladas e as cinturas apertadas\ldots{} Este é o
primeiro ponto, e dele segue o segundo.

Todos se remexeram, agitados.

--- O segundo ponto: odeio as aventuras amorosas e os aventureiros.
Especialmente os aventureiros!

--- O terceiro ponto: amo a verdade, a sinceridade e a honestidade ---
continuei quase maquinalmente, porque eu mesmo começara a gelar de
terror, sem compreender como eu poderia falar daquela maneira\ldots{} --- Amo
o pensamento, \textit{monsieur} Zverkov; amo a verdadeira camaradagem,
em pé de igualdade, e não\ldots{} Hum\ldots{} Amo\ldots{} Mas pensando bem, por que
não? Eu também beberei à sua saúde, \textit{monsieur} Zverkov. Seduza
as circassianas, atire nos inimigos da pátria e\ldots{} e\ldots{} À sua saúde,
\textit{monsieur} Zverkov!

Zverkov levantou"-se de sua cadeira, inclinou"-se em minha direção e
disse:

--- Agradeço"-lhe muito.

Estava terrivelmente ofendido, até empalidecera.

--- Ao diabo com você --- berrou Trudoliúbov, batendo com o punho na mesa.

--- Não, senhor, por isso merecia levar uma na cara! --- guinchou
Ferfítchkin.

--- É preciso expulsá"-lo! --- resmungou Símonov.

--- Nem uma palavra, senhores, nem um movimento! --- gritou Zverkov
solenemente, contendo a indignação geral. --- Agradeço a todos vocês, mas
eu mesmo posso mostrar a ele o preço de suas palavras.

--- Senhor Ferfítchkin, amanhã mesmo você me dará uma satisfação pelas
palavras que disse hoje! --- disse eu em voz alta, dirigindo"-me com ar de
importância a Ferfítchkin.

--- O senhor quer dizer um duelo? De pleno acordo --- respondeu ele, mas
certamente eu parecia tão ridículo ao desafiá"-lo, e de tal maneira
aquilo não combinava com minha figura, que todos, e com eles também
Ferfítchkin, caíram na risada.

--- Sim, é claro, deixe"-o para lá! Já está completamente bêbado! --- disse
com repulsa Trudoliúbov.

--- Nunca vou me perdoar por tê"-lo incluído! --- resmungou novamente
Símonov.

“Agora seria a hora de dar com a garrafa em todos eles”, pensei; peguei
a garrafa e\ldots{} enchi meu copo.

“\ldots{}Não, é melhor ficar até o final!”, continuei pensando. “Ficariam
contentes, esses senhores, se eu saísse. De forma alguma. Vou ficar
aqui sentado, de propósito, e beberei até o fim, para mostrar que não
lhes dou a mínima importância. Ficarei aqui sentado, bebendo, porque
aqui é uma taverna e eu paguei muito dinheiro para entrar. Ficarei
sentado, bebendo, porque considero vocês nada mais que fantoches,
fantoches inexistentes. Ficarei sentado, bebendo\ldots{} e vou cantar, se
quiser, sim, senhor, vou cantar, porque tenho esse direito\ldots{} de
cantar\ldots{} Hum.”

Mas não cantei. Tentava apenas não olhar para nenhum deles; fazia as
poses mais independentes e com impaciência esperava o momento em que
eles mesmos falariam comigo \textit{primeiro}. Mas por azar eles não
falaram. \mbox{E como}, como eu desejei naquele momento fazer as pazes com
eles! Deram oito horas, depois nove. Eles passaram da mesa para os
sofás. Zverkov estendeu"-se num divã, colocando uma das pernas sobre uma
mesinha redonda. Levaram o vinho para lá. Ele realmente havia trazido
três garrafas por sua conta. A mim, é claro, não ofereceram. Todos se
sentaram ao seu redor no sofá. Eles o ouviam quase com devoção. Era
nítido que o amavam. “Por quê? Por quê?”, pensava comigo mesmo. Às
vezes eram acometidos por um êxtase ébrio, e então beijavam"-se. Falaram
do Cáucaso; do que é uma verdadeira paixão; de jogos de cartas; dos
postos mais vantajosos de serviço; de quanto lucro obtivera o hussardo
Podkharjevski, que nenhum deles conhecia pessoalmente, e se alegraram
com o fato de que ele tivera muito lucro; da extraordinária beleza e
graça da princesa D., que nenhum deles tampouco já tinha visto;
finalmente chegaram à imortalidade de Shakespeare.

Eu sorria com desdém e andava do outro lado do quarto, bem em frente ao sofá,
ao longo da parede, da mesa até a chaminé e de volta. Queria com todas as
forças mostrar que podia passar sem eles; e enquanto isso, batia de propósito
com os pés no chão, pisando com os saltos. Mas foi tudo em vão. \textit{Eles}
sequer prestaram atenção. Tive a paciência de andar assim, bem em frente deles,
das oito às onze horas, sempre no mesmíssimo lugar, da mesa até a chaminé e da
chaminé de volta para a mesa. “Vou ficar andando, ninguém pode me proibir.” Um
criado algumas vezes entrou no quarto e parou para olhar para mim; por conta
das incessantes voltas, minha cabeça girava; por vezes, tinha a impressão de
delirar. Ao longo dessas três horas, por três vezes fiquei coberto de suor e
depois sequei. Por vezes, era com uma dor profundíssima e virulenta que me
invadia a mente o seguinte pensamento: que se passariam dez anos, vinte anos,
quarenta anos, e mesmo assim, mesmo depois de quarenta anos, eu me lembraria
com repulsa e com humilhação daqueles momentos mais imundos, mais ridículos e
mais horríveis de toda a minha vida. Já era impossível rebaixar"-me de maneira
ainda mais descarada e voluntária, e eu sabia disso perfeitamente,
perfeitamente, e mesmo assim continuava a caminhar da mesa até a chaminé e de
volta.  “Oh, se vocês apenas soubessem de que sentimentos e pensamentos sou
capaz e como sou evoluído!”, pensava eu por vezes, dirigindo"-me em pensamento
para o sofá em que meus inimigos estavam sentados. Mas os meus inimigos se
comportavam como se eu nem estivesse no quarto. Uma vez, uma única vez, eles se
voltaram para mim, exatamente quando Zverkov começou a falar de Shakespeare e
eu de repente comecei a rir com desprezo. Minha risada foi tão artificial e
enojada que todos eles de uma vez pararam de falar e me observaram em silêncio
por uns dois minutos, sérios, sem rir, enquanto eu andava ao longo da parede,
da mesa até a chaminé, \textit{sem prestar neles nenhuma atenção}.  Mas não deu
em nada: eles não falaram comigo e depois de dois minutos novamente me
largaram. Soaram as onze.

--- Senhores --- gritou Zverkov, levantando"-se do sofá. --- Agora vamos todos
\textit{para lá}.

--- É claro, é claro! --- disseram os outros.

Voltei"-me bruscamente em direção a Zverkov. Eu estava a tal ponto
agoniado, a tal ponto abatido, que daria a vida para acabar com aquilo!
Eu tremia de febre; os cabelos molhados de suor haviam secado, aderindo
à testa e às têmporas.

--- Zverkov! Peço"-lhe desculpas --- disse eu de maneira abrupta e resoluta.
--- Ferfítchkin, a você também, a vocês todos, a vocês todos, eu ofendi a
todos!

--- A"-há! Não é chegado a duelos! --- resmungou com malícia Ferfítchkin.

Senti como que uma facada no coração.

--- Não, eu não temo o duelo, Ferfítchkin! Estou pronto a me bater com
você amanhã mesmo, ainda que depois da reconciliação. Até insisto
nisso, você não pode me recusar. Quero provar a vocês que eu não temo
um duelo. Você vai atirar primeiro, e eu atirarei para o ar.

--- Está distraindo a si mesmo --- notou Símonov.

--- Está delirando! --- declarou Trudoliúbov.

--- Mas permita"-nos passar, você está travando a passagem!\ldots{} Mas o que é
que você quer? ---respondeu Zverkov com desprezo. Estavam todos
vermelhos; os olhos de todos eles faiscavam: haviam bebido demais.

--- Eu peço sua amizade, Zverkov, eu o ofendi mas\ldots{}

--- Ofendeu? O s"-senhor! A mi"-im! Saiba, meu caro senhor, que você nunca,
em nenhuma circunstância, pode \textit{me} ofender!

--- Estamos fartos de você, fora! --- disse com firmeza Trudoliúbov. ---
Vamos.

--- Olímpia é minha, senhores, eu exijo! --- gritou Zverkov.

--- Não contestamos! Não contestamos! --- responderam, rindo.

Fiquei ali, ultrajado. A tropa saiu ruidosamente do cômodo, enquanto
Trudoliúbov puxava alguma canção estúpida. Símonov permaneceu por mais
um brevíssimo instante, para dar a gorjeta aos criados. De repente me
aproximei dele.

--- Símonov! Dê"-me seis rublos! --- disse eu em tom decidido e desesperado.

Ele olhou para mim de maneira extremamente surpresa mas com olhos
inexpressivos. Também estava bêbado.

--- Mas será possível que você também vai \textit{para lá} conosco?

--- Sim!

--- Não tenho dinheiro! --- cortou ele, sorrindo desdenhosamente e saindo do
cômodo.

Peguei"-o pelo casaco. Era um pesadelo.

--- Símonov! Eu vi que você tem dinheiro, por que você está negando? Sou
por acaso um canalha? Trate de não me recusar: se você soubesse, se
soubesse para que estou pedindo! Disso depende todo o meu futuro,
dependem todos os meus planos.

Símonov tirou o dinheiro e por pouco não o jogou em mim.

--- Pegue, se você é assim desavergonhado! --- disse ele impiedosamente,
correndo para alcançar os outros.

Fiquei por um minuto sozinho. A desordem, os restos de comida, a taça
quebrada no chão, o vinho derramado, as pontas de cigarro, a embriaguez
e o delírio na cabeça, uma melancolia torturante no peito e,
finalmente, o criado, que tinha visto e ouvido tudo e que me olhava nos
olhos com curiosidade.

--- \textit{Para lá}! --- gritei. --- Ou eles todos de
joelhos irão abraçar"-me as pernas e implorar pela minha amizade ou\ldots{}
ou eu darei um bofetão no Zverkov!


\section{parte V}

--- Aí está finalmente, aí está finalmente o choque com a realidade ---
balbuciei ao descer as escadas correndo a toda a pressa. --- Pelo visto já
não é mais o papa deixando Roma e partindo para o Brasil; pelo visto já
não é o baile no lago de Como!

“Você é um canalha”, passou"-me pela cabeça, “se está rindo disso agora!”

--- Que seja! --- gritei em resposta a mim mesmo. --- Agora tudo está perdido!

Eles já haviam sumido sem deixar rastro, mas não importava: eu sabia
aonde eles haviam ido.

No terraço de entrada, havia um solitário cocheiro do turno da noite,
vestindo um casaco de lã rústica, todo coberto pela neve molhada e um
tanto morna que ainda caía. O tempo estava úmido e abafado. Seu pequeno
cavalinho, felpudo e malhado, estava também todo coberto de neve e
tossia; lembro"-me muito bem disso. Lancei"-me ao trenó de tília; mas mal
fizera menção de erguer meu pé para me sentar quando a lembrança do
fato de que Símonov acabara de me emprestar seis rublos abateu"-me de
tal forma que caí no trenó como um saco.

--- Não! É preciso fazer muito para compensar tudo isso! --- gritei. --- Mas
eu hei de compensar ou nesta noite mesmo morrer de vez. Vamos!

Partimos. Um verdadeiro turbilhão girava em minha cabeça.

“Implorar pela minha amizade de joelhos eles não vão. É uma miragem, uma
miragem vulgar, repulsiva, romântica e fantástica; como o baile no lago
de Como. E por isso \textit{devo} dar um bofetão no Zverkov! Sou
obrigado a dar. Então está decidido: vou agora correndo dar esse
bofetão.”

--- Mais depressa!

O cocheiro puxou as rédeas.

“Assim que eu entrar, dou"-lhe. Será necessário dizer algumas palavras
antes do bofetão à guisa de prefácio? Não! Vou simplesmente entrar e dar.
Eles estarão todos sentados no salão, e ele no sofá com Olímpia.
Maldita Olímpia! Uma vez ela riu da minha cara e me recusou.

Vou arrastar Olímpia pelos cabelos e Zverkov pelas orelhas! Não, é
melhor se for por uma das orelhas, por uma orelha levá"-lo pelo quarto
todo. Talvez eles todos comecem a me bater e me expulsem aos empurrões.
Isso é quase certeza. Mas que seja! Ainda assim eu terei dado o
primeiro bofetão: a iniciativa terá sido minha; e pelos códigos de
honra isso é o que importa; ele já estará manchado e não poderá lavar
esse bofetão com surra nenhuma, apenas com um duelo. Vai ter que
duelar. Eles que me batam agora. Que seja, aqueles ignóbeis!
Principalmente o Trudoliúbov vai querer me bater: ele é muito forte;
Ferfítchkin virá pelos flancos e sem dúvida me agarrará pelos cabelos,
com certeza. Mas que seja, que seja! Foi para isso mesmo que vim. Suas
cabecinhas de vento serão forçadas a finalmente atinar para o trágico
que há em tudo isso! Quando eles me arrastarem em direção à porta, eu
gritarei para eles que na realidade eles valem menos que meu dedo
mindinho.”

--- Mais depressa, boleeiro, mais depressa! --- gritei para o cocheiro.

Ele até estremeceu ao brandir o chicote. Gritava já de maneira demasiado
selvagem.

“Vamos nos bater ao amanhecer, está decidido. Com o departamento está
tudo acabado. Ferfítchkin disse agora há pouco, em vez de
“departamento”, “lepartamento”. Mas onde conseguir as pistolas? Que
tolice! Pedirei um adiantamento e comprarei. E a pólvora, e a bala?
Isso é tarefa da testemunha. Terei tempo de fazer tudo isso antes do
amanhecer? E onde vou arranjar uma testemunha? Não tenho conhecidos\ldots{}”

--- Que tolice! --- gritei, sentido o turbilhão elevar"-se cada vez mais. ---
Tolice!

“O primeiro transeunte com quem eu falar na rua é obrigado a ser minha
testemunha, da mesma forma como se deve tirar da água alguém que se
afoga. Os fatos mais excêntricos devem ser admitidos. Mesmo se amanhã
eu pedisse para o próprio diretor ser minha testemunha, ele deveria
concordar apenas pelo espírito cavalheiresco e guardar segredo! Anton
Antônitch\ldots{}”

Acontece que naquele mesmo instante me era mais claro e nítido que a
qualquer pessoa no mundo todo o infame absurdo de minhas suposições,
todo o reverso da moeda, mas\ldots{}

--- Mais depressa, boleeiro, mais depressa, velhaco, mais depressa!

--- Ah, meu senhor! --- disse a força da terra.

O frio de repente me envolveu.

“Mas será que não é melhor\ldots{} Não é melhor\ldots{} Ir agora direto para casa?
Oh, meu Deus! Para quê, para que eu fui me convidar ontem para esse
almoço! Mas não, não podia! E a caminhada de três horas da mesa até a
chaminé? Não, eles, eles e ninguém mais é que devem me pagar por aquela
caminhada! Eles devem lavar essa desonra!”

--- Mais depressa!

``E se eles chamarem a polícia? Não ousarão! Temerão um escândalo. E se o
Zverkov se recusar a duelar, por desprezo? Isso é quase certeza; mas então eu
provarei a eles\ldots{} Eu vou então me lançar à estação de posta, quando ele
estiver partindo amanhã, agarrá"-lo pelas pernas e arrancar o casaco dele quando
estiver subindo no carro. Vou cravar os dentes em seu braço, vou mordê"-lo.
``Vejam a que ponto pode chegar um homem desesperado!'' Que ele me bata na
cabeça, e todos eles pelas costas.  Gritarei para todos os presentes: ``Vejam,
esse é o jovem fedelho que vai seduzir as circassianas com o meu escarro em seu
rosto!''

É claro que depois disso já estará tudo acabado! O departamento terá
desaparecido da face da terra. Vão me deter, me julgar, me expulsar do serviço,
me colocar na prisão, me mandar para a Sibéria, me deportar.  Pouco importa!
Depois de quinze anos irei atrás dele, me arrastando, em andrajos, um mendigo,
quando me soltarem da prisão. Eu o encontrarei em algum lugar numa cidade de
província. Estará casado e feliz. Terá uma filha adulta\ldots{} Direi: \textit{Olhe,
seu monstro, olhe para minhas bochechas cavadas e para meus andrajos! Perdi
tudo: a carreira, a felicidade, a arte, a ciência, \emph{a mulher amada},
tudo por sua culpa. Aqui estão as pistolas. Vim descarregar minha pistola
e\ldots{} e o perdoo}. Então eu atirarei para o ar e de mim não se terá mais
sinal\ldots{}

Quase comecei a chorar, embora soubesse perfeitamente bem naquele mesmo
instante que tirara tudo aquilo do Silvio e do \textit{Baile de máscaras} de
Liêrmontov.\footnote{ Silvio é o personagem principal da novela \textit{O tiro} de
Aleksandr Púchkin (1799--1837). \textit{Baile de máscaras} é um drama em versos
escrito pelo poeta Mikhail Liêrmontov (1814--1841).} E subitamente senti uma
imensa vergonha, tamanha vergonha que parei o cavalo, saí do trenó e fiquei
parado sobre a neve no meio da rua. O cocheiro olhava para mim surpreso,
suspirando.

O que fazer? Não podia ir para lá: seria uma tolice; desistir de tudo não era
possível, porque já ia sair\ldots{} Senhor! Como poderia desistir! E depois de
tais ofensas!

--- Não! --- gritei, lançando"-me novamente ao trenó --- Está predestinado, é o
destino! Depressa, depressa, vamos para lá!

E com impaciência socava o cocheiro no pescoço.

--- Mas o que é isso, por que me bate? --- gritou o mujiquezinho, fustigando,
porém, o cavalo de tal maneira que ele começou a dar coices com as patas
traseiras.

Uma neve molhada caía em flocos; me descobri, não estava preocupado com ela.
Esquecera todo o resto, porque decidira definitivamente dar o bofetão e com
horror sentia que aquilo estava \textit{prestes a acontecer}, seria agora mesmo
e ``já não poderia ser evitado por força alguma''. Os solitários postes de
luz faiscavam lugubremente em meio às brumas e à neve como tochas em um
funeral. A neve se acumulava sob meu casaco, na sobrecasaca, sob a gravata e lá
derretia; eu não me protegia: já estava tudo perdido mesmo! Finalmente
chegamos. Saltei, quase fora de mim, subi correndo os degraus e comecei a bater
na porta com as mãos e com os pés. Sentia"-me terrivelmente enfraquecido
especialmente nas pernas, nos joelhos.  Abriram muito depressa; pareciam saber
que eu chegaria. (Realmente, Símonov advertira que talvez chegasse mais um, já
que ali era necessário advertir e geralmente tomar todas as precauções. Era uma
das \textit{lojas da moda} de então, dessas que agora há tempos foram extintas pela
polícia. Durante o dia era de fato uma loja; mas à noite às pessoas que
tivessem uma recomendação era permitido visitar.) Atravessei a passos rápidos a
loja escura, entrei no salão que já me era conhecido e onde queimava apenas uma
única vela e parei, perplexo: não havia ninguém.

--- Onde estão eles? --- perguntei para alguém.

Mas eles, é claro, já haviam tido tempo de se dispersar\ldots{}

Diante de mim, havia apenas uma pessoa de sorriso estúpido, a própria
dona do local, que eu conhecia por alto. Um minuto depois, abriu"-se a
porta e entrou outra pessoa.

Sem prestar atenção em nada, eu caminhava pelo cômodo, aparentemente
falando sozinho. Era como se tivesse sido salvo da morte e com alegria
sentisse isso com todo o meu ser: teria dado o bofetão, teria dado
seguramente, seguramente! Mas agora eles não estavam lá e\ldots{} Tudo
sumira, tudo mudara\ldots{}! Olhei ao redor. Ainda não voltara à razão.
Olhei maquinalmente para a moça que entrara: diante de mim divisava"-se
um rosto fresco, jovem, um pouco pálido, com sobrancelhas retas e
escuras, com um olhar sério e de certa forma um tanto surpreso. Gostei
imediatamente dele; teria passado a odiá"-la se ela tivesse sorrido.
Pus"-me a fitá"-la atentamente, como que com esforço: os pensamentos
ainda não haviam se reordenado. Havia algo de ingênuo e bondoso naquele
rosto, mas ele era sério de uma maneira que beirava a estranheza. Tenho
certeza de que ali aquilo era uma desvantagem para ela, e de que nenhum
daqueles idiotas ainda a notara. Porém, não se poderia dizer que era
bonita, embora fosse de alta estatura, forte, de bela compleição.
Estava vestida de maneira extremamente simples. Algo vil me perturbava;
caminhei reto em direção a ela\ldots{}

Olhei por acaso para o espelho. Meu rosto perturbado pareceu"-me
repugnante ao extremo: pálido, mau, infame, os cabelos desgrenhados.
``Tudo bem, fico contente'', pensei. ``Fico contente justamente por
parecer a ela repugnante; gosto disso\ldots{}''


\section{parte VI}

Em algum lugar atrás de um tapume, como que movido por uma forte
pressão, como que sufocado por alguém, rouquejou um relógio. Após um
ronco exageradamente longo, seguiu"-se um tinido fininho, repelente e
como que inesperadamente acelerado, como se alguém tivesse subitamente
saltado para a frente. Bateram as duas. Despertei, embora não estivesse
dormindo, mas apenas deitado num estado modorrento.

O cômodo estreito, apertado e baixo, atravancado por um imenso armário
de roupas e atulhado de caixas de papelão, trapos e todo tipo de roupas
inúteis, estava quase completamente escuro. O toco de vela que queimava
sobre a mesa num canto do quarto se extinguira de todo, inflamando"-se
um pouquinho de quando em quando. Dentro de alguns minutos a escuridão
seria completa.

Não demorei para voltar a mim; de uma vez, sem esforço, lembrei"-me de
imediato de tudo, como se me vigiasse para novamente atacar. Mesmo
durante o torpor, em minha lembrança ainda assim persistia algo como um
ponto, que de maneira alguma se perdia de vista e ao redor do qual
vagavam pesadamente meus devaneios de sono. Mas era estranho: tudo que
acontecera comigo naquele dia parecia"-me agora, ao despertar, ter
acontecido havia muito, muito tempo, como se eu tivesse passado por
tudo aquilo havia muito, muito tempo.

Minha cabeça parecia inebriada. Algo como que pairava sobre mim e me
tocava, me incitava e me perturbava. A tristeza e o mau humor novamente
acumulavam"-se e buscavam uma saída. De repente, vi ao meu lado dois
olhos abertos, que me examinavam curiosa e obstinadamente. Seu olhar
era frio e indiferente, sombrio, como que completamente estranho;
produzia uma sensação pesada.

Um sombrio pensamento nasceu em meu cérebro e percorreu todo o meu corpo
com uma desagradável sensação, semelhante àquela que se sente quando se
entra no subsolo, úmido e bolorento. Era de certa forma pouco natural
que apenas justamente agora esses dois olhos decidissem começar a me
observar. Lembrei"-me também de que ao longo de duas horas eu não
dissera uma única palavra àquela criatura e não considerara aquilo
necessário em absoluto; por algum motivo eu gostara daquilo havia
pouco. Naquele mesmo momento, surgiu nítida e repentina, absurda e
asquerosa como uma aranha, a ideia da perversão que, sem amor,
grosseira e desavergonhadamente, começa diretamente a partir do ponto
em que o verdadeiro amor é coroado. Olhamos um para o outro daquela
forma por muito tempo, mas ela não baixava seus olhos diante dos meus e
não mudava sua expressão, de maneira que acabei por algum motivo
sentindo pavor.

--- Qual é seu nome? --- perguntei de maneira entrecortada, para terminar
mais rapidamente.

--- Liza --- respondeu ela quase num sussurro, mas de uma forma totalmente
fria e desviando o olhar.

Fiquei em silêncio.

--- Hoje o tempo\ldots{} essa neve\ldots{} está horrível! --- disse eu quase comigo
mesmo, colocando tristemente os braços atrás da cabeça e olhando para o
teto. Ela não respondeu. Era tudo revoltante.

--- Você é daqui? --- perguntei depois de um minuto, quase com raiva,
virando levemente a cabeça na direção dela.

--- Não.

--- De onde?

--- De Riga --- disse ela a contragosto.

--- Alemã?

--- Russa.

--- Está aqui faz tempo?

--- Onde?

--- Na casa.

--- Duas semanas --- ela falava de maneira cada vez mais entrecortada. A
velinha apagou"-se completamente; eu já não podia distinguir seu rosto.

--- Tem pai e mãe?

--- Sim\ldots{} Não\ldots{} Tenho.

--- Onde eles estão?

--- Lá\ldots{} em Riga.

--- E o que eles são?

--- Ah\ldots{}

--- Como assim? São de que classe?

--- Pequeno"-burgueses.

--- Você ainda morava com eles?

--- Sim.

--- Quantos anos você tem?

--- Vinte.

--- E por que é que você partiu da casa deles?

--- Ah.

Aquilo significava: deixe"-me em paz, estou enojada. Ficamos em silêncio.

Sabe Deus por que eu não fui embora. Eu mesmo ficava cada vez mais
enojado e melancólico. As imagens de todo aquele dia de certa forma,
como que por conta própria, contra a minha vontade, começaram a passar
desordenadamente por minha cabeça. Lembrei"-me de repente de uma cena
que havia visto de manhã na rua, quando trotava preocupado para o
serviço.

--- Hoje vi levarem um caixão e quase derrubarem --- subitamente disse eu em
voz alta, sem querer em absoluto começar a conversa, mas simplesmente
por acaso.

--- Um caixão?

--- Sim, na Sennaia;\footnote{ Antiga praça na região central de São
Petersburgo.} tiravam de um porão.

--- De um porão?

--- Não de um porão, de um andar subterrâneo\ldots{} Sabe, de baixo\ldots{} De uma
casa de má fama\ldots{} Havia tamanha sujeira ao redor\ldots{} Cascas, lixo\ldots{} Um
cheiro forte\ldots{} Abominável.

Silêncio.

--- Ruim enterrar alguém hoje! --- comecei novamente, apenas para não ficar
em silêncio.

--- Por que ruim?

--- A neve, a umidade\ldots{} (Bocejei.)

--- Dá no mesmo --- disse ela de repente após um breve silêncio.

--- Não, é horrível\ldots{} (Bocejei de novo.) Os coveiros certamente ralhavam
porque a neve os molhava. E no túmulo certamente havia água.

--- E por que havia água no túmulo? --- perguntou ela com certa curiosidade,
mas falou de maneira ainda mais ríspida e entrecortada do que antes.
Algo de repente começou a me incitar.

--- Ora, água, no fundo, mais de um palmo. Lá no Vôlkovo\footnote{
Cemitério localizado na porção sul da cidade.} não se consegue cavar um
túmulo seco.

--- Por quê?

--- Como \textit{por quê}? O lugar é só água. Aqui todo lugar é um pântano. Por
isso colocam na água mesmo. E eu já vi\ldots{} Muitas vezes\ldots{}

(Não havia visto uma vez sequer, nunca nem mesmo estivera no Vôlkovo,
apenas ouvira as pessoas falarem.)

--- Será que para você dá no mesmo morrer?

--- E do que é que eu vou morrer? --- respondeu ela, como que se defendendo.

--- Em algum momento vai morrer, e vai morrer exatamente como essa
falecida de hoje. Também era\ldots{} uma moça\ldots{} Morreu de tísica.

--- No hospital uma mulher da vida morreria também\ldots{} (``Ela já aprendeu'',
pensei, ``e disse: \textit{mulher da vida}, e não \textit{mulher}''.)

--- Ela devia para a patroa --- objetei, incitando cada vez mais a discussão
---, e quase até o fim serviu a ela, embora fosse tísica. Uns cocheiros
ao redor conversavam com uns soldados e contavam isso. Certamente seus
antigos conhecidos. Riam. E ainda pretendiam beber à memória dela no
botequim. (Essa parte eu também aumentara muito.)

Silêncio, um profundo silêncio. Ela sequer se mexia.

--- E por acaso é melhor morrer no hospital?

--- E não dá no mesmo\ldots{}? E por que é que eu vou morrer? --- acrescentou ela
irritada.

--- Não agora, mas talvez depois.

--- Mesmo depois\ldots{}

--- De jeito nenhum! Você agora é jovem, bonita, cheia de vida: dão muito
valor a você por isso. Mas daqui a um ano nessa vida você já não será a
mesma, vai definhar.

--- Daqui a um ano?

--- Em todo caso, daqui a um ano seu valor será menor --- continuei num tom
maldoso. --- Você vai passar daqui para algum lugar mais baixo, para
outra casa. Um ano depois disso para uma terceira casa, cada vez mais e
mais baixa, e depois de uns sete anos você acabará chegando no porão da
Sennaia. E seria até bom. Pior seria se você além disso pegasse alguma
doença, não sei, uma fraqueza de peito\ldots{} Ou um resfriado ou algo
assim. Nessa vida uma doença custa a passar. Ela pega mesmo, e às vezes
não solta mais. E então você morre.

--- E então eu morro --- respondeu ela já completamente raivosa e
remexendo"-se num movimento rápido.

--- Mas dá pena.

--- De quem?

--- Dá pena da vida.

Silêncio.

--- Você teve um noivo? Hein?

--- Por que a pergunta?

--- Não estou interrogando você. Eu não tenho nada com isso. Por que você
está irritada? É claro que você pode ter passado pelos seus
contratempos. E eu com isso? Mas tenho pena.

--- De quem?

--- De você.

--- Não tem por quê\ldots{} --- disse ela num sussurro quase imperceptível e
novamente remexendo"-se.

Aquilo me irritou imediatamente. Como! Fora tão dócil com ela, e ela\ldots{}

--- E você, o que acha? Que está num bom caminho?

--- Não acho nada.

--- Pois é ruim que não pense. Caia em si, enquanto ainda há tempo. E há
tempo. Você ainda é jovem, bonita; poderia amar, casar"-se, ser feliz\ldots{}

--- Nem todas as casadas são felizes --- cortou ela no tom grosseiro e
atropelado de antes.

--- Nem todas, é claro. Mas mesmo assim é muito melhor do que estar aqui.
Mil vezes melhor. Pode"-se seguir vivendo com amor mas sem felicidade.
Mesmo na tristeza a vida é boa, é bom viver no mundo, não importa como
se viva. E o que há aqui além\ldots{} do fedor? Arre!

Virei"-me com repulsa; eu já não filosofava friamente. Eu mesmo começara
a sentir aquilo que dizia e me inflamava. Eu já ansiava por expor as
\textit{ideiazinhas} íntimas que eu relegara a um canto. Algo
subitamente se acendera em mim, algum objetivo \textit{surgira}.

--- Não olhe para mim, que estou aqui e não sou exemplo para você. Eu
talvez seja até pior do que você. Cheguei aqui bêbado, por sinal ---
disse, porém, apressando"-me a justificar"-me. --- Além disso, um homem
não é, em absoluto, um exemplo para uma mulher. São coisas distintas;
embora eu me suje, me manche, não sou em compensação escravo de
ninguém; fui e voltei, já não estou mais lá. Sacudo a poeira e já não
sou mais aquele. Mas você, convenhamos, desde o início é uma escrava.
Sim, uma escrava! Você abriu mão de tudo, toda a sua liberdade. E se
depois você quiser romper essas correntes, já não poderá: vão amarrar
você com mais e mais força. É uma corrente maldita. Eu a conheço. Não
falo mais de qualquer outra coisa, você talvez nem entenda, mas me
diga: certamente você já está devendo para a patroa, não está? Pois
está vendo! --- acrescentei, embora ela não me respondesse, mas apenas
ouvisse em silêncio com todo o seu ser. --- São essas as suas correntes!
Já não vai conseguir nunca pagar seu resgate. É assim que vão fazer. Dá
no mesmo que vender a alma ao diabo\ldots{}

E além disso eu\ldots{} talvez seja tão infeliz quanto --- como você poderia
saber? --- e me arraste na lama de propósito, também de tristeza. Não
bebem por desespero? Eu mesmo estou aqui agora por desespero. Mas então
me diga, o que há de bom aqui: olhe para nós dois\ldots{} Nos encontramos\ldots{}
agora há pouco, e o tempo todo não trocamos uma palavra sequer um com o
outro, e você, como uma selvagem, só depois começou a me observar; e eu
a observar você também. Por acaso é assim que se ama? Por acaso é assim
que uma pessoa deve se encontrar com outra? É uma vergonha, isso sim!

--- Sim! --- disse ela ríspida e apressadamente, fazendo coro ao que eu
dissera. Fiquei até mesmo surpreso com a precipitação daquele \textit{sim}.
Seria talvez possível que o mesmo pensamento lhe passara pela cabeça
quando ela agora há pouco me observava? Seria possível que ela também
era capaz de certos pensamentos?\ldots{} ``Com mil diabos, isso é curioso, é
uma \textit{afinidade}'', pensei eu, quase esfregando as mãos. ``E como
não querer dominar uma alma assim tão jovem?\ldots{}''

Era o jogo que mais me entusiasmava.

Ela virou a cabeça, trazendo"-a mais perto de mim e, ao que me pareceu na
escuridão, apoiou"-se na mão. Talvez me observasse. Como eu lamentava
não poder ver seus olhos. Eu ouvia sua respiração profunda.

--- Por que você veio para cá? --- comecei já com certo tom de autoridade.

--- Ah\ldots{}

--- Mas como é bom viver na casa dos pais! É quente, confortável; é o seu
ninho.

--- E se for pior?

``É preciso acertar o tom'', passou"-me pela cabeça. ``Talvez não aceite
muito sentimentalismo.''

Porém, foi apenas um pensamento passageiro. Juro, ela de fato me
interessava. Além disso, eu estava de certa forma debilitado e
indisposto. Mas a infâmia tão facilmente caminha lado a lado com o
sentimento.

--- Quem pode dizer?! --- apressei{}"-me em responder. --- Acontece de tudo.
Mas eu tenho certeza de que alguém a ofendeu, e é mais provável que
\textit{eles} sejam culpados e não você. Não sei nada de sua história,
mas uma moça como você certamente não vem parar aqui por vontade
própria\ldots{}

--- Como assim uma moça como eu? --- sussurrou ela de maneira quase
imperceptível; mas eu ouvi.

``Com mil diabos, eu a estou adulando. Era horrível. Ou talvez fosse
bom\ldots{}'' Ela seguia em silêncio.

--- Veja, Liza, vou falar de mim mesmo! Se eu tivesse uma família desde a
infância, não seria como sou agora. Penso nisso com frequência. Então
por pior que seja a família, ainda assim são pai e mãe, e não inimigos,
não estranhos. Ao menos uma vez por ano demonstrarão amor por você.
Pelo menos você sabe que está em casa. Já eu cresci sem família; por
conta disso, certamente, é que me tornei assim\ldots{} insensível.

Esperei novamente.

``Talvez não entenda'', pensei. ``E é ridículo, mesmo: a moral.''

--- Se eu fosse pai e tivesse uma filha, creio que amaria mais a filha que
os filhos, mesmo --- comecei indiretamente, como se falasse de outra
coisa, para distraí"-la. Reconheço que corei.

--- E por quê? --- perguntou ela.

Ah, então escutava!

--- Ah, não sei, Liza. Veja: conheci um pai que era uma pessoa severa,
rígida, mas que diante da filha ficava de joelhos, beijava"-lhe as mãos
e os pés, não se cansava mesmo de admirá"-la. Ela dançava nas festas e
ele ficava cinco horas parado no mesmo lugar, não tirava os olhos dela.
Era louco por ela; eu entendo isso. Ela à noite caía no sono do
cansaço, e ele acordava e ia beijá"-la e benzê"-la enquanto dormia. Ele
vivia com uma sobrecasaca ensebada, era avarento com todos, mas por ela
gastaria seu último rublo, comprava ricos presentes, e era uma grande
alegria para ele quando ela gostava do presente. Um pai sempre ama mais
as filhas que a mãe. Como algumas garotas são felizes quando moram em
casa! Creio que eu não deixaria minha filha se casar.

--- Como assim? --- perguntou ela, dando um pequeno sorriso.

--- Teria ciúmes, juro por Deus. Como assim, beijar outro homem? Amar um
estranho mais que o pai? É difícil até imaginar. É claro que isso é
tudo uma tolice; é claro que no final todo mundo cria juízo. Mas eu,
pelo visto, antes de conceder a mão, iria me atormentar com uma única
preocupação: iria rejeitar todos os pretendentes. E no fim eu acabaria
casando"-a com aquele que ela mesma amasse. E o que a filha ama é sempre
aquele que o pai acha o pior. É assim. Isso causa muito mal às
famílias.

--- Já outros ficam felizes em vender a filha em vez de conceder a mão
honradamente --- disse ela de repente.

Ah! Então era isso!

--- Isso, Liza, é em famílias amaldiçoadas, em que não há nem Deus, nem
amor --- secundei acaloradamente. --- E onde não há amor, tampouco há bom
senso. Há famílias assim, é verdade, mas não é delas que estou falando.
Pelo visto você, na sua família, não viu nenhum bem, para falar
assim. Você é genuinamente uma pessoa infeliz. Hum\ldots{} É por conta da
pobreza que isso na maioria das vezes acontece.

--- E entre os nobres por acaso é melhor? Mesmo na pobreza as pessoas
honradas vivem bem.

--- Hum\ldots{} sim. Talvez. Novamente, Liza: as pessoas gostam de levar em
conta apenas sua tristeza, a sua alegria não levam em conta. Mas se
contasse como se deve, veria que tem uma boa porção de felicidade
reservada. E se nessa família tudo vai bem, Deus abençoará, o marido
será bom, amará você, mimará você, não se afastará de você! Nessa
família é bom viver! Algumas vezes é bom até em meio ao sofrimento;
pois onde é que não há tristeza? Se você se casar,
\textit{você mesma saberá}. Por
outro lado, se tomarmos apenas os primeiros tempos da vida de casada
com aquele que se ama: que felicidade, quanta felicidade se tem muitas
vezes! Para dar e vender. Nos primeiros tempos até as brigas com o
marido acabam bem. Algumas, quanto mais amam, mais brigas arranjam com
o marido. É verdade; eu conheci uma assim: ``É que eu o amo muito'',
dizia, ``e por conta desse amor o torturo e assim o faço sentir''. Você
sabia que pelo amor pode"-se torturar uma pessoa propositalmente? As
mulheres ainda mais. E ela pensa consigo mesma: ``Depois vou amá"-lo
tanto, vou acariciá"-lo tanto que não faz mal agora torturar um
pouquinho''. E em casa todos se alegram ao vê"-la, o ambiente é bom, é
alegre, é tranquilo, é honrado\ldots{} Mas também há algumas que são
ciumentas. Se ele saía para algum lugar --- conheci uma que era assim ---,
ela não se aguentava, saltava da cama no meio da noite e corria para
ver, às escondidas, se ele não estaria lá, não estaria naquela casa,
não estaria com aquela outra. Isso já não é bom. E ela mesma sabe que
não é bom, e seu coração parece parar, ela se martiriza, mas continua
amando; tudo pelo amor. E como é bom fazer as pazes depois de uma
briga, declarar"-se culpada perante ele ou perdoá"-lo! E como é bom para
ambos, como fica bom de repente: é como se tivessem se conhecido
novamente, se casado novamente e o amor tivesse começado novamente. E
ninguém, ninguém deve saber o que se passa entre marido e mulher se
eles se amam. E não importa o motivo de suas brigas, sequer suas mães
deveriam chamar para julgá"-los e acusar um ao outro. Devem julgar a si
mesmos. O amor é um mistério divino e deve ficar escondido dos olhos de
todos os estranhos, haja o que houver. Quanto mais sagrado isso o
fizer, melhor. Acabam respeitando mais um ao outro, e muito se baseia
no respeito. E se uma vez houve amor, se se casaram por amor, por que o
amor deveria passar?! Será que é impossível mantê"-lo? São raros os
casos em que não se pode mantê"-lo. E também, quando se consegue como
marido um homem bom e honesto, como é que o amor poderia passar? O
primeiro amor, o do casamento, passará, mas então virá um amor ainda
maior. Suas almas se unirão, todas as coisas serão feitas
conjuntamente; entre eles não haverá mais segredos. E quando chegarem
os filhos, todos os momentos, mesmo os mais difíceis, parecerão
felizes; basta amarem e serem corajosos. Mesmo o trabalho será uma
alegria, mesmo se às vezes for necessário abrir mão do pão para dar aos
filhos, até isso será uma alegria. Eles afinal mais tarde amarão você
por conta disso; ou seja, você economiza para si mesmo. Os filhos vão
crescer e você vai sentir que é um exemplo para eles, um apoio; mesmo
quando você morrer, eles irão carregar em si por toda a vida os seus
sentimentos e os seus pensamentos, tais como os receberam de você,
assumirão sua imagem e semelhança. Trata"-se portanto de um grande
dever. Como isso não aproximará ainda mais o pai e a mãe? Dizem que ter
filhos é difícil. Mas quem diz isso? É uma alegria celestial! Você
gosta de crianças pequenas, Liza? Eu gosto muitíssimo. Sabe: um
menininho assim todo rosadinho, mamando de seu peito; mas que marido
não terá seu coração voltado para a esposa ao olhar para ela enquanto
segura a criança! Uma criancinha rosadinha, gorduchinha, se esticando,
se espreguiçando; as mãozinhas e os pezinhos roliços, as unhinhas
limpinhas, pequeninas, tão pequeninas, que é até engraçado olhar, os
olhinhos como se entendessem tudo. E enquanto ele mama, aperta seu
peito com a mãozinha, brinca com ele. O pai chega perto e ele se solta
do peito, curva"-se inteirinho para trás, olha para o pai e ri, como se
fosse muitíssimo engraçado, e novamente, novamente se põe a mamar. Ou
então pega e morde o peito da mãe, se os dentinhos já começaram a
nascer, e olha para ela com o canto de seus olhinhos como se dissesse:
\textit{Veja, mordi você!} Será que não é tudo alegria quando os três estão
juntos, o marido, a mulher e a criança? Por esses momentos se pode
perdoar muito. Não, Liza, é preciso primeiro aprender por si mesmo a
viver, e só depois culpar os outros!

``É com quadros, com esses quadros que é preciso pegá"-la!'', pensei comigo
mesmo, embora, juro por Deus, falasse com sentimento, e de repente
corei. ``E se de repente ela começar a gargalhar, onde é que eu vou me
enfiar?'' Essa ideia me enfureceu. No fim do discurso eu realmente me
exaltara, e agora meu amor"-próprio como que sofria. O silêncio
prolongou"-se. Eu quis até mesmo empurrá"-la.

--- Mas por que\ldots{} --- começou ela de repente e parou.

Mas eu já compreendera tudo: em sua voz já vibrava alguma outra coisa,
não ríspida, não grosseira e obstinada, como havia pouco, mas algo
suave e envergonhado, envergonhado a tal ponto que eu mesmo de repente
comecei a ficar com vergonha diante dela e a me sentir culpado.

--- O quê? --- perguntei com uma curiosidade meiga.

--- É que você\ldots{}

--- O quê?

--- É que você\ldots{} fala como se fosse num livro --- disse ela, e de repente
algo como que zombeteiro novamente se ouviu em sua voz.

Doeu"-me agudamente aquela observação. Não era o que eu esperava.
 

Eu não compreendia que ela propositalmente usava a zombaria como
máscara, que aquele era costumeiramente o último subterfúgio das
pessoas acanhadas e puras de coração quando sentem suas almas invadidas
de maneira grosseira e impertinente, e que elas até o último minuto não
se entregam por orgulho e temem expressar seus sentimentos diante de
outros. Já pela timidez com a qual ela por várias vezes pusera"-se a
fazer suas zombarias, decidindo"-se a dizê"-las apenas com muito esforço,
eu deveria ter percebido. Mas eu não percebi, e um sentimento ruim se
apoderou de mim.

\textit{Espere um pouco}, pensei.


\section{parte VII}

--- Mas basta, Liza, como pode ser como se fosse num livro se quando eu
mesmo, vendo de fora, me enojo? Embora não seja de fora. Tudo isso
agora despertou"-me no âmago\ldots{} Mas será possível, será possível que
você mesma não se enoje aqui? Não, nota"-se que o hábito é capaz de
tudo! Deus sabe o que o hábito pode fazer a um homem. Será possível que
você pense realmente que não vai envelhecer nunca, que será bonita para
sempre e que vão mantê"-la aqui até o fim dos tempos? E eu nem estou
falando da obscenidade daqui\ldots{} Mas por outro lado, deixe"-me dizer uma
coisa sobre isso, sobre a vida que você agora leva: agora você pode ser
jovem, encantadora, bonita, com alma e sentimento; mas talvez você não
saiba que eu, assim que caí em mim, agora há pouco, fiquei
imediatamente enojado de estar aqui com você! Apenas num estado de
embriaguez pode"-se vir parar aqui. Mas se você estivesse em outro
lugar, se vivesse como as pessoas boas vivem, talvez eu não apenas a
cortejaria, eu simplesmente me apaixonaria por você, ficaria contente
com um olhar seu, imagine com uma palavra; espreitaria os portões de
sua casa, ficaria de joelhos diante de você; olharia para você como se
fosse minha noiva e ainda consideraria isso uma honra para mim. Não
ousaria pensar algo impuro a seu respeito. Mas aqui, sei que basta um
assovio meu para que você, quer queira quer não, venha atrás de mim, e
já não serei eu a me submeter à sua vontade, mas sim você à minha. O
último dos mujiques vende seu trabalho, mas mesmo assim não se torna um
completo escravo, já que sabe que há um prazo para aquilo acabar. Mas
onde está o seu prazo? Pense bem: o que é que você está entregando
aqui? O que está vendendo? A alma, a alma, que não é sua posse, é isso
que você vende junto com o corpo! Você entrega o seu amor para ser
profanado por um bêbado qualquer! O amor! Mas ele é tudo, ele é um
diamante, o tesouro de uma donzela é o amor! Alguns estão dispostos a
dar a alma, a enfrentar a morte para merecer esse amor. Mas quanto vale
o seu amor agora? Você se vendeu, se vendeu inteira, e agora para que
tentar obter esse amor quando mesmo sem amor é tudo possível. Não há
ofensa maior para uma moça, você entende? Eu ouvi que consolam vocês,
tolas, permitindo que tenham amantes aqui. Mas isso não passa de
travessura, de engano, um motivo para rirem de vocês, mas vocês
acreditam. E de verdade, ele por acaso ama você, esse amante? Não
creio. Como ele vai amar se sabe que a qualquer momento vão chamá"-la e
você vai deixá"-lo. Que porco não seria depois disso! Terá ele uma gota
de respeito por você? O que você terá em comum com ele? Ele vai rir de
você e roubar você, esse é que é o amor dele! Se ele não bater em você
já será bom. Mas talvez bata. Pergunte para ele, se você tiver um, se
ele vai se casar com você. Ele vai rir da sua cara, isso se não cuspir
ou bater em você; sendo que ele mesmo não deve valer nem um tostão
furado. E para que afinal você foi desperdiçar sua vida aqui? Porque
dão café para beber e comida à vontade? Para que é que dão comida? Uma
outra mulher, honesta, não conseguiria nem engolir isso, porque sabe
para que é que dão comida. Você tem uma dívida aqui, e vai continuar
devendo, até o fim de tudo vai continuar devendo, até os hóspedes
começarem a ter nojo de você. E isso será em breve, não conte com sua
juventude. Porque aqui o tempo tem asas. Vão expulsá"-la. E não vão
simplesmente expulsá"-la; primeiramente, por um bom tempo vão começar a
criticá"-la, vão começar a repreendê"-la, vão começar a xingá"-la, como se
você não tivesse dado sua saúde por ela, não tivesse arruinado a troco
de nada sua juventude e sua alma em benefício dela, mas como se fosse
você quem a tivesse prejudicado, reduzido à miséria, roubado. E não
espere apoio: as suas outras amigas também vão atacá"-la, tudo para
bajular a patroa, porque aqui todas estão na escravidão, há muito tempo
perderam a consciência e a compaixão. Tornaram"-se infames, e não existe
no mundo nada mais abjeto, mais infame e ofensivo do que esses
insultos. E você deposita tudo aqui, tudo, sem ressalvas: a saúde, a
juventude, a beleza, as esperanças, e aos vinte e dois anos você vai
aparentar trinta e cinco, e já será bom se você não ficar doente, peça
a Deus por isso. Porque você agora decerto pensa que nem trabalha, que
é tudo um passatempo! Mas não há e nunca houve no mundo trabalho mais
pesado e desumano. É como se o coração se debulhasse em lágrimas. Você
não ousará dizer uma palavra, nem meia palavra quando expulsarem você
daqui; sairá como culpada. Você se mudará para outro lugar, depois para
um terceiro, depois para algum outro ainda e finalmente você chegará à
Sennaia. E lá vai começar a apanhar sem mais nem menos; lá, essa é a
amabilidade; os hóspedes não sabem fazer carinho sem bater por lá. Você
não acredita que é assim horrendo? Vá até lá algum dia e olhe, talvez
você veja com seus próprios olhos. Uma vez, no Ano"-Novo, eu vi uma
dessas lá, na porta. Ela tinha sido jogada para fora de brincadeira
pelos amigos dela, para congelar um pouco, tudo porque berrara demais,
e fecharam a porta atrás dela. Eram nove horas da manhã e ela já estava
completamente bêbada, descabelada, seminua, toda machucada. Estava toda
pintada, mas com os olhos roxos; saía sangue do nariz e da boca: um
cocheiro qualquer acabara de dar"-lhe uma. Estava sentada numa escadinha
de pedra, segurando algum tipo de peixe salgado; berrava, lamentando"-se
de sua \textit{sorte} e batendo com o peixe nos degraus da escada. Enquanto
isso, no terraço de entrada, amontoavam"-se uns cocheiros e soldados
bêbados, provocando"-a. Você não acredita que você também ficará assim?
Nem eu quero acreditar nisso, mas como se pode saber? Talvez há uns oito
ou dez anos essa mesma mulher do peixe salgado tenha chegado aqui,
vinda de algum lugar, fresca como um jovem querubim, inocente, pura;
não conhecia o mal, corava a cada palavra. Talvez tenha sido, assim
como você, orgulhosa e melindrosa, diferente das outras, com o olhar de
uma rainha e sabendo que toda a felicidade estava à espera de quem se
apaixonasse por ela e por quem ela se apaixonasse. E você vê como ela
acabou? E se nesse mesmo momento em que ela batia com o peixe naqueles
degraus sujos, bêbada e descabelada, e se nesse momento ela se
lembrasse de todos os seus puros anos passados na casa dos pais, quando
ainda ia à escola e o filho do vizinho vinha espreitá"-la no caminho,
prometia amá"-la por toda a vida, dedicar a ela todo o seu destino, e
quando juntos prometeram amar um ao outro para sempre e casar"-se assim
que fossem grandes! Não, Liza, a felicidade, a felicidade para você
será, em algum lugar, num canto, num porão como aquela de hoje, tísica,
morrer o quanto antes. No hospital, você diz? Muito bem, vão levar; mas
e se a patroa ainda precisar de você? Tísica é uma doença e tanto; não
é uma febre qualquer. Com ela até o último minuto a pessoa tem
esperança e diz que está saudável. Pois se ilude. E quem se dá bem é a
patroa. Não duvide, assim é; quer dizer, você vendeu sua alma e ainda
por cima deve dinheiro, portanto não ouse dar um pio. Mas vai morrer,
todos irão abandoná"-la, todos se afastarão; afinal, o que se poderá
tirar de você então? Ainda vão repreendê"-la, já que ocupa um lugar à
toa e não morre logo. Não vai conseguir beber um copo de água sem ser
xingada: ``Quando'', dirão, ``é que você vai morrer, seu traste? Atrapalha
o sono dos outros; fica gemendo, os hóspedes estão com nojo''. É
verdade; eu mesmo já ouvi essas palavras. Vão enfiá"-la, moribunda, no
canto mais fétido do porão; escuridão, umidade; e o que é que você irá
pensar então, deitada sozinha? E quando você morrer, mãos estranhas
levarão você embora, com pressa, com resmungos e impaciência; ninguém
abençoará você, ninguém suspirará por você, irão apenas se desfazer de
você o mais rápido que puderem. Vão comprar um ataúde qualquer e vão
levar, da mesma maneira que levaram aquela pobre coitada de hoje, e
depois vão beber no botequim à sua memória. No túmulo haverá lama,
lixo, neve molhada: por que fazer cerimônias para você? ```'Baixe ela aí,
Vaniukha; então a \textit{sorte} até aqui ficou de pernas para o ar para essa
aí. Puxe a corda, seu moleque.'' --- ``Está bom assim.'' --- ``Como está bom?
Está de lado. Era uma pessoa também ou não? Mas tudo bem, pode
enterrar.'' Nem discutir por muito tempo por sua causa vão querer. Vão
enterrar bem rápido com um barro azulado e úmido e vão embora para o
botequim\ldots{} E assim sumirá sua memória na terra; o túmulo dos outros
será visitado pelos filhos, pelos pais, pelos maridos; mas no seu não
haverá nem uma lágrima, nem um suspiro, nem recordações, e ninguém,
ninguém jamais no mundo todo virá visitá"-la; seu nome desaparecerá da
face da terra, como se você nunca sequer tivesse existido ou nascido!
Sujeira e lodo, ainda que você bata lá embaixo, na tampa do caixão, à
noite, quando os mortos se levantam: ``Deixem"-me, pessoas boas, viver no
mundo! Vivi, mas não conheci a vida; de nada serviu minha vida;
beberam"-na num botequim na Sennaia; deixem"-me, pessoas boas, mais uma
vez viver no mundo!\ldots{}''

Alcancei tamanho entusiasmo que minha garganta parecia prestes a ter um
espasmo e\ldots{} de repente parei, levantei"-me num sobressalto e,
inclinando timidamente a cabeça, com o coração batendo forte, comecei a
ouvir com atenção. E havia motivo para perturbar"-se.

Há um tempo eu pressentia que transtornara sua alma e partira seu
coração, e quanto mais eu me certificava disso, mais desejava alcançar
depressa e com a maior força possível o meu objetivo. Era o jogo, o
jogo me entusiasmava; porém, não apenas o jogo\ldots{}

Eu sabia que falava de maneira tensa, artificial, quase livresca;
resumindo, não conseguia falar de outra maneira que não \textit{como num
livro}. Mas isso não me atrapalhava; eu afinal sabia, pressentia que
seria compreendido e que esse próprio caráter livresco serviria para
apoiar ainda mais o meu caso. Mas agora, ao obter o efeito, subitamente
me acovardei. Não, jamais, jamais fora testemunha de tamanho desespero!
Ela estava deitada, de bruços; enfiara com força o rosto num
travesseiro, que segurava com ambas as mãos. Seu coração estava
partido. Todo o seu jovem corpo estremecia, como que em convulsões. Os
soluços contidos em seu peito sufocavam"-na, cortavam"-na, e de repente
escapavam em gemidos e gritos. Então ela colava"-se ao travesseiro com
ainda mais força: ela não queria que alguém ali, nenhuma vivalma que
fosse, soubesse de seu tormento e de suas lágrimas. Ela mordia o
travesseiro, com mordidas arrancava sangue de sua mão (vi isso depois)
ou, agarrando com os dedos suas tranças desgrenhadas, parecia
imobilizar"-se pelo esforço, contendo a respiração e cerrando os dentes.
Fiz menção de dizer"-lhe alguma coisa, de pedir"-lhe que se acalmasse,
mas senti que não conseguiria, e de repente, sentindo calafrios pelo
corpo todo, quase horrorizado, lancei"-me às apalpadelas atrás de minhas
coisas, querendo sair de qualquer jeito e às pressas dali. Estava
escuro: por mais que eu tentasse, não conseguia acabar logo com aquilo.
De repente, tateei uma caixa de fósforos e um castiçal com uma vela
inteira e intocada nele. Assim que a luz invadiu o quarto, Liza de
repente saltou, sentou"-se e, com um rosto como que contorcido, com um
sorriso semidesvairado, deu"-me um sorriso quase apalermado. Sentei"-me
ao lado dela e segurei suas mãos; ela voltou a si, lançou"-se em minha
direção, fez menção de me abraçar, mas não teve coragem, e
silenciosamente inclinou a cabeça para mim.

--- Liza, minha amiga, foi por mal que\ldots{} Me perdoe --- fiz menção de
começar, mas ela apertou minhas mãos em seus dedos com tanta força que
percebi que estava dizendo a coisa errada, e parei.

--- Esse é o meu endereço, Liza, venha à minha casa.

--- Irei\ldots{} --- sussurrou ela em tom decidido, sem levantar a cabeça em
momento algum.

--- Agora partirei, adeus\ldots{} Até logo.

Levantei"-me; ela também se levantou, e de repente corou toda,
estremeceu, pegou o lenço que estava sobre a cadeira e lançou"-o sobre
os ombros, amarrando"-o quase no queixo. Ao fazer isso, ela novamente
sorriu de um modo um tanto doentio, corou e olhou para mim com uma
expressão estranha. Eu me sentia dolorido; apressei"-me a sair,
desaparecer.

--- Espere --- disse ela subitamente, já no saguão, junto à porta,
detendo"-me com a mão e segurando o casaco; colocou a vela a toda a
pressa e saiu correndo: pelo visto, ela se lembrara de alguma coisa ou
queria trazer algo para me mostrar. Ao sair correndo, ela corara
inteira, seus olhos brilhavam, em seus lábios havia um sorriso: que
seria aquilo? Esperei a contragosto; ela voltou depois de um minuto,
com a expressão de quem pedia perdão por alguma coisa. Já não era em
absoluto aquele rosto, aquela expressão de antes: sombria, desconfiada
e obstinada. Sua expressão era agora suplicante, suave, e ao mesmo
tempo confiante, carinhosa, tímida. Era o mesmo olhar que as crianças
lançam às pessoas muito amadas por elas e a quem pedem algo. Seus olhos
eram castanho"-claros, olhos belíssimos, vivos, capazes de refletir em
si tanto o amor como um ódio sombrio.

Sem me explicar nada --- como se eu, por me tratar de algum tipo de ser
superior, devesse saber tudo sem explicações ---, ela me estendeu um papel. Seu
rosto inteiro brilhava nesse momento com o mais ingênuo ar de triunfo, um ar
quase infantil. Abri o papel. Era uma carta escrita para ela por um estudante
de medicina ou algo do gênero; uma declaração de amor muito grandiloquente,
floreada, mas extremamente respeitosa.  Não me lembro agora das expressões, mas
me lembro muito bem de que, em meio ao estilo elevado, notava"-se um verdadeiro
sentimento, que não se pode simular. Quando terminei de ler, deparei com o
olhar dela fixo em mim, um olhar ardente, curioso e de uma impaciência
infantil. Seus olhos se colaram em meu rosto e aguardavam com impaciência o que
eu diria. Em algumas palavras, às pressas, mas com certa alegria e como que se
vangloriando, ela me explicou que estivera em um baile em algum lugar, numa
casa de família, de ``pessoas muito, muito boas, \textit{pessoas de família} e
onde \textit{ainda não sabem de nada}, de absolutamente nada; porque ela
estava ali fazia pouco tempo e apenas por estar\ldots{} Mas ainda não estava de
todo decidida a ficar ali em absoluto e iria embora imediatamente, assim que
pagasse a dívida\ldots{} ``Bem, e lá havia esse estudante, que dançara a noite
toda, que conversara com ela, e então revelou"-se que ainda em Riga, quando
criança, já era conhecido dela, haviam brincado juntos, mas fazia muito tempo;
conhecia os pais dela, mas \textit{daquilo} ele não sabia nada, nada, nada, e
nem suspeitava! E então, no dia após a dança (três dias atrás), ele enviara
esta carta por uma amiga com quem ela fora à festa\ldots{} e\ldots{} bom, e
isso era tudo.''

Como que envergonhada, ela baixou seus olhos brilhantes ao terminar de
contar.

A pobrezinha: ela guardara a carta daquele estudante como uma joia e
saíra correndo para buscar essa sua única joia, não desejando que eu
saísse sem saber que ela também era amada de forma honesta e sincera,
que com ela falavam com respeito. Certamente aquela carta estava fadada
a jazer no fundo de uma caixinha sem quaisquer consequências. Mas não
importava; tenho certeza de que ela a guardaria a vida toda como uma
joia, como seu orgulho e sua justificativa, e agora num momento como
aquele ela se lembrara e trouxera aquela carta para vangloriar"-se
ingenuamente diante de mim, erguer"-se aos meus olhos, para que eu
também visse, para que eu também a louvasse. Eu não disse nada, apertei
sua mão e saí. Queria tanto ir embora\ldots{} Fiz todo o caminho a pé,
embora a neve molhada ainda caísse em flocos. Eu estava extenuado,
abatido, perplexo. Mas a verdade já brilhava em meio à perplexidade.
Uma verdade abjeta!


\section{parte VIII}

Demorei, porém, a admitir essa verdade. Ao acordar de manhã, após
algumas horas de um sono profundo e pesado como chumbo, e imediatamente
recordando todo o dia anterior, fiquei até mesmo admirado com o meu
\textit{sentimentalismo} do dia anterior com relação a Liza, com todos
aqueles arroubos de \textit{horror e compaixão de ontem}. ``Mas que ataque de
histeria feminina, arre!'', concluí. ``E para que é que eu fui empurrar o
meu endereço para ela? E se ela vier? Mas também, que venha; tudo
bem\ldots{}'' Mas, \textit{nitidamente}, a questão principal e mais
importante agora não era essa: era preciso se apressar e salvar a
qualquer preço minha reputação aos olhos de Zverkov e de Símonov. Essa
era a questão principal. Fiquei tão atarefado que esqueci completamente
Liza naquela manhã.

Antes de qualquer coisa, era necessário saldar depressa a dívida
contraída com Símonov no dia anterior. Decidi apelar para um recurso
desesperado: pegar emprestados quinze rublos com Anton Antônitch.
Providencialmente, ele estava naquela manhã numa ótima disposição de
espírito, e imediatamente me emprestou, ao primeiro pedido. Fiquei tão
contente com isso que, ao assinar o recibo, com uma expressão temerária
e \textit{desleixada}, declarei a ele que no dia anterior ``farreara com
os amigos no Hôtel de Paris; despediram"-se de um
camarada, até se poderia dizer que era um amigo de infância, e, sabe,
era um grande pândego, um homem bajulado por todos; bom, mas é claro
que era de uma boa família, tinha uma fortuna significativa e uma
carreira brilhante, era espirituoso, adorável, andava sempre metido com
damas, compreende: bebêramos uma \textit{meia"-dúzia} a mais e\ldots{}'' E correu
tudo bem; disse tudo aquilo com muita facilidade, desembaraço e
satisfação.

Ao chegar em casa, apressei"-me a escrever para Símonov.

Até hoje fico admirado ao me lembrar do tom verdadeiramente
cavalheiresco, bondoso e franco de minha carta. De maneira sagaz e
nobre, e, principalmente, absolutamente sem palavras desnecessárias,
assumia a culpa por tudo. Justificava"-me, ``se apenas me fosse ainda
lícito justificar"-me'', pelo fato de que, devido a uma completa falta de
costume com o vinho, me embriagara com a primeira tacinha, que
(afirmava eu) bebera antes da chegada deles, quando os esperava no
Hôtel de Paris das cinco às seis horas. Pedia desculpas
principalmente a Símonov; mas pedia"-lhe que transmitisse minhas
explicações a todos os outros, especialmente a Zverkov, que eu, ``pelo
que me lembrava como que num sonho'', aparentemente, ofendera.
Acrescentei que eu mesmo os visitaria, mas a cabeça doía, e além de
tudo tinha vergonha. Ficara particularmente satisfeito com essa \textit{certa
leveza}, que beirava o desleixo (perfeitamente decente, porém) e que
subitamente se refletia em minha pena, dando"-lhes logo a entender,
melhor que todos os argumentos possíveis, que eu via ``toda aquela
patifaria de ontem'' de maneira bastante independente; mas que não
estava em absoluto, de forma alguma mortificado como vocês, senhores,
certamente estarão pensando, mas pelo contrário, via as coisas como um
cavalheiro que respeita serenamente a si próprio deve. Como se diz,
águas passadas não movem moinho.

--- Mas é de uma brejeirice digna de um marquês! --- admirava"-me ao reler o
bilhete. --- E tudo porque sou uma pessoa evoluída e instruída! Outros em
meu lugar não saberiam como se livrar dessa, enquanto que eu não só me
safei como já estou aqui novamente a farrear, e tudo porque sou ``um
homem instruído e evoluído de nossos tempos''. E talvez tudo tenha de
fato acontecido ontem por conta do vinho. Hum\ldots{} mas não, não foi por
conta do vinho. Vodca eu não bebi nada, das cinco às seis horas,
enquanto os esperava. Menti para Símonov; menti sem vergonha alguma;
mesmo agora não tenho vergonha\ldots{}

Mas também, que me importa! O mais importante é que me livrei.

Coloquei seis rublos na carta, selei e pedi para Apollon levá"-la a
Símonov. Percebendo que havia dinheiro na carta, Apollon ficou mais
respeitoso e concordou em ir. No fim da tarde saí para dar um passeio.
Minha cabeça ainda doía e rodava devido ao dia anterior. Mas quanto
mais se aproximava a noite e se adensava o crepúsculo, mais mudavam e
se confundiam as minhas impressões, e com elas também meus pensamentos.
Algo não perecera em meu âmago, nas profundezas de meu coração e de
minha consciência, não queria perecer e se expressava numa pungente
tristeza. Vagava sobretudo pelas ruas mais povoadas e operárias, pelas
Meschánskie,\footnote{ Havia na região central de São Petersburgo três
ruas denominadas \textit{Meschánskaia}: a Bolcháia Meschánskaia (hoje
Kazánskaia), a Sriêdniaia Meschánskaia (hoje Grajdánskaia) e a Málaia
Meschánskaia (hoje Kaznatchéiskaia). Refere"-se comumente a essa parte
da cidade como \textit{a Petersburgo de Dostoiévski}.} pela Sadóvaia, próximo
ao Jardim de Iussúpov. Gostava especialmente de passear por essas ruas
sempre ao entardecer, exatamente quando lá se condensava essa multidão
de variados transeuntes, operários e artesãos, com feições furiosamente
preocupadas, indo cada um para sua casa após a jornada diária de
trabalho. Gostava justamente daquela azáfama miserável, daquele
prosaísmo insolente. Dessa vez, todo aquele tropel na rua me irritava
ainda mais. Não conseguia de forma alguma recobrar o controle de mim
mesmo, achar um caminho. Algo agitava"-se, apertava meu peito sem cessar
e não queria aquietar"-se. Voltei para casa completamente abatido.
Exatamente como se em minha alma pesasse algum crime.

Torturava"-me constantemente o pensamento de que Liza viria. Era estranho
o fato de que, de todas as lembranças do dia anterior, a lembrança dela
me torturava como que especialmente, de maneira como que totalmente
separada. Todo o resto eu já conseguira esquecer por completo até o fim
da tarde, deixara tudo para trás e continuava ainda de todo satisfeito
com minha carta a Símonov. Mas com essa questão eu não estava
satisfeito. Era como se apenas Liza me atormentasse. ``E se ela vier?'',
pensava eu sem parar. ``Mas e daí? Tudo bem, que venha. Hum! A única
coisa desagradável é que ela vai ver, por exemplo, como eu vivo. Ontem
eu pareci a ela\ldots{} um herói\ldots{} e agora, hum! É aliás desagradável que eu
tenha decaído tanto. Esse apartamento é uma total miséria. E eu decidi
ontem ir com aquela roupa almoçar! E o meu sofá de oleado, com o forro
saindo! E o meu roupão, que nem serve para me cobrir! Uns farrapos\ldots{} E
ela vai ver tudo isso; e vai ver o Apollon também. Esse animal na certa
vai ofendê"-la. Vai aborrecê"-la só para fazer uma grosseria comigo. E
eu, é claro, como me é de costume, vou me acovardar, começarei a
saltitar na frente dela, a cobrir"-me com as abas do roupão, a sorrir,
começarei a mentir. Ah, que detestável! E não é nem isso a coisa mais
detestável! Há algo pior, mais abjeto, mais infame! Sim, mais infame!
Que é novamente, novamente vestir essa desonesta máscara da
mentira!\ldots{}''

Ao chegar a esse pensamento, acabei por inflamar"-me:

``Mas por que desonesta? Como desonesta? Ontem falei sinceramente.
Lembro"-me de que também havia em mim um sentimento verdadeiro. Eu
queria justamente despertar nela sentimentos nobres\ldots{} É bom que ela
tenha chorado, será para o bem\ldots{}''

Mas mesmo assim eu de maneira alguma conseguia me acalmar.

Por toda aquela noite, depois de já haver voltado para casa, já depois
das nove horas, quando, de acordo com meus cálculos, Liza de forma
alguma poderia chegar, ela mesmo assim parecia surgir diante de mim, e
vinha"-me à lembrança sempre numa mesmíssima posição. Um momento preciso
de toda a noite anterior me voltava à mente de maneira especialmente
vívida: foi quando iluminei o quarto com um fósforo e vi seu rosto
pálido, contorcido, com uma expressão de martírio. E quão lastimável,
quão artificial, quão contorcido era seu sorriso naquele momento! Mas
eu ainda não sabia então que mesmo depois de quinze anos continuaria
imaginando Liza justamente com aquele sorriso lastimável, contorcido e
desnecessário que ela tinha naquele momento.

No dia seguinte, eu já estava pronto novamente para achar tudo aquilo
uma tolice, algo criado por uma excitação nervosa, mas, principalmente,
\textit{um exagero}. Sempre tivera consciência desse
meu ponto fraco, e às vezes muito o temia: ``tudo eu exagero, é nisso
que peco'', repetia para mim mesmo de hora em hora. Mas por outro lado,
``por outro lado talvez Liza ainda assim venha''; esse era o refrão que
concluía todas as minhas reflexões de então. Eu estava a tal ponto
agitado que às vezes me enfurecia. ``Ela virá! Certamente virá!'',
exclamava, correndo pelo quarto. ``Se não hoje, virá amanhã, mas vai me
encontrar! É o maldito romantismo de todos esses
\textit{corações puros}! Ah, a torpeza, a estupidez,
a mediocridade dessas \textit{sórdidas almas sentimentais}! Mas como não
entender, como se pode não entender?\ldots{}'' Mas nesse ponto, eu mesmo
parei, em grande perturbação.

``E quão poucas'', pensava eu de passagem, ``quão poucas palavras seriam
necessárias, quão poucos idílios seriam necessários (e ainda por cima
um idílio fajuto, livresco, inventado), para imediatamente moldar toda
a alma de uma pessoa de acordo com a minha vontade. Isso é que é
virgindade! Isso é que é um terreno fértil!''

Às vezes me ocorria o pensamento de eu mesmo ir até ela, \textit{contar tudo a
ela} e convencê"-la a não vir até minha casa. Mas ao pensar nisso,
tamanho ódio se erguia em mim que, pelo visto, eu poderia esmagar essa
\textit{maldita} Liza, se ela de repente surgisse perto de mim; eu a
ofenderia, cuspiria nela, expulsaria, bateria!

Passou"-se, porém, um dia, outro, um terceiro; ela não veio, e eu comecei
a me acalmar. Ficava especialmente animado e à vontade depois das nove
horas, até comecei a sonhar às vezes, e de maneira bastante doce: ``Eu,
por exemplo, salvava Liza, justamente pelo fato de que ela viera até
mim e de que eu falara com ela\ldots{} Eu fazia com que se desenvolvesse,
educava"-a. Finalmente, percebia que ela me amava, me amava
apaixonadamente. Eu dissimulava, fazia que não compreendia (não sei,
aliás, para que dissimulava; provavelmente para tudo parecer mais
bonito). Finalmente, ela, toda embaraçada, bela, tremendo e soluçando,
atirava"-se às minhas pernas, dizendo que eu era seu salvador e que ela
me amava mais que tudo no mundo. Fico admirado, mas\ldots{} \textit{Liza}, dizia
eu, \textit{mas será que você pensa que eu não percebi o seu amor? Eu vi tudo,
adivinhei, mas não ousei atentar contra seu coração primeiro porque
exercia uma influência sobre você e temia que você por gratidão se
visse forçada a corresponder o meu amor, suscitando em si mesma à força um 
sentimento que talvez não exista, e eu não queria isso porque
isso é\ldots{} despotismo\ldots{} É indelicado (bem, resumindo, eu me enrolava
nessas sutilezas meio europeias, no estilo de George Sand,\footnote{
Pseudônimo de Amantine Lucile Aurore Dupin (1804--1876), escritora
francesa cujos romances foram muito populares nos anos 1850.}
indizivelmente nobres\ldots{}). Mas agora, agora você é minha, é minha
criação, é pura, é bela, é minha bela esposa.}''

%\medskip

\begin{verse}
\textit{E então, senhora, firme e livremente,\\
entra em minha casa, soberana!}\footnote{Versos finais do poema de
Nekrássov usado como epígrafe da parte 2 deste livro. A tradução é de Rafael
Frate.}
\end{verse}

%\medskip

``Depois começaríamos a viver juntos, iríamos para o exterior etc. etc.''.
Resumindo, até eu começava a achar aquilo infame, e acabava tudo comigo
mostrando a língua para mim mesmo.

``Mas não vão deixá"-la sair, uma \textit{ordinária}!'', pensei. ``Parece que não
as deixam passear muito, ainda mais à noite (por algum motivo me
parecia que ela certamente deveria chegar à noite, mais precisamente às
sete horas). Mas por outro lado, ela disse que ainda não se vendera
completamente lá, ainda tinha certos direitos privados; então, hum! Com
mil diabos, virá, certamente virá!''

O bom era que, enquanto isso, Apollon me distraía com suas grosserias.
Tirava"-me completamente do sério! Ele era minha chaga, meu flagelo,
enviado a mim pela providência. Trocávamos farpas constantemente,
havia alguns anos, e eu o odiava. Deus meu, como eu o odiava! Creio que
não tenha nunca odiado tanto alguém na vida como eu o odiava,
especialmente em certos momentos. Era um homem idoso, altivo, que
durante parte do tempo trabalhava de alfaiate. Mas por algum motivo
desconhecido, ele me desprezava desmesuradamente, e olhava para mim com
um ar insuportavelmente arrogante. Ele, aliás, olhava para todos com ar
arrogante. Bastava olhar para aquela cabeça de um loiro desbotado e
cabelo penteado liso, para aquele topete batido sobre a testa e untado
com óleo, para aquela boca séria, sempre em forma de triângulo, para se
sentir que se tratava de um ser que nunca duvidava de si mesmo. Era um
pedante de marca maior, o maior pedante de todos que eu já conheci na
terra; e tudo isso com um amor"-próprio digno talvez apenas de um
Alexandre Magno. Era apaixonado por cada um de seus botões, por cada
uma de suas unhas; absolutamente apaixonado, e aquilo transparecia!
Tratava"-me de maneira totalmente despótica, falava comigo pouquíssimo,
e se acontecia de ele lançar um olhar para mim, olhava"-me com uma
expressão dura, majestosamente autoconfiante e constantemente
zombeteira, que às vezes me enfurecia. Cumpria suas funções com um ar
tal que parecia me fazer um imenso favor. Porém, ele não fazia
absolutamente nada para mim e sequer considerava seu dever fazer alguma
coisa. Não podia haver dúvida de que ele me considerava o maior idiota
de todo o mundo, e se \textit{me mantinha junto de si} era unicamente porque
podia receber de mim o salário todo mês. Ele consentia em \textit{não fazer
nada} em minha casa por sete rublos mensais. Por conta dele muitos de
meus pecados serão perdoados. O ódio às vezes chegava ao ponto de quase
ter convulsões só de ouvir seus passos. Mas o que me parecia
particularmente detestável era o seu ceceio. Ele tinha a língua um
pouco mais longa do que se deve ou algo desse tipo, e por isso ele
constantemente ceceava e balbuciava e, pelo visto, orgulhava"-se
terrivelmente disso, imaginando que isso lhe conferia muitíssima
dignidade. Falava baixo, comedidamente, com os braços atrás das costas
e os olhos voltados para o chão. Ele me enfurecia particularmente
quando, por vezes, começava a ler atrás de seu tapume os Salmos. Travei
muitas batalhas por conta dessas leituras. Mas ele gostava muitíssimo
de ler durante a noite, numa voz baixa e regular, cantada, como que
numa missa fúnebre. O curioso é que ele acabou desse jeito: agora
trabalha na leitura dos Salmos em missas para os mortos, e além disso
extermina ratos e produz graxa. Mas na época eu não conseguia
expulsá"-lo, como se ele tivesse se fundido quimicamente com a minha
existência. Além disso, ele não concordaria em ir embora da minha casa
por nada. Era"-me impossível viver num \textit{chambre
garnie}:\footnote{ No original, em francês, ``quarto
mobiliado''.} meu apartamento era o meu palacete, a minha casca, a minha
caixinha em que eu me escondia de toda a humanidade, e Apollon, sabe
Deus por quê, parecia"-me pertencer a este apartamento, e por longos
sete anos eu não pude expulsá"-lo.

Atrasar por exemplo o salário dele por dois ou três dias que fosse era
impossível. Ele criaria tamanho caso que eu não saberia onde me enfiar.
Mas naqueles dias eu estava a tal ponto exasperado com todos que
decidi, por algum motivo e para algum fim, \textit{punir} Apollon e não
entregar a ele o salário por mais duas semanas. Fazia já tempo, uns
dois anos, que eu pretendia fazer aquilo; unicamente para provar"-lhe
que ele não deveria ousar fazer"-se de importante comigo e que, se eu
quisesse, poderia a qualquer momento não lhe entregar o salário.
Propus"-me a não falar sobre isso com ele, até mesmo a me manter em
silêncio propositalmente para vencer seu orgulho e fazer com que ele
mesmo começasse a falar primeiro sobre o salário. Eu então tiraria os
sete rublos da caixa, mostraria a ele que eu tinha o dinheiro já
devidamente separado, mas que eu ``não queria, não queria, simplesmente
não queria entregar"-lhe o salário, não queria, porque eu
\textit{assim quisera}'', porque aquela era a \textit{minha
vontade senhoril}, porque ele era desrespeitoso, porque ele era um
grosseirão; mas que se ele pedisse respeitosamente, eu talvez me
abrandasse e desse; do contrário ele esperaria mais duas semanas, mais
três semanas, mais um mês inteiro\ldots{}

Mas não importava quão raivoso eu fosse, ele vencia mesmo assim. Não
aguentei nem quatro dias. Começou do mesmo jeito que sempre começava em
situações como aquela, porque situações como aquela já haviam
acontecido, já haviam sido experimentadas (e devo notar que eu, de
antemão, conhecia perfeitamente sua tática infame), a saber: começava
que ele então dirigia a mim um olhar extremamente severo, passava
alguns minutos sem desviar os olhos, especialmente ao me encontrar ou
seguindo para fora de casa. Se eu, por exemplo, suportasse e fizesse
cara de que não percebia esses olhares, ele, sempre em silêncio,
procedia a novas torturas. Às vezes, sem mais nem menos, ele de repente
entrava em silêncio e suavemente em meu quarto, quando eu estava apenas
caminhando ou lendo, parava junto à porta, colocava os braços atrás das
costas, afastava os pés e dirigia a mim o seu olhar, já não somente
severo, mas de completo desprezo. Se eu porventura perguntasse a ele o
que queria, ele não respondia nada, continuava a olhar para mim
fixamente por mais alguns segundos, e depois, apertando os lábios com
uma força peculiar e com um ar significativo, lentamente dava
meia"-volta e saía lentamente em direção a seu quarto. Umas duas horas
depois, novamente saía e novamente aparecia, da mesma maneira, na minha
frente. Acontecia de eu, em fúria, já sequer perguntar o que ele
queria, mas simplesmente levantar brusca e imperiosamente a cabeça e
também começar a olhar para ele fixamente. Às vezes ficávamos assim,
olhando um para o outro, por uns dois minutos; finalmente ele dava
meia"-volta, lentamente e com ares de importância, e saía novamente por
duas horas.

Se eu, mesmo com isso, continuasse sem me dar conta e persistisse em
rebelar"-me, ele começava de repente a suspirar, olhando para mim, a
suspirar longa e profundamente, como se sondasse apenas com esses
suspiros todas as profundezas de minha decadência moral, e é claro que
tudo terminava finalmente com ele triunfando por completo: eu me
enfurecia, gritava, mas no ponto central da discussão eu era mesmo
assim forçado a ceder.

Desta vez, mal começaram as habituais manobras de \textit{olhares severos} e eu
imediatamente saí de mim e em fúria me lancei contra ele. Eu já estava
farto, mesmo sem aquilo.

--- Pare! --- gritei em delírio, quando ele dava meia"-volta lenta e
silenciosamente, com uma mão atrás das costas, e voltava para seu
quarto. --- Pare! Volte, volte, estou falando com você! --- E devo ter
urrado de maneira tão pouco natural que ele deu meia"-volta e até com
certa surpresa começou a me observar. Seguiu, porém, sem dizer uma
palavra sequer, o que me enfureceu.

--- Como você ousa entrar no meu quarto sem ser chamado e ficar olhando
assim para mim? Responda!

Mas após olhar calmamente para mim por meio minuto, ele novamente
começou a dar meia"-volta.

--- Pare! --- berrei, aproximando"-me dele apressadamente. --- Não se mova!
Então. Responda já: você entrou aqui para olhar o quê?

--- Se o senhor tem agora alguma ordem para me dar, meu dever é cumpri"-la
--- respondeu ele, calando"-se novamente em seguida, ceceando tranquila e
comedidamente, erguendo as sobrancelhas e calmamente inclinando a
cabeça de um ombro a outro: tudo isso com uma serenidade aterradora.

--- Não é isso, não é isso que eu estou perguntando, seu carrasco! ---
gritei, tremendo de ódio. --- Estou perguntando, seu carrasco, por que
você veio até aqui: está vendo que eu não vou entregar seu salário, por
orgulho não quer você mesmo se inclinar e pedir, e por isso vem aqui
com seus olhares estúpidos para me punir, torturar, e nem suspeita,
seu carrasco, o quanto isso é estúpido, estúpido, estúpido, estúpido,
estúpido!

Ele fez menção de novamente dar meia"-volta em silêncio, mas eu o detive.

--- Escute --- gritei para ele. --- Aqui está o dinheiro, está vendo? Está
aqui! (Tirei"-o da mesinha.) Os sete rublos, mas você não vai
recebê"-los, não vai recebê"-los enquanto não vier respeitosamente, com
a cabeça abaixada, pedir perdão para mim. Ouviu?!

--- Isso não pode ser! --- respondeu ele com uma autoconfiança um tanto
artificial.

--- Mas será! --- gritei. --- Dou a você a minha palavra, será!

--- Nem tenho por que pedir perdão a você --- continuou ele, como se não
percebesse em absoluto os meus gritos. --- Foi o senhor afinal que me
chamou de \textit{carrasco}, e por essa ofensa posso me queixar do senhor a
qualquer momento à polícia do bairro.

--- Vá! Faça a queixa! --- berrei. --- Vá agora, neste minuto, neste segundo!
E mesmo assim você é um carrasco! Carrasco! Carrasco! --- mas ele apenas
olhou para mim, depois virou"-se e, já sem ouvir os gritos com que o
chamava, foi suavemente até seu quarto, sem se virar.

``Se não fosse pela Liza, nada disso aconteceria!'', decidi comigo mesmo.
Depois, após ficar parado por um minuto, com um ar de importância e
solene, mas com o coração batendo lentamente e com força, eu mesmo me
dirigi até ele atrás do biombo.

--- Apollon! --- disse eu em voz baixa e pausadamente, mas arquejando. --- Vá
imediatamente e sem demora até a delegacia de polícia do bairro!

Ele, entrementes, já se sentara junto à sua mesa, pusera os óculos e
começara a costurar alguma coisa. Mas, ao ouvir a minha ordem, ele
subitamente soltou uma risada.

--- Agora, vá nesse momento! Vá ou você não imagina o que acontecerá!

--- O senhor de fato não está em seu juízo perfeito --- notou ele, sem
sequer erguer a cabeça, ceceando lentamente como sempre e continuando a
enfiar a linha na agulha. --- E onde já se viu um homem mandar chamar a
polícia contra si mesmo? Mas se quer me assustar, está se esgoelando à
toa, pois não vai dar em nada.

--- Vá! --- gani, segurando"-o pelos ombros. Senti que bateria nele a
qualquer momento.

Mas não ouvi que, nesse mesmo momento, de repente, a porta de entrada
abriu"-se lenta e silenciosamente, e uma figura entrou, parou, perplexa,
e começou a nos observar. Olhei, fiquei morto de vergonha e corri para
meu quarto. Lá, puxando meus cabelos com ambas as mãos, encostei a
cabeça na parede e congelei naquela posição.

Uns dois minutos depois, ouviram"-se os passos vagarosos de Apollon.

--- Há \textit{uma mulher} aqui perguntando pelo
senhor --- disse ele, olhando para mim de maneira especialmente severa,
depois deu"-lhe passagem e deixou"-a entrar: Liza. Ele não queria sair e
nos observava com um ar zombeteiro.

--- Vá embora! Vá embora! --- ordenei a ele, desconcertado. Nesse momento,
meu relógio retesou"-se, sibilou e bateu as sete horas.


\section{parte IX}

\setlength{\epigraphwidth}{.55\textwidth}
\begin{epigraphs} 
\qitem{
E então, senhora, firme e livremente,\\
entra em minha casa, soberana!}
{\textsc{n.\,a.\,nekrássov}}
\end{epigraphs}

\noindent{}Fiquei parado diante dela, mortificado, embaraçado, repugnantemente
constrangido e, aparentemente, sorria, tentando com todas as forças me
cobrir com a aba do meu felpudo roupãozinho de algodão; enfim, era tal
e qual eu ainda há pouco imaginara durante meu ataque de depressão.
Apollon ficou parado entre nós por uns dois minutos e depois saiu, mas
não me senti melhor. O pior de tudo foi que ela também de repente ficou
constrangida, a tal ponto que eu mesmo não esperava. Ao olhar para mim,
é claro.

--- Sente"-se --- disse eu maquinalmente, empurrando para ela uma cadeira ao
lado da mesa, enquanto eu me sentei no sofá. Ela imediatamente me
obedeceu e sentou"-se, olhando para mim com os olhos arregalados e
nitidamente aguardando agora algo de mim. Até essa ingenuidade da
espera me levou à fúria, mas eu me contive.

O certo seria tentar não reparar em nada, como se tudo estivesse normal,
mas ela\ldots{} E eu tinha a vaga sensação de que ela me pagaria caro
\textit{por tudo aquilo}.

--- Você me achou numa posição estranha, Liza --- comecei, gaguejando, e
sabendo que era justamente dessa maneira que eu não devia começar.

--- Não, não, não imagine nada! --- gritei ao ver que ela subitamente
corara. --- Eu não me envergonho de minha pobreza\ldots{} Pelo contrário, vejo
minha pobreza com orgulho. Sou pobre, porém honrado\ldots{} É possível ser
pobre e honrado --- resmunguei. --- A propósito\ldots{} Quer chá?

--- Não\ldots{} --- ela fez menção de começar.

--- Espere!

Saltei e corri até o quarto de Apollon. Precisava ir para algum lugar,
sumir.

--- Apollon --- cochichei num tom febril e atropelado, jogando diante dele
os sete rublos, que haviam permanecido o tempo todo em minha mão. ---
Aqui está seu salário. Está vendo? Estou entregando. Mas por outro lado
você deve me salvar: traga imediatamente da taverna um pouco de chá e
dez torradas. Se você não quiser ir, fará um homem infeliz! Você não
sabe que mulher é essa\ldots{} É tudo! Você deve estar pensando alguma
coisa\ldots{} Mas você não sabe que mulher é essa!\ldots{}

Apollon, que já se sentara para trabalhar e já pusera novamente os
óculos, antes de tudo, sem soltar a agulha, em silêncio olhou de
soslaio para o dinheiro; depois, sem prestar atenção alguma em mim e
sem me responder nada, continuou a manusear a linha que ainda tentava
enfiar na agulha. Esperei uns três minutos, em pé diante dele, com os
braços numa posição \textit{à la} Napoleão. Minhas
têmporas estavam empapadas de suor; eu estava até pálido, podia sentir.
Mas, graças a Deus, decerto ele ficou com pena ao olhar para mim.
Depois de terminar com sua linha, ele lentamente levantou"-se do lugar,
lentamente afastou a cadeira, tirou lentamente os óculos, contou
lentamente o dinheiro e afinal, perguntando a mim por cima dos
ombros ``devo pegar uma porção inteira?'', lentamente saiu do cômodo.
Quando voltava ao local em que estava Liza, no caminho ocorreu"-me o
pensamento: não seria o caso de sair correndo, do jeito que eu estava,
de roupãozinho, sem rumo, e depois ver o que aconteceria?

Sentei"-me novamente. Ela olhava para mim com inquietação.

Ficamos em silêncio por alguns minutos.

--- Vou matá"-lo! --- gritei subitamente, batendo com força o punho sobre a
mesa, de forma que a tinta jorrou do tinteiro.

--- Ai, mas o que está dizendo! --- gritou ela, estremecendo.

--- Vou matá"-lo, matá"-lo! --- gani, batendo na mesa, em completo delírio e
ao mesmo tempo compreendendo perfeitamente quão estúpido era estar em
tal delírio.

--- Você não sabe, Liza, o que é esse carrasco para mim. Ele é o meu
carrasco\ldots{} Agora ele foi buscar torradas; ele\ldots{}

E de repente rebentei em lágrimas. Estava tendo um ataque. Que vergonha
sentia em meio aos soluços; mas eu já não podia contê"-los. Ela se
assustou.

--- O que há com você?! O que há com você?! --- gritava ela, agitando"-se ao
meu redor.

--- Água, traga"-me água, está ali! --- resmunguei numa voz fraca, embora
tivesse, no âmago, a consciência de que poderia muito bem passar sem 
água e sem resmungar com uma voz fraca. Mas eu, como se diz,
\textit{representava} para salvar as aparências, embora o ataque fosse
real.

Ela me trouxe a água, olhando para mim consternada. Nesse momento,
Apollon trouxe o chá. Tive de repente a sensação de que aquele chá
comum e prosaico era terrivelmente indecente e miserável depois de tudo
que houve, e corei. Liza olhava para Apollon quase com espanto. Ele
saiu sem sequer olhar para nós.

--- Liza, você me despreza? --- disse eu, olhando para ela fixamente,
tremendo de impaciência por saber o que ela pensava.

Ela pareceu constrangida, e não conseguiu responder nada.

--- Beba o chá! --- disse eu com raiva. Tinha raiva de mim mesmo, mas é
claro que era a ela que caberia pagar por isso. Um ódio imenso por ela
ardeu de repente em meu coração; tamanho que eu poderia até matá"-la,
creio. Para me vingar dela, jurei mentalmente não falar com ela um
palavra sequer o tempo todo. ``Se ela é que é o motivo de tudo'', pensei.

Nosso silêncio já durava uns cinco minutos. O chá continuava em cima da
mesa; não o tocamos: cheguei ao ponto de não querer começar a beber
propositalmente, apenas para fazê"-la sentir"-se ainda mais incomodada; ela
mesma parecia sem jeito de começar. Algumas vezes ela olhou para mim de
uma maneira perplexa e triste. Eu seguia obstinadamente em silêncio. O
maior mártir era, é claro, eu mesmo, porque tinha plena consciência de
toda a repulsiva baixeza de minha estupidez raivosa, e ao mesmo tempo
não podia de forma alguma me conter.

--- Eu quero\ldots{} sair de lá\ldots{} de uma vez por todas --- ela fez menção de
começar para de alguma maneira quebrar o silêncio, mas pobrezinha! Era
justamente assim que ela não devia começar a falar num momento já
bastante estúpido como aquele, para um homem já bastante estúpido como
eu. Meu coração até doeu de pena por sua inabilidade e franqueza
desnecessária. Mas algo horrível esmagou em mim imediatamente toda a
pena; até me provocou ainda mais: que tudo sumisse no mundo! Mais cinco
minutos se passaram.

--- Não estou atrapalhando? --- começou ela timidamente, numa voz quase
inaudível, e pôs"-se a levantar.

Mas assim que vi esse primeiro vislumbre de dignidade ofendida, tremi de
raiva e imediatamente estourei.

--- Por que é que você veio até aqui? Diga"-me, por favor --- comecei,
ofegante e sem sequer concatenar logicamente as minhas palavras. Queria
dizer tudo de uma vez, de um só fôlego; sequer me importava com que
começar.

--- Por que você veio? Responda! Responda! --- gritava eu, quase fora de mim.
--- Eu vou dizer a você, minha cara, por que é que você veio. Você veio porque
eu disse a você \textit{palavras de compaixão} naquele dia. E aí você ficou
enternecida e quis novamente \textit{palavras de compaixão}. Pois saiba, saiba que eu
estava caçoando de você naquele dia. Agora também estou caçoando. Por que está
tremendo? Sim, estava caçoando! Eu havia sido ofendido antes, durante o almoço,
por aqueles mesmos que chegaram lá antes de mim. Fui até lá para dar uma surra em um
deles, um oficial; mas não consegui, não os alcancei; era preciso descontar a
ofensa em alguém, conseguir o que eu queria, você apareceu, eu despejei em você
a minha raiva e me diverti. Fui humilhado e quis eu também humilhar; fui
tratado como um trapo, então eu também quis demonstrar algum poder\ldots{} Foi
isso que aconteceu, mas você achou que eu tinha vindo apenas com o intuito de
salvá"-la, não foi? Você não achou? Você não achou?

Eu sabia que ela talvez se enrolasse e não entendesse os pormenores; mas
eu também sabia que ela entenderia perfeitamente bem a essência. Foi o
que aconteceu. Ela ficou branca como um lençol, tentou dizer algo, seus
lábios contorceram"-se dolorosamente; mas ela caiu sobre a cadeira como
que partida por um machado. Depois, ficou o tempo inteiro me ouvindo,
com a boca e os olhos abertos e tremendo de terror e medo. O cinismo, o
cinismo de minhas palavras esmagou"-a\ldots{}

--- Salvar! --- continuei, saltando da cadeira e correndo diante dela de um
lado para o outro pelo quarto. --- Salvar de quê?! Eu mesmo devo ser pior
que você. Você mesma jogou na minha cara naquele dia, quando eu passava
um sermão: ``E você'', disse, ``veio aqui para quê? Por acaso para dar
lição de moral?'' Era de poder, de poder que eu precisava naquele dia,
de um jogo que eu precisava, tinha de conseguir extrair as suas
lágrimas, a sua humilhação, a sua histeria: era disso que eu precisava
naquele dia! Eu mesmo não suportei naquele dia, porque sou um calhorda,
fiquei assustado e, sabe Deus por quê, caí na besteira de dar meu endereço
para você. Depois, antes mesmo de chegar em casa, praguejei como um
louco contra você por conta desse endereço. Eu já a odiava porque eu
havia mentido para você. Porque eu gosto apenas de brincar com as
palavras, de sonhar em minha mente, mas na verdade sabe do que eu
preciso? Que sumam todos vocês, é disso que eu preciso! Preciso de
tranquilidade. Venderia imediatamente o mundo inteiro por um copeque
para que não me perturbassem. Se quero que o mundo suma ou que eu
possa beber meu chá? Pois digo que prefiro que o mundo suma, desde
que eu possa sempre beber meu chá. Você sabia disso ou não? Bom, eu sei
que sou um ordinário, um canalha, um egoísta, um preguiçoso. Passei
esses três dias tremendo de medo de que você viesse. Mas sabe o que me
perturbou especialmente nesses três dias? O fato de que naquele dia eu
pareci um herói a você, enquanto que agora você está me vendo de
repente nesse roupãozinho rasgado, miserável, abjeto. Eu disse a você
agora há pouco que não tenho vergonha de minha pobreza; pois saiba que
me envergonho, me envergonho mais do que qualquer outra coisa, temo
muito mais do que tudo, muito mais do que se eu roubasse, porque sou
vaidoso como se me tivessem arrancado a pele e até o ar me causasse
dor. Mas será possível que você, mesmo agora, ainda não tenha percebido
que eu jamais irei perdoá"-la por ter me visto nesse roupãozinho, quando
eu me atirava como um cãozinho raivoso sobre Apollon? O salvador, o
antigo herói, lançando"-se, como um cachorrinho desgrenhado e sarnento,
sobre seu criado, e este rindo"-se dele! Também as lágrimas de agora há
pouco, que eu, como uma mulherzinha envergonhada, não pude conter
diante de você, jamais perdoarei! O fato de que eu agora esteja
confessando tudo isso \textit{a você} também jamais
perdoarei! Sim, você: somente você deverá pagar por tudo isso, porque
você apareceu assim, porque eu sou um ordinário, porque eu sou o mais
abjeto, o mais patético, o mais mesquinho, o mais estúpido, o mais
invejoso de todos os vermes da terra, que não são em absoluto melhores
que eu, mas que, sabe Deus por que motivo, nunca ficam constrangidos;
enquanto eu vou passar a vida levando tabefes de qualquer piolho: essa
é a minha maior característica! E que tenho eu com o fato de que você
não vai entender nada disso! E que tenho eu, mas o que tenho, que tenho
a ver com você e com o fato de que você vai ou não morrer lá? Mas você
entende que agora, ao dizer a você tudo isso, eu vou odiá"-la pelo fato
de que você estava aqui escutando? Uma pessoa só se manifesta dessa
maneira uma vez na vida, e ainda assim em histeria!\ldots{} Do que mais você
precisa? Por que depois de tudo isso você continua aí parada na minha
frente, me torturando, e não vai embora?

Mas então algo estranho subitamente aconteceu.

Eu estava a tal ponto acostumado a pensar e imaginar tudo como num livro
e a conceber tudo que havia no mundo tal como antes havia inventado em
meus sonhos, que demorei então a entender uma estranha circunstância. O
que aconteceu foi o seguinte: Liza, ofendida e destruída por mim,
compreendeu muito mais do que eu imaginara. De tudo o que foi dito, ela
compreendeu justamente aquilo que uma mulher sempre entende melhor, se
ela ama sinceramente: que eu mesmo era \mbox{infeliz}.

Ao sentimento amedrontado e ofendido sucedeu"-se em seu rosto,
inicialmente, uma amarga perplexidade. Quando eu comecei a me chamar de
canalha e de ordinário e minhas lágrimas começaram a correr (disse toda
aquela tirada em lágrimas), todo o rosto dela estremeceu em convulsões.
Ela fez menção de se levantar, de me conter; quando terminei, não foi
aos meus gritos que ela prestou atenção: ``Por que você está aqui, por
que não vai embora?!''; mas sim ao fato de que para mim devia ser muito
difícil dizer tudo aquilo. E ainda parecia amedrontada, a pobrezinha;
ela se considerava infinitamente inferior a mim; como é que poderia ela
se irritar ou se ofender? Ela subitamente saltou da cadeira numa
espécie de ímpeto incontrolável e, precipitando"-se em minha direção,
mas ainda assim acanhada e sem ousar sair do lugar, estendeu os braços
para mim\ldots{} Meu coração então confrangeu"-se. Ela então lançou"-se
subitamente em minha direção, envolveu meu pescoço com seus braços e
começou a chorar. Eu também não me contive e solucei de uma forma que
nunca antes me acometera\ldots{}

--- Não me deixam\ldots{} Eu não posso ser\ldots{} bom! --- mal consegui dizer, indo
depois até o sofá, onde caí de bruços e solucei por quinze minutos num
verdadeiro ataque histérico. Ela apertou"-se contra mim, me abraçou e
como que congelou naquele abraço.

Mas ainda assim a questão era que o ataque histérico deveria passar. E
com isso (e estou escrevendo a mais repugnante verdade), deitado de
bruços no sofá, e enterrando com força o rosto em meu imprestável
travesseiro de couro, comecei pouco a pouco, de longe,
involuntariamente, mas de maneira irresistível, a sentir que agora já
me seria incômodo levantar a cabeça e olhar para Liza direto nos olhos.
Do que eu tinha vergonha? Não sei, mas eu tinha vergonha. Também passou
pela minha cabeça perturbada que os papéis agora haviam se invertido
definitivamente, que a heroína agora era ela, e que eu era uma criatura
tão humilhada e destruída como ela fora diante de mim naquela noite,
quatro dias antes\ldots{} E tudo isso me ocorreu ainda naqueles momentos em
que eu estava deitado de bruços no sofá!

Meu Deus! Será possível que eu então tenha sentido inveja dela?

Não sei, até hoje não consigo dizer, e na época é claro que podia
entender isso ainda menos do que agora. Sem poder e tirania sobre
alguém eu não posso viver\ldots{} Mas\ldots{} Mas com raciocínio não se explica
nada, e por conseguinte não há o que raciocinar.

Eu, porém, me contive e ergui a cabeça; era necessário erguê"-la em algum
momento\ldots{} E até hoje tenho certeza de que justamente por ter sentido
vergonha de olhar para ela é que, em meu coração, de repente acendeu"-se
e inflamou"-se então um outro sentimento\ldots{} Um sentimento de dominação e
de posse. Meus olhos brilharam de paixão, e apertei com força suas
mãos. Como eu a odiava e como me sentia atraído por ela naquele
momento! Um sentimento reforçava o outro. Era quase semelhante a um ato
de vingança!\ldots{} Em seu rosto transpareceu primeiramente uma certa
perplexidade, até mesmo um certo medo, mas apenas por um momento. Ela
me abraçou com entusiasmo e ardor.


\section{parte X}

Depois de quinze minutos, eu corria pelo quarto, de um lado para o outro, numa
impaciência furiosa, aproximando"-me de minuto em minuto do biombo e olhando por
uma frestinha para Liza. Ela estava sentada no chão, com a cabeça apoiada na
cama, e devia estar chorando. Mas ela não ia embora e era isso que me irritava.
Dessa vez ela já sabia de tudo.  Eu a ofendera definitivamente, mas\ldots{} não
há por que contar. Ela percebeu que o meu arroubo de paixão era justamente uma
vingança, uma nova humilhação para ela, e que ao meu ódio já existente --- que
era quase indefinido ---, somava"-se agora um ódio já \textit{pessoal e
invejoso} dirigido a ela\ldots{} Não posso garantir, porém, que ela tenha
compreendido tudo isso claramente; mas em compensação ela compreendera
perfeitamente que eu era um homem vil e, principalmente, sem condições de
amá"-la.

Sei que me dirão que é inverossímil: não é possível ser tão mau, tão
estúpido como eu; talvez ainda acrescentem, que era impossível não
amá"-la ou pelo menos não estimar esse amor. Mas por que impossível? Em
primeiro lugar, eu já não conseguia amar, porque, repito, amar para mim
significava tiranizar e ter a supremacia moral. Por toda a minha vida,
nunca pude sequer imaginar outro tipo de amor, e cheguei ao ponto de,
agora, às vezes pensar que o amor consiste na concessão voluntária, por
parte do objeto amado, do direito de tiranizá"-lo. Sequer em meus sonhos
de subsolo imaginava o amor de outra maneira que não como uma batalha;
começava sempre por ódio e terminava com uma submissão moral, sendo que
depois não podia sequer imaginar o que fazer com o objeto submetido. E
o que há de tão impossível, se eu já conseguira a tal ponto me
corromper moralmente, se estava a tal ponto desacostumado com a \textit{vida
real} que há pouco inventara de repreendê"-la e envergonhá"-la pelo fato
de ter vindo até mim para ouvir \textit{palavras de compaixão}; mas eu mesmo
não percebi que ela não viera em absoluto para ouvir palavras de
compaixão, mas para me amar, porque para a mulher é no amor que se
encerra toda a ressurreição, toda a salvação --- de qualquer tipo
possível de ruína --- e todo o renascimento, e que de outra maneira que
não essa não pode jamais surgir. Eu, aliás, não a odiara tanto enquanto
corria pelo quarto e a observava pela frestinha do biombo. Era"-me
apenas insuportavelmente penoso que ela estivesse ali. Eu queria que
ela desaparecesse. Era a \textit{tranquilidade} que eu desejava, ficar sozinho
no subsolo. A falta de costume com a \textit{vida real} fazia com que ela me
oprimisse a tal ponto que até respirar era difícil.

Mas alguns minutos mais se passaram, e ela continuava sem se levantar, como se
estivesse num torpor. Tive a desonestidade de bater de leve no biombo para
lembrá"-la\ldots{} Ela de repente agitou"-se, deu um salto do lugar em que estava
e pôs"-se a procurar seu xale, seu chapéu, o casaco, como que se apressando para
algum lugar, para se salvar de mim\ldots{} Depois de dois minutos, ela saiu
lentamente de detrás do biombo e olhou para mim gravemente. Dei um sorriso
maldoso, porém forçado, \textit{para manter as aparências}, e me virei para
evitar seu olhar.

--- Adeus --- disse ela dirigindo"-se à porta.

De repente, aproximei"-me dela correndo, peguei sua mão, abri"-a,
coloquei\ldots{} e depois novamente a fechei. Depois, imediatamente me virei
e me afastei correndo para o outro canto, para pelo menos não ver\ldots{}

Queria mentir nesse momento: escrever que eu fiz isso inadvertidamente,
fora de mim, desnorteado, por besteira. Mas não quero mentir e por isso
digo francamente que eu abri a mão e nela coloquei\ldots{} por raiva.
Ocorreu"-me fazer isso enquanto corria de um lado para o outro pelo
quarto e ela estava sentada do outro lado do biombo. Mas o que eu posso
dizer com segurança é o seguinte: fiz aquela crueldade, embora de
propósito, mas não com o coração, e sim com a minha cabeça ruim. Aquela
crueldade fora a tal ponto afetada, a tal ponto cerebral, inventada
propositalmente, \textit{livresca}, que eu mesmo não suportei um minuto
sequer: primeiro me afastei para um canto, para não ver, e depois, com
vergonha e desespero, lancei"-me ao encalço de Liza. Abri a porta que
dava para o saguão e comecei a ouvir atentamente.

--- Liza! Liza! --- gritei em direção à escada, mas de maneira hesitante, a
meia"-voz\ldots{}

Não houve resposta, pareceu"-me ter ouvido seus passos nos degraus
inferiores.

--- Liza! --- gritei mais alto.

Nenhuma resposta. Mas naquele mesmo momento, vindo de baixo, ouvi
abrir"-se, com dificuldade e um som estridente, a maciça porta de vidro
externa que dava para a rua, fechando"-se com força em seguida. Um eco
ergueu"-se pela escada.

Ela partira. Voltei para o quarto absorto em pensamentos. Sentia"-me
terrivelmente mal.

Parei junto à mesa ao lado da cadeira na qual ela se sentara, e olhei
inexpressivamente para o vazio. Um minuto se passou, e de repente estremeci por
inteiro: bem diante de mim, sobre a mesa, eu vi\ldots{} Resumindo, vi uma nota
azul de cinco rublos amassada, a mesma que um minuto atrás eu colocara em sua
mão. Era \textit{a mesma} nota; não podia ser outra; nem havia outra em casa.
Ela, portanto, tivera tempo de jogá"-la sobre a mesa no momento em que eu me
afastara para o outro canto.

E daí? Eu podia esperar que ela fizesse aquilo. Podia esperar? Não. Eu
era a tal ponto egoísta, a tal ponto não respeitava as pessoas na
realidade que sequer podia imaginar que ela fizesse aquilo. Não pude
suportar aquilo. Um momento depois, eu, como um louco, corri para me
trocar, vesti o que pude às pressas, e saí correndo atrás dela
precipitadamente. Ela ainda não conseguira se afastar nem duzentos
passos quando saí correndo pela rua.

Estava calmo, a neve caía copiosamente, caía de forma quase
perpendicular, estendendo um tapete sobre a calçada e a rua deserta.
Não havia nenhum transeunte, não se ouvia nenhum som. Os postes de luz
cintilavam, tristes e inúteis. Correndo, percorri duzentos passos até a
esquina e parei.

``Aonde ela foi? E por que eu estou correndo atrás dela? Por quê? Cair
diante dela, soluçar pelo arrependimento, beijar seus pés, suplicar o
perdão!'' Era isso que eu queria; meu peito parecia despedaçar"-se, e
jamais, jamais me lembrarei com indiferença daquele momento. ``Mas por
quê?'', pensava eu. ``Será que eu não passarei a odiá"-la, talvez amanhã
mesmo, justamente pelo fato de que hoje beijei seus pés? Será que darei
a ela felicidade? Será que não descobri hoje, novamente, pela centésima
vez, o quanto valho? Será que não a torturarei?!''

Fiquei parado na neve, fitando as trevas difusas e pensando nisso.

``E não será melhor, não será melhor'', fantasiava eu já em casa, mais
tarde, sufocando com fantasias a vívida dor que havia em meu peito,
``não será melhor se ela agora levar para sempre consigo a ofensa? A
ofensa é afinal uma purificação; é a mais mordaz e dolorosa tomada de
consciência! Amanhã mesmo eu sujaria sua alma e extenuaria seu coração.
Mas agora a ofensa jamais morrerá nela, e não importa quão abjeta seja
a sujeira que a aguarda, a ofensa haverá de elevá"-la e purificá"-la\ldots{}
Pelo ódio\ldots{} Hum\ldots{} Talvez também pelo perdão\ldots{} Mas será que tudo isso
fará as coisas melhores para ela?''

De fato, quero agora de minha parte fazer uma pergunta inútil: o que é
melhor, uma felicidade barata ou um sofrimento elevado? O que é
melhor, hein?

Era nisso que eu pensava naquela noite, sentado em minha casa e quase
morrendo de dor na alma. Nunca havia experimentado tamanho sofrimento e
tamanho arrependimento; mas poderia por acaso haver alguma dúvida,
quando saí correndo de meu apartamento, que eu voltaria do meio do
caminho para casa? Nunca mais encontrei Liza e não ouvi mais nada a
respeito dela. Devo também acrescentar que por muito tempo fiquei
satisfeito com a \textit{frase} a respeito da utilidade da ofensa e do
ódio, embora eu mesmo tenha quase adoecido na época de tanta tristeza.

Mesmo agora, depois de tantos anos, para mim é de certa forma muito
\textit{ruim} me lembrar de tudo isso. Para mim agora é ruim me lembrar
de muitas coisas, mas\ldots{} Não seria o caso de terminar por aqui essas
\textit{Memórias}? Creio que tenha cometido um erro ao começar a escrevê"-las.
Pelo menos senti vergonha o tempo todo que passei escrevendo esta
\textit{história}: já não é literatura, portanto, mas sim um castigo,
um corretivo. Porque contar, por exemplo, longas histórias a respeito
de como eu negligenciei minha vida pela corrupção moral de mim mesmo
num canto, pela carência do ambiente, pela falta de costume com a vida,
e por um rancor vaidoso no subsolo; meu Deus, como seria
desinteressante. Um romance precisa de um herói, sendo que aqui estão
reunidas \textit{propositalmente} todas as características de um
anti"-herói, e o principal é que tudo isso causa uma impressão das mais
desagradáveis, porque todos nós nos desacostumamos com a vida, somos
todos falhos, uns mais, outros menos. A tal ponto nos desacostumamos que
sentimos por vezes uma certa repulsa em relação à autêntica \textit{vida
real}, e por isso não suportamos quando nos lembram dela. Chegamos
afinal a tal ponto, que quase consideramos a autêntica \textit{vida real} um
esforço, quase um trabalho, e todos nós concordamos, no íntimo, que nos
livros é melhor. E por que nos remexemos às vezes, por que inventamos
tantas extravagâncias, pedimos tanta coisa? Nós mesmo não sabemos por
quê. Será ainda pior para nós se os nossos pedidos extravagantes forem
atendidos. Mas tentem, tentem dar"-nos mais independência, por exemplo,
desatem as mãos de qualquer um de nós, ampliem nosso círculo de
atividades, diminuam o controle, e nós\ldots{} Eu garanto a vocês: nós
imediatamente pediremos que aumentem novamente o controle. Sei que
vocês talvez se irritem comigo por conta disso, talvez gritem, batam o
pé e digam: ``Fale apenas de si mesmo e de sua miséria no subsolo, mas
não ouse dizer: \textit{todos nós}''. Permitam"-me,
senhores, não estou me justificando com essa \textit{generalização}. No
que me diz respeito particularmente, apenas levei às últimas
consequências em minha vida aquilo que vocês não ousaram levar sequer
até a metade, e ainda por cima tomaram como sensatez a sua covardia, e
se consolaram enganando a si mesmos. Com isso, pode ser que no fim eu
esteja mais \textit{vivo} que vocês. Pois observem mais atentamente! Porque
sequer sabemos onde reside agora esse tal viver e o que ele é, como se
chama. Deixem"-nos sozinhos, sem livros, e nós imediatamente ficaremos
confusos, perdidos: não saberemos a que aderir, o que seguir; o que
amar e o que odiar, o que respeitar e o que desprezar. Sentimos o peso
até mesmo de sermos humanos, humanos com um corpo e com sangue
verdadeiros, \textit{próprios}; temos vergonha disso, consideramos uma
desonra e fazemos de tudo para sermos algum tipo hipotético de homem
universal. Somos natimortos, e há tempos não temos nascido de pais
vivos, e isso nos agrada mais e mais. Tomamos gosto. Em breve daremos
um jeito de nascermos de ideias. Mas basta; não quero mais escrever \textit{do
subsolo}\ldots{}

Aqui, porém, ainda não acabam as \textit{memórias} deste paradoxista. Ele não
aguentou e continuou a escrever. Mas também nos parece que podemos
parar por aqui.


