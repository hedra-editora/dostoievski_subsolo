\textbf{Fiódor Mikháilovitch Dostoiévski} (1821--1881) é possivelmente o nome mais emblemático
da literatura russa. Nascido em Moscou, desde cedo demonstra interesse pelos livros, lendo não apenas os autores russos,
mas ingleses, alemães e franceses. Abatido com a morte súbita da mãe em 1837, abandona pouco depois a escola e ingressa 
no Instituto de Engenharia Militar de Nikolayev, onde se forma e passa a trabalhar como engenheiro. 
Para complementar a renda, começa a traduzir nas horas vagas. O grande sucesso de seu romance de estreia, \textit{Gente pobre}, 
de 1844, permite"-lhe participar dos círculos literários de Moscou. Em 1849, foi
preso e condenado à morte por sua participação no Círculo de Petrashevsky, 
espécie de sociedade secreta de aspirações utópicas e liberais, onde se discutia literatura também. 
A prisão revelou"-se uma farsa destinada a assustar os integrantes do círculo e sua pena foi comutada para 
trabalhos forçados na Sibéria. Durante esse período, seus surtos de epilepsia se tornam cada vez mais
frequentes. Ao retornar do exílio, foi aos poucos reconstruindo sua reputação de grande escritor, consolidada com a
publicação dos romances \textit{Crime e castigo}, \textit{O idiota} e \textit{Os irmãos Karamázov}. 
O legado de sua obra, marcado por um exame minucioso da condição humana e dos recessos mais sombrios da alma,
é composto por romances, novelas, contos e ensaios. Faleceu em São Petersburgo, a 28 de janeiro de 1881.

\textbf{Memórias do subsolo} (1864), considerada por muitos a primeira
novela existencialista, é um dos textos mais marcantes e influentes de Dostoiévski. 
Inserida no intenso debate da crítica literária russa do século~\textsc{xix}, a novela 
rapidamente angariou admiradores em todo o mundo --- de Sigmund Freud a Jean"-Paul Sartre ---,
graças à profundidade de sua cartografia psicológica e à pungência de seu estilo.
        
\textbf{Lucas Simone} formou"-se em História pela
Universidade de São Paulo. É tradutor e professor de língua russa,
tendo traduzido a peça \textit{Pequeno"-burgueses} e a coletânea de contos \textit{A velha
Izerguil e outros contos}, ambos de Maksim Górki (Hedra, 2010).
Traduziu ainda cinco contos para a \textit{Nova antologia do conto
russo} (1792--1998), organizada por Bruno Barretto Gomide (Editora 34,
2011), além do romance \textit{A aldeia de Stepántchikovo e seus
habitantes}, de Fiódor Dostoiévski (Editora 34, 2012).


